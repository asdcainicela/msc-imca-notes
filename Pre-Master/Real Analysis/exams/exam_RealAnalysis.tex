\documentclass{article}
\usepackage[utf8]{inputenc}
\usepackage[spanish]{babel}
\usepackage{amsmath, amssymb}
\usepackage{mathrsfs}
\usepackage{graphicx}
\usepackage{enumitem}
\usepackage{geometry}
\usepackage{xcolor}
\usepackage{tikz}
\usepackage{helvet}
\renewcommand{\familydefault}{\rmdefault} % Mantiene serif por default
%\usepackage{mathpazo} % Fuente math

\geometry{a4paper, margin=1in}
\setlength{\parindent}{0pt}

% Colores IMCA
\definecolor{imca-orange}{HTML}{F76707}
\definecolor{imca-light}{HTML}{444444}

% Comando para el logo IMCA
\newcommand{\IMCALogo}{
	\begin{tikzpicture}[baseline=(textnode.base)]
		\node[inner sep=0pt] (textnode) {
			\sffamily
			{\fontsize{40}{40}\selectfont\textcolor{imca-light}{I}}%
			{\fontseries{bx}\fontsize{40}{40}\selectfont\textcolor{imca-orange}{M}}%
			{\fontsize{40}{40}\selectfont\textcolor{imca-light}{CA}}%
		};
	\end{tikzpicture}
}

\begin{document}

\begin{flushleft}
	\begin{minipage}{0.4\linewidth}
		\IMCALogo
	\end{minipage}
	\begin{minipage}{0.5\linewidth}
		\begin{center}
			{\Large \textsc{  Análisis Real - Pre Maestría} }\\
			\large \textsc{Pre Maestría} \\
			\textsc{Verano 2025-0}
		\end{center}
	\end{minipage}
\end{flushleft}

\vspace{0.5cm}

Instrucciones:
\begin{itemize}[label=\textbullet]
	\item Duración: 180 minutos
	\item Cada problema tiene un valor de 5 puntos.
	\item El examen es personal y no se permite el uso de artículos con conexión a internet, libros o apuntes de clase.
\end{itemize}

\rule{\textwidth}{0.4mm}

\begin{enumerate}

	\item Dado \( X \subset \mathbb{R} \), para una aplicación \( f : X \to X \) y un punto \( x \in X \) considere la sucesión:
	      \begin{itemize}[label=\textbullet]
		      \item \( f^0(x) = x \),
		      \item Para cada entero positivo \( n \): \( f^n(x) = f \left( f^{n-1}(x) \right) \),
	      \end{itemize}

	      Para
	      \[ f(x) = \frac{3 + 3x}{3 + x}, \quad x \geq -\sqrt{3}, \]
	      y para cada \( x \) en el dominio de \( f \), determine el conjunto
	      \[ A(x) = \left\{ y \in \mathbb{R}: \lim_{k \to \infty} f^{n_k}(x) = y, \, \text{donde} (n_k) \subset \mathbb{N}, \, n_k < n_{k+1} \right\}. \]



	\item Sea \(\mathscr{C}\) el conjunto de las sucesiones de Cauchy de números racionales. Considere la relación sobre \(\mathscr{C}\) dada por

	      \[ (x_n) \sim (y_n) \iff \lim_{n \to \infty} (x_n - y_n) = 0. \]

	      Demuestre que
	      \begin{enumerate}[label=\alph*)]
		      \item \(\sim\) es una relación de equivalencia sobre \(\mathscr{C}\).
		      \item  Dados \( x = [x_n], y = [y_n] \) en el espacio cociente \(\mathscr{C}/\sim\), las operaciones

		            \[ x + y = [x_n + y_n], \quad x \cdot y = [x_n \cdot y_n] \]

		            están bien definidas.
		      \item Todo elemento \( x \) en \(\mathscr{C}/\sim\) cuyo representante no esté en la clase de la sucesión constante igual a cero, admite un inverso multiplicativo.

	      \end{enumerate}

	\item ¿Existe una función continua \( f : \mathbb{R} \to \mathbb{R} \) tal que

	      \[ f(f(x)) = e^{-x}, \]

	      para todo número real \( x \)?

	\item Sea \( f : [0, 2] \to \mathbb{R} \) una función continua. Demuestre que

	      \[ \int_0^2 [4f(x) - f^4(x)] \, dx \leq 6. \]
\end{enumerate}

\vspace{1cm}

\rule{\textwidth}{0.4mm}

\end{document}