% Conjuntos numéricos
\newcommand{\R}{\mathbb{R}}            % Conjunto de los números reales R
\newcommand{\N}{\mathbb{N}}            % Conjunto de los números naturales N
\newcommand{\Real}[1]{\mathbb{R}^{#1}} % Espacio euclidiano R^n dado un argumento n
\newcommand{\Rn}{\Real{n}}             % Notación abreviada para R^n
\newcommand{\Rm}{\Real{m}}             % Notación abreviada para R^m
\newcommand{\Rp}{\Real{p}}             % Notación abreviada para R^p
\newcommand{\Ou}{\mathds{O}}           % Letra O en doble trazo, requiere \usepackage{dsfont}

% Funciones especiales y operadores
\newcommand{\lcm}{\operatorname{lcm}}  % Mínimo común múltiplo
\newcommand{\dom}{\operatorname{dom}}  % Dominio de una función
\newcommand{\Dom}{\operatorname{Dom}}  % Variante con mayúscula, para uso formal
\newcommand{\ran}{\operatorname{ran}}  % Rango o imagen de una función

\newcommand{\Ll}{\mathscr{L}}          % Letra L caligráfica, espacio de operadores lineales
\newcommand{\Lineal}[2]{\Ll\left( #1, #2 \right)}  % Espacio de operadores lineales de #1 en #2

\newcommand{\inner}[2]{\langle #1, #2 \rangle}     % Producto interno usual ⟨x, y⟩

\newcommand{\interior}[1]{\operatorname{int}\left(#1\right)}
