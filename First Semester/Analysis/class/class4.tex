\chapter{Clase 4}
\clasedate{09 de abril de 2025}

\thmr{}{teorema4-1}{
	Sea  \( f :  X \subset \mathbb{R}^m \to \mathbb{R}^n \) una función definida por
	\[
		f = (f_1, f_2, \ldots, f_n)
	\]
	donde cada \( f_i : X \to \mathbb{R} \) es la componente escalar de \( f \). Entonces:

	\[
		f \text{ es continua en } a \in X \iff f_i \text{ es continua en } a \in X
	\]
}

\pf{ Prueba del teorema \ref{thm:teorema4-1}\\
	$(\Rightarrow)$ Supongamos que $f$ es continua,
	$$
		\longrightarrow \pi_i \circ f = f_i, \quad \forall i \in \{1, 2, \ldots, n\}
	$$
	$\pi_i \circ f$ es continua, pues es composición de continuas.

	$(\Leftarrow)$ Supongamos que $f_i \colon X \rightarrow \R$ es continua en $a\in X, \;\forall i \in \{1, 2, \ldots, n\} $

	Dado $\varepsilon >0$, como $\forall i \in \{1, \ldots, n\}, f_i \colon X \to \R$ es continua en $a \in X$\\
	$\exists \delta_i>0, \forall x \in X$
	$$
		\abs{x-a} < \delta_i \rightarrow \abs{f_i(x)-f(a)} <\varepsilon
	$$
	Sea $\delta = \min \{\delta_1, \delta_2, \ldots, \delta_n >0\}$
	$$
		\abs{x-a}<\delta\leq \delta_i \; \forall i \in \{1,\ldots, n\}
	$$
	$$
		\longrightarrow \forall i \in \{1, 2, \ldots, n\}, \abs{f_i(x)-f_i(a)}< \varepsilon
	$$
	$$
		\longrightarrow \max\limits_{1\leq i \leq n}\abs{f_i(x)-f_i(a)}< \varepsilon
	$$
	$$
		\longrightarrow\norm{f_i(x)-f_i(a)}_{\infty}< \varepsilon
	$$
	Así, $f$ es continua en $a \in X$.
}

\cor{
	Sea $X \subset \mathbb{R}^p$, sean $f \colon X \to \mathbb{R}^m$ y $g \colon X \to \mathbb{R}^n$. Consideramos la aplicación $(f,g) \colon X \to \mathbb{R}^m \times \mathbb{R}^n$ definida por $(f,g)(x) = (f(x), g(x))$.

	Entonces, $(f,g)$ es continua si y sólo si $f$ y $g$ son continuas.

	\rmk{Obs: $\mathbb{R}^m \times \mathbb{R}^n \cong \mathbb{R}^{m+n}$}
}

\corp{
	Sean $f \colon X \to \mathbb{R}^m$, $g \colon X \to \mathbb{R}^n$ y $\alpha \colon X \to \R$ funciones continuas. Entonces
	\begin{description}
		\item[(i)] $f+g$ es continua, $(f+g)(x) = f(x)+g(x)$
		\item[(ii)] $\alpha(x) \in \R, f(x)\in \Rn$
		      \begin{align*}
			      \alpha \cdot f \colon X & \longrightarrow \Rn                                   \\
			      x                       & \longmapsto (\alpha\cdot f)(x) = \alpha(x)\cdotp f(x)
		      \end{align*}
		      también es continua, $(f+g)(x) = f(x)+g(x)$
		\item[(iii)]
		      \begin{align*}
			      \inner{f}{g} \colon X & \longrightarrow \R                              \\
			      x                     & \longmapsto \inner{f}{g}(x) =\inner{f(x)}{g(x)}
		      \end{align*}
		\item[(iv)]
		      \begin{align*}
			      \frac{1}{\alpha} \colon X & \longrightarrow \R                                          \\
			      x                         & \longmapsto  \pqty{\frac{1}{\alpha}}(x)=\frac{1}{\alpha(x)}
		      \end{align*}
		      Es continua donde $\alpha(x)\neq 0, \forall x \in X$
	\end{description}
}{
	\begin{description}
		\item[(i)]
		      \begin{align*}
			      (f+g)(x) = f(x)+g(x) & = \pqty{f_1(x),\ldots, f_n(x)} + \pqty{g_1(x),\ldots, g_n(x)} \\
			                           & = \pqty{f_1(x)+g_1(x),\ldots, f_n(x)+g_n(x)}
		      \end{align*}
		      Como $f$ es continua $\rightarrow \forall i \in \{1,2, \ldots, n\}, f_i\colon X \to \R $ es continua.\\
		      Como $g$ es continua $\rightarrow \forall i \in \{1,2, \ldots, n\}, g_i\colon X \to \R $ es continua.\\
		      Queda demostracion para el lector, $\rightarrow \forall i \in \{1,2, \ldots, n\}, f_i+g_i  $ es continua.\\
		      $$
			      \rightarrow f+g \text{ es continua.}
		      $$
		\item[(ii), (iii), (iv)]
		      Sean,
		      \begin{align*}
			      S \colon \Rn \times \Rn & \longrightarrow\Rn      \\
			      (x,y)                   & \longmapsto S(x,y)= x+y
		      \end{align*}
		      $S$ es una transformación lineal,
		      \begin{itemize}
			      \item Adición, \begin{align*}
				            S((x,y)+(x',y')) & =S(x+x', y+y')               \\
				                             & =x+x' {\color{blue} + } y+y' \\
				                             & =x+x'+ y+y'                  \\
				                             & =S(x,y)+S(x',y')
			            \end{align*}
			      \item Multiplicación por un escalar, \begin{align*}
				            S(\lambda(x,y) ) & =S(\lambda x,\lambda x') \\
				                             & =\lambda x+ \lambda y    \\
				                             & =\lambda (x+y)           \\
				                             & =\lambda S(x,y)
			            \end{align*}
		      \end{itemize}
		      $S$ es continua ya que es una transformación lineal.\\
		      $\phi $ y $\xi$ son bilineales $\Rightarrow$ $\phi$ y $\xi$ son continuas.
		      \begin{align*}
			      \phi \colon \R  \times \Rn & \longrightarrow\Rn                     \\
			      (\lambda,x)                & \longmapsto \phi(\lambda,x)= \lambda x
		      \end{align*}
		      \begin{align*}
			      \xi \colon \Rn  \times \Rn & \longrightarrow\Rn                                            \\
			      (x,y)                      & \longmapsto \xi(x,y)= \inner{x}{y} = \sum_{i = 1}^{n} x_i y_i
		      \end{align*}
		      donde $x=(x_1, \ldots, x_n)$ e $y=(y_1, \ldots, u_n)$.

		      $\rho $ es continua,
		      \begin{align*}
			      \rho \colon \R-\{0\}  \R & \longrightarrow\Rn               \\
			      t                        & \longmapsto \rho(t)= \frac{1}{t}
		      \end{align*}
		      $$
			      \bqty{S\circ (f,g)}(x)\qty((f,g)(x))= S\pqty{(f(x), g(x))} = f(x)+g(x) = (f+g)(x), \forall x \in X
		      $$
		      $\longrightarrow S\circ(f,g) =f+g $, $S\circ(f,g)$ es continua, pues es composición de continuas.
		      \begin{align*}
			      \bqty{\varphi \circ (\alpha, f)}(x) = \varphi((\alpha, f)(x)) & = \varphi(\alpha(x), f(x))      \\
			                                                                    & =\alpha(x) f(x)                 \\
			                                                                    & =(\alpha f)(x), \forall x \in X
		      \end{align*}
		      $\rightarrow \varphi \circ(\alpha, f) = \alpha f $, $ \varphi \circ(\alpha, f)$ es continua ya que $\varphi$ es continua por ser bilineal $(\alpha, f)$ es continua pues cada coordenada $\alpha \colon X\to \R$ y $f \colon X \to \Rn$ son continuas.

	\end{description}
}

\vspace{-2em}
\begin{figure}[H]
	\centering
	\includegraphics[width=0.5\linewidth]{img/class4_figure1.png}
\end{figure}
\vspace{-1em}

\thmr{}{teorema4-2}{
	Sea $X \subset \mathbb{R}^m$ y sea $f \colon X \to \mathbb{R}^n$ una función.

	Entonces, $f$ es continua en $a \in X$ si y sólo si, para toda sucesión $(x_k)_{k \in \mathbb{N}} \subset X$ tal que $\lim\limits_{k \to \infty} x_k = a$, se cumple que:
	\[
		\lim_{k \to \infty} f(x_k) = f(a).
	\]
}
\pf{Prueba del teorema \ref{thm:teorema4-2}\\
	($\Rightarrow$) Supongamos que $f$ es continua en $a \in X$.\\
	Dado
	$$
		\varepsilon>0, \exists \delta >0, \forall x \in X, \abs{x-a}<\delta \rightarrow\abs{f(x)-f(a)}<\varepsilon
	$$
	Sea $(x_k)_{k\in \N} \subset X$ tal que $\lim\limits_{k\to \infty} x_k = a$

	Como $\lim\limits_{k \to \infty} x_k =a$, para $\delta>0$ anterior, $\exists k_0 \in \N$,
	$$
		\forall k \in \N, k\geq k_0 \rightarrow \abs{x_k-a}<\delta \rightarrow \abs{f(x_k)-f(a)}< \varepsilon
	$$
	$$
		\therefore \lim\limits_{k \to \infty} f(x_K) = f(a)
	$$

	($\Leftarrow$) Supongamos que $\forall (x_k)_{k\in \N} \subset X, \lim\limits_{k \to \infty} x_k =a \rightarrow  \lim\limits_{k \to \infty} f(x_k) =f(a) $\\
	Mostremos que $f$ es continua en $a \in X$.\\
	Supongamos que $f$ no es continua en $a \in X$
	$$
		\sim \bqty{ \forall\varepsilon>0, \exists \delta >0, \forall x \in X, \abs{x-a}<\delta \rightarrow\abs{f(x)-f(a)}<\varepsilon }
	$$
	$$
		\equiv \exists \varepsilon_0, \forall \delta>0, \exists x_{\delta} \in X, \abs{x_{\delta}-a}<\delta \wedge \abs{f(x_{\delta} -f(a))} \geq \varepsilon_0
	$$
	$$
		\forall k \in \N, \delta_k = \frac{1}{k}>0, \exists x_k \in X, \abs{x_k-a}<\frac{1}{k} \wedge \abs{f(x_k)-f(a)}\geq \varepsilon_0
	$$
	Así, $\lim\limits_{k \to \infty} \abs{x_k-a}=0$
	$$
		\lim\limits_{k \to \infty}  x_k=a \wedge \forall k \in  \N, \abs{f(x_k)-f(a)}\geq \varepsilon_0
	$$
	de la hipótesis, $	\lim\limits_{k \to \infty}  f(x_k)=f(a) \rightarrow \lim\limits_{k \to \infty}  \abs{f(x_k)-f(a)}=0  $\\
	$\rightarrow 0 \geq \varepsilon_0>0$ (Contradicción)\\
	Por lo tanto $f$ es continua en $a \in X$.
}

%-------------------------------

Sea \( f \colon X \subset \mathbb{R}^m \to \mathbb{R}^n \)

\begin{itemize}
	\item[\textbf{1.}] \( f \) es continua si:
	      \[
		      \forall a \in X, \quad f \text{ es continua en } a
	      \]
	      Es decir:
	      \[
		      \forall a \in X, \quad \forall \varepsilon > 0, \; \exists \delta > 0, \; \forall x \in X, \; \lvert x - a \rvert < \delta \Rightarrow \lvert f(x) - f(a) \rvert < \varepsilon
	      \]
	      \textcolor{orange}{\(\delta = \delta(a, \varepsilon)\)} $\rightarrow$ \textit{\(\delta\) depende de \(a\) y \(\varepsilon\)}.

	\item[\textbf{2.}] \( f \) es \textbf{uniformemente continua} si:
	      \[
		      \forall \varepsilon > 0, \; \exists \delta > 0, \; \forall x, a \in X, \; \lvert x - a \rvert < \delta \Rightarrow \lvert f(x) - f(a) \rvert < \varepsilon
	      \]
	      \textcolor{orange}{\(\delta = \delta(\varepsilon)\)} $\rightarrow$ \textit{\(\delta\) solo depende de \(\varepsilon\)}.

	\item[\textbf{3.}] \( f \) es \textbf{Lipschitziana} si:
	      \[
		      \exists C > 0 \text{ tal que } \forall x, y \in X, \quad \lvert f(x) - f(y) \rvert \leq C \lvert x - y \rvert
	      \]
	      \textcolor{green}{\(C\) es una constante de Lipschitz de \(f\)}.

	      \rmkb{
		      \textbf{ Afirmación:}  Toda función Lipschitziana es uniformemente continua.

		      \textbf{Demostración:}

		      Sea \( f \colon X \subset \mathbb{R}^m \to \mathbb{R}^n \) Lipschitziana. Entonces:
		      \[
			      \exists C > 0 \text{ tal que } \forall x, y \in X, \quad \lvert f(x) - f(y) \rvert \leq C \lvert x - y \rvert
		      \]

		      Dado \( \varepsilon > 0 \), tomamos \( \delta = \frac{\varepsilon}{C} > 0 \). Entonces:
		      \[
			      \forall x, y \in X, \; \lvert x - y \rvert < \delta \Rightarrow C \lvert x - y \rvert < \varepsilon \Rightarrow \lvert f(x) - f(y) \rvert < \varepsilon
		      \]

		      Por lo tanto, \(f\) es uniformemente continua.
	      }
\end{itemize}



\propp{
	Sea \(f\colon X \subset \mathbb{R}^m \to \mathbb{R}^n\).
	Si \(f\) es uniformemente continua, entonces
	\[
		\forall\, (x_k)_{k\in\mathbb{N}} \subset X,
	\]
	si $(x_k)_{k \in \N}$ es de Cauchy, la sucesión
	\[
		\bigl(f(x_k)\bigr)_{k\in\mathbb{N}} \subset \mathbb{R}^n
	\]
	también es de Cauchy.
}{ (Suponga que $f \colon X \subset \Rm \to \Rn$ es uniformemente continua y sea $(x_k)_{k\in \N} \subset X$ una sucesión de Cauchy.\\
	Mostremos que $(f(x_k))_{k \in \N}$ es de cauchy.
	\begin{equation}
		\text{Dado } \varepsilon>0, \exists \delta >0, \forall x, w \in X \text{ tal que } \abs{x-w}< \delta \rightarrow \abs{f(x)-f(w)} < \varepsilon \label{eq:proposion-1}
	\end{equation}
	Como $(x_k)_{k \in \N}$ es de Cauchy, entonces para $\delta >0$ anterior, $\exists k_0 \in \N, \forall k, p \in \N$,
	$$
		k, p \geq k_0 \rightarrow \abs{ x_k -x_p} < \delta
	$$
	Por la uniformidad de \(f\) (según la ecuación~\eqref{eq:proposion-1}), se tiene:
	\[
		\lvert f(x_k) - f(x_p) \rvert < \varepsilon.
	\]
	Por lo tanto, \((f(x_k))_{k \in \mathbb{N}}\) es una sucesión de Cauchy en \(\mathbb{R}^n\).
}

\expf{
	¿Es cierto el recíproco de la proposición anterior? Proporciona una prueba o un contraejemplo.
}{\\
	Por contraejemplo\\
	Agregar grafico!!!!!!!!

	Sea
	\[
		f\colon \mathbb{R}\setminus\{0\} \;\longrightarrow\; \mathbb{R},
		\qquad
		f(x)=\frac1x.
	\]
	Consideremos la sucesión
	\[
		x_k = \frac1k,\quad k\in\mathbb{N}.
	\]
	Entonces \((x_k)\) es de Cauchy en \(\mathbb{R}\setminus\{0\}\), pero
	\[
		f(x_k) = k,
	\]
	que no es de Cauchy en \(\mathbb{R}\). Por tanto, \(f\) no es uniformemente continua.
}

\ex{
	Demostrar que
	\[
		f\colon X\subset \mathbb{R}^m \;\longrightarrow\; \mathbb{R}^n
	\]
	es uniformemente continua si y solo si para cualesquiera dos sucesiones
	\((x_k)\), \((y_k)\subset X\) tales que
	\(\lim_{k\to\infty}\|x_k - y_k\| = 0\),
	se tiene
	\(\lim_{k\to\infty}\|f(x_k) - f(y_k)\| = 0\).
}
%---------------

\section{Límites}

Sean $f \colon X \subset \Rm \to \Rn, a \in X'$ y $b \in \Rn$
\begin{itemize}
	\item Decimos que el límite de $f(x)$ es $b$ cuando $x$ tiene para $a$ si ocurre lo siguiente:
	      $$
		      \forall \varepsilon >0, \exists \delta >0, \forall x \in X, 0 < \abs{x-a} < \delta \rightarrow \abs{f(x)-b}<\varepsilon
	      $$
	      donde $d(x,a)=\abs{x-a}$  y $d(f(x),b) = \abs{f(x)-b}$\\
	      \rmk{Notación: $\lim\limits_{x \to a} f(x) = b$}
\end{itemize}
Corregir dibujo! [ver codigo en python]
\vspace{-1em}
\begin{figure}[H]
	\centering
	\includegraphics[width=0.5\linewidth]{img/class4_figure2.png}
\end{figure}
\vspace{-1em}
Dado $\varepsilon>0, \exists \delta>0, \forall x \in X$ si $x \in ] a-\delta, a+\delta [\backslash \{a\} \rightarrow f(x) \in ]b-\varepsilon, b+\varepsilon[$ es equivalente a
$$
	\varepsilon>0, \exists \delta>0, \forall x \in X, 0<\abs{x-a} < \delta \rightarrow \abs{f(x) - b} \varepsilon
$$
\rmkb{ Sean $f\colon X \subset \Rm \to \Rn, \, a \in X' \text{ y } b, c \in \Rn$\\
	Si $\lim\limits_{x \to a} f(x) =b$ y $\lim\limits_{x \to a} f(x) = c$, entonces $b=c$
	\pf{
		Dado $\varepsilon >0$,
		\begin{align*}
			\exists \delta >0, \; \forall x \in X, 0\abs{x-a} < \delta \rightarrow \abs{f(x)-b} < \varepsilon \\
			\exists \eta >0, \;\forall x \in X, 0\abs{x-a} < \eta \rightarrow \abs{f(x)-b} < \varepsilon
		\end{align*}
		Sea $\delta_0 =\min\{\delta, \eta\}>0$, si $0<\abs{x-a}<\delta_0 \rightarrow \abs{b-c} \leq \abs{b-f(x)}+\abs{f(x)-c}<2\varepsilon$
		\begin{align*}
			 & \text{Por tanto, } \lvert b - c \rvert < 2\varepsilon, \quad \forall \varepsilon > 0.              \\
			 & \text{Pero esto implica que } \lvert b - c \rvert \leq \inf\{2\varepsilon : \varepsilon > 0\} = 0. \\
			 & \text{Es decir, } \lvert b - c \rvert = 0 \Rightarrow b = c.
		\end{align*}

	}
}

\


\thmrpf{}{teorema4-20}{
	Sea $f \colon X \subset \mathbb{R}^m \to \mathbb{R}^n$ tal que $f = (f_1, \ldots ,f_n)$. Además $a\in X'$ y $b= (b_1, \ldots , b_n) \in \Rn$. Entonces
	\[
		\lim_{x\to a} f(x) \;=\; b
	\]
	si y solo si
	\[
		\lim_{x\to a} f_i(x) \;=\; b_i,
		\quad
		\text{para todo } i=1,2,\dots,n.
	\]
}{
	%Prueba del teorema \ref{thm:teorema4-20}\\

	($\Rightarrow$)
	Ejercicio para el lector.

	($\Leftarrow$)
	Ejercicio para el lector.
}
