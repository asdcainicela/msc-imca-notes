\chapter{Clase 7}
\clasedate{21 de abril de 2025}
\section{Conjuntos compactos}

\[
	K \subset \mathbb{R}^n \text{ es compacto si y solo si } K \text{ es cerrado y acotado}
\]
\[
	\overline{K} = K
\]

\propp{
Sea \( X \subset \mathbb{R}^n \). Entonces, \( X \) es \textbf{sucesionalmente compacto} \(\Leftrightarrow\)
\begin{equation}
	X \text{ es compacto} \iff \text{Toda sucesión } (x_k)_{k \in \mathbb{N}} \subset X \text{ tiene una subsucesión convergente a un punto de } X.
	\label{eq:class7-1}
\end{equation}
}{
(\( \Rightarrow \))


Supongamos que \( X \) es compacto \( \Rightarrow X \) es cerrado y acotado.

Sea \( (x_k)_{k \in \mathbb{N}} \subset X \)

\[
	\Rightarrow (x_k)_{k \in \mathbb{N}} \text{ es acotada}
\]

Por el Teorema de Bolzano–Weierstrass:

\[
	\exists i : \mathbb{N} \to \mathbb{N} \text{ estrictamente creciente y } \exists w_0 \in \mathbb{R}^n \text{ tal que }
\]

\[
	\lim_{k \to \infty} x_{i(k)} = w_0 \in \mathbb{R}^n
\]

Como \( X \) es cerrado y \( (x_k) \subset X \)

\[
	\Rightarrow w_0 \in   \overline{X} = X
\]

(\( \Leftarrow \)) Supongamos \eqref{eq:class7-1}

Sea \( w \in \overline{X} \). Entonces existe una sucesión \( (x_k)_{k \in \mathbb{N}} \subset X \) tal que
\begin{equation}
	\lim_{k \to \infty} x_k = w \in \mathbb{R}^n.
	\label{eq:class7-2}
\end{equation}


Como $(x_k)_{k \in \N} \subset X,$ por \eqref{eq:class7-1}.

$\exists j \colon \N \rightarrow \N$ estricatamente creciente y $\exists z \in X$ tal que $\lim_{k \to \infty}  x_{j(k)} = z \in X$

De \eqref{eq:class7-2}, $\lim_{k \to \infty} x_{j(k)} = w$

Por la unicidad del límite $w = z \in X$

$$
	\therefore \overline{X}  \subset X
$$
Así, $X$ es cerrado

Mostremos que $X$ es acotado $\rightarrow \exists c>0, \forall z \in X, \abs{z} \geq c$

Supongamos que \( X \) no es acotado. Entonces:
\begin{equation}
	\forall k \in \mathbb{N}, \exists z_k \in X \text{ tal que } \|z_k\| > k.
	\label{eq:class7-3}
\end{equation}


Hemos construido $(z_k)_{k \in \N} \subset X$ tal que $$
	\lim_{k \to \infty} \abs{z_k} = +\infty$$

Dado  $i \colon \N \to \N$ estrictamente creciente. $i(k) \leq k, \forall k \in \N$

$$
	\forall k \in \N, \quad \abs{Z_{i(k)}} > i(k)
$$
$$
	\lim_{k \to \infty} \abs{Z_{i(k)}} = +\infty
$$
Así $(z_{i(k)})_{k \in \N} $ no converge, esto contradice a \eqref{eq:class7-1}.

}

\rmkb{En espacios métricos, el hecho de ser cerrado y acotado \textbf{no implica} ser sucesionalmente compacto. Esta propiedad es equivalente a la compacidad solo en \(\mathbb{R}^n\).}

\propp{
Sea \( \{K_p\}_{p \in \N} \) una colección de subconjuntos compactos de \( \R^n \) tal que:
\[
	K_p \supset K_{p+1}, \quad \forall p \in \N, \qquad K_p \neq \varnothing
\]
Entonces:
\[
	K := \bigcap\limits_{p \in \N} K_p \text{ es compacto y } K \neq \varnothing
\]
}{
\begin{itemize}
	\item \( \bigcap\limits_{p \in \N} K_p \subset K_1 \Rightarrow \bigcap\limits_{p \in \N} K_p \) es acotado.
	\item \( \bigcap\limits_{p \in \N} K_p \) es cerrado, pues cada \( K_p \) es cerrado.
\end{itemize}
\[
	\Rightarrow \bigcap\limits_{p \in \N} K_p \text{ es compacto}
\]

\textbf{Mostremos que \( K \neq \varnothing \colon \)}

Como \( K_p \neq \varnothing, \ \forall p \in \N \), fijamos \( x_p \in K_p \subset K_1 \Rightarrow (x_p)_{p \in \N} \subset K_1 \)

\[
	\Rightarrow (x_p)_{p \in \N} \text{ es acotada}
	\Rightarrow \exists i \colon \N \to \N \text{ estrictamente creciente tal que }
	\lim\limits_{p \to \infty} x_{i(p)} = z_0 \in K_1
\]

Tomemos \( (x_{i(p)})_{p \geq 2} \). Como \( x_{i(p)} \in K_{i(p)} \subset K_2 \) y \( i(p) \leq i(2) \leq 2 \), para todo \( p \leq 2 \), entonces:

\[
	\lim_{\substack{p \to \infty \\ p \leq 2}} x_{i(p)} = z_0, \quad x_{i(p)} \in K_2
\]

Fijemos \( k \in \N \). Como \( x_{i(p)} \in K_{i(p)} \subset K_k \) para \( p \leq k \), se tiene:

\[
	\lim_{\substack{p \to \infty \\ p \leq k}} x_{i(p)} = z_0 \in \overline{K_k} = K_k
\]

\[
	\therefore \forall k \in \N, \quad z_0 \in K_k
	\Rightarrow z_0 \in \bigcap\limits_{k \in \N} K_k = K
	\Rightarrow K \neq \varnothing
\]
}

\section{Cubrimiento}

Sea \( \{C_{\lambda}\}_{\lambda \in L} \) una familia de subconjuntos de \( \mathbb{R}^n \).

Decimos que \( \{C_{\lambda}\}_{\lambda \in L} \) es un \textbf{cubrimiento} de un conjunto \( X \subset \mathbb{R}^n \) si:
\[
	X \subset \bigcup\limits_{\lambda \in L} C_{\lambda}
\]

Un \textbf{subcubrimiento finito} de \( \{C_{\lambda}\}_{\lambda \in L} \) es una subfamilia \( \{C_{\lambda}\}_{\lambda \in L'} \), donde \( L' \subset L \) es un conjunto finito, tal que:
\[
	X \subset \bigcup\limits_{\lambda \in L'} C_{\lambda}
\]

\thmrpf{}{}{
Sea \( K \subset \mathbb{R}^n \) un conjunto compacto.

Entonces, todo cubrimiento abierto de \( K \) admite un subcubrimiento finito.
}{
Sea \( \{A_{\lambda}\}_{\lambda \in L} \) un cubrimiento abierto de \( K \), es decir:
\[
	K \subset \bigcup\limits_{\lambda \in L} A_{\lambda}, \quad A_{\lambda} \subset \mathbb{R}^n \text{ abierto } \forall \lambda \in L.
\]

i.e   \( \forall \lambda \in L, \{A_{\lambda}\}_{\lambda \in L} \) ees abierto de $\Rn$, $$K \subset \bigcuplim{\lambda \in L} A_{\lambda}$$

\lem{Lindelöf}{
Sea \( X \subset \mathbb{R}^n \) y sea \( \{C_{\lambda}\}_{\lambda \in \Lambda} \) un cubrimiento abierto de \( X \), es decir,
\[
	X \subset \bigcup\limits_{\lambda \in \Lambda} C_{\lambda}, \quad C_{\lambda} \text{ abierto } \forall \lambda \in \Lambda.
\]
Entonces, existe un subconjunto numerable \( \Lambda_0 \subset \Lambda \) tal que
\[
	X \subset \bigcup\limits_{\lambda \in \Lambda_0} C_{\lambda}.
\]
}

Por (Lindelöf), $\exists \Lambda_0 = \{\lambda_i \in \Lambda \colon i \in \N\} \subset\Lambda$, $\Lambda$ es numerable.

Tal que
\begin{equation}
	K \subset \bigcuplim{\lambda \in \Lambda_0} A_{\lambda} = \bigcuplim{i \in \N} A_{\lambda_{i}}
	\label{eq:class7-4}
\end{equation}
Sea
\begin{align*}
	F_0    & = K \supset F_1 \supset F_2 \supset \ldots                                                                                       \\
	F_1    & = K-A_{\lambda_1} = K \cap \bqty{A_{\lambda_1}^{C}}                                                                              \\
	F_2    & = K-\bqty{A_{\lambda_1} \cup A_{\lambda_2}} = K \cap \bqty{A_{\lambda_1} \cup A_{\lambda_2}}^{C}                                 \\
	\vdots & =\qquad \vdots\hspace{3cm} \vdots                                                                                                \\
	F_m    & =  K-\bqty{ \bigcuplim{i = 1 }{m} A_{\lambda_i} } = K \cap \bqty{ \bigcuplim{i = 1 }{m} A_{\lambda_i} }^{C} \quad \text{Cerrado}
\end{align*}
Tenemos
\[
	F_0 \supset F_1 \supset F_2 \supset F_3 \supset \ldots \supset F_m \supset \ldots; \qquad F_0 \text{ acotado } \Rightarrow F_m \text{ es acotado } \forall m \in \mathbb{N}
\]
donde cada \( F_m \) es cerrado como intersección de cerrados, y acotado, luego \textbf{compacto}  en \( \mathbb{R}^n \).

Supongamos \( F_m \neq \varnothing \) para todo \( m \in \mathbb{N} \), y que \( F_m \supset F_{m+1} \), entonces, por el teorema del conjunto descendente de compactos no vacíos:
\[
	\bigcap_{m \in \mathbb{N}} F_m \neq \varnothing
\]
Pero:
\begin{align*}
	\bigcap_{m \in \mathbb{N}} F_m
	 & = \bigcap_{m \in \mathbb{N}} \left( K \cap \left( \bigcup_{i = 1}^{m} A_{\lambda_i} \right)^C \right) \\
	 & = K \cap \bigcap_{m \in \mathbb{N}} \left( \bigcup_{i = 1}^{m} A_{\lambda_i} \right)^C                \\
	 & = K \cap \left( \bigcup_{m \in \mathbb{N}} \bigcup_{i = 1}^{m} A_{\lambda_i} \right)^C                \\
	 & = K \cap \left( \bigcup_{i \in \mathbb{N}} A_{\lambda_i} \right)^C                                    \\
	 & = K - \bigcup_{i \in \mathbb{N}} A_{\lambda_i}                                                        \\
	 & = K - K \quad \text{(por \eqref{eq:class7-4})}                                                        \\
	 & = \varnothing
\end{align*}
\[
	\Rightarrow \bigcap_{m \in \mathbb{N}} F_m = \varnothing \quad \text{contradicción}
\]

Por lo tanto, existe \( m_0 \in \mathbb{N} \) tal que \( F_{m_0} = \varnothing \), es decir:
\[
	K - \bigcup_{i=1}^{m_0} A_{\lambda_i} = \varnothing \Rightarrow K \subset \bigcup_{i=1}^{m_0} A_{\lambda_i}
\]

\[
	\therefore \text{Existe un subcubrimiento finito de } \{A_\lambda\}_{\lambda \in L} \text{ que cubre } K.
\]
}

\thmrpf{}{}{
	\begin{align}
		\text{Si todo cubrimiento abierto de } \mathbb{R}^n \text{ admite algún subcubrimiento finito,} \label{eq:class7-5} \\
		\text{entonces } K \text{ es compacto.} \nonumber
	\end{align}
}{
	Sea $K \subset \Rn$ tales que verifica \eqref{eq:class7-5}.
	\begin{itemize}
		\item Mostremos que $K$ es acotado.
		      $$
			      K \subset \bigcuplim{a \in K} B(a,1), \quad \text{abierto en } \Rn
		      $$
		      Entonces, gracias a \eqref{eq:class7-5} existen $a_1, a_2, \ldots, a_n \in K$\\
		      tales que
		      $$
			      K \subset \bigcuplim{i =1}{m} B(a_i, 1)
		      $$
		      donde $z \in K \rightarrow \exists i \in \{1, \ldots, m\}$ tal que $z \in B(a_i, 1)$\\
		      Sea $L = \max\limits_{1 \leq i \leq m} \{\abs{a_1}\}+1$, además $\abs{z-a_i} <1$
		      $$
			      \abs{z } \leq \abs{z-a_i} +\abs{a_i} < L
		      $$
		      $$
			      \therefore K \text{ es acotado}
		      $$
		\item Mostremos que $K$ es cerrado, $\overline{K} \subset K$.

		      Suponga que $K$ no es acotado ( $\overline{K} \not\subset K $)
		      \begin{equation}
			      \Rightarrow \exists w_0 \in \overline{K}, \quad w_0 \not\in K
			      \label{eq:class7-6}
		      \end{equation}
		      $$
			      \rightarrow\{w_0\} \cap K = \varnothing
		      $$
		      $$
			      \rightarrow K \subset \Rn-\{w_0\}
		      $$
		      \rmkb{
			      $$
				      \{w_0\} = \bigcaplim{p \in  \N} B\qty[w_0, \frac{1}{p}]
			      $$
		      }
		      $$
			      \Rightarrow K \subset\Rn-\{w_0\} = \bigcuplim{p \in \N}\bqty{\Rn-B\bqty{w_0, \frac{1}{p}}}
		      $$
		      donde $\Rn-B\bqty{w_0, \frac{1}{p}}  $ es abierto.
		      $$
			      \Rightarrow K \subset  \bigcuplim{p \in \N}\bqty{\Rn-B\bqty{w_0, \frac{1}{p}}}
		      $$
		      Por \eqref{eq:class7-5}\; $\exists p_1, p_2, \ldots, p_l \in \N$, además podemos suponer $p_1 < p_2< \ldots < p_l$
		      $$
			      K \subset  \bigcuplim{p \in \N}{l}\bqty{\Rn-B\bqty{w_0, \frac{1}{p_l}}} = \Rn-B\bqty{w_0, \frac{1}{p_l}}
		      $$
		      $$
			      \Rightarrow K \cap B\bqty{w_0, \frac{1}{p_l}} = \varnothing \qquad (\rightarrow \leftarrow)
		      $$
		      Esto contradice \eqref{eq:class7-6}, pues $w_0 \in \overline{K}$.

	\end{itemize}
}

\thmrpf{Lindelöf}{}{
Sea $X \subset \Rn$. Todo cubrimiento abierto del conjunto $X$ admite algún subcubrimiento numerable.
}{
Por lo visto antes, existe $E \subset X$ numerable y dentro de $X$, $X \subset \overline{E}$.

\noindent Sea $E = \{x_i\}_{i \in \N}$\\
Sea $\{A_{\lambda}\}_{\lambda \in \Lambda} $ un cubrimiento abierto de $X \rightarrow X \subset \bigcuplim{\lambda \in \Lambda} A_{\lambda}$\\
Sea $\mathcal{F} = \qty{ B\pqty{x_i, \frac{1}{i}}}_{(i, j) \in \N \times \N \text{(numerable)}} $
\begin{align*}
	\exists \varphi \colon \N & \longrightarrow \N \times \N \text{ biyección} \\
	k                         & \longmapsto \varphi(k) = (i_k, i_k)
\end{align*}
Sea $\mathcal{F}$ una función numerable de bolas abiertas
$$
	\mathcal{F}  = \{B_k\}_{k \in \N} \quad\text{tal que}\quad B_k = B\pqty{x_{i_k}, \frac{1}{i_{k}}}
$$
$$
	\widehat{\N} = \{k \in \N \colon B_k \subset A_{\lambda}\; \text{para algún}\;  \lambda \in \Lambda\}
$$

\clmp{}{
	Sea $B_k \subset A_{\lambda_k}$
	$$
		X \subset \bigcuplim{k \in \widehat{\N}} B_k\subset \bigcuplim{k \in \widehat{\N}} A_{\lambda_k}
	$$
}{
	Sea $x \in X$. Como,
	$$
		X \subset \bigcuplim{k \in \widehat{\N}} A_{\lambda_k} \rightarrow \exists \lambda_0 \in \Lambda \text{ tal que } x \in A_{\lambda_0}
	$$
	donde $A_{\lambda_0} $ es abierto.

	Así $\exists \varepsilon >0$ tal que $B(x, \varepsilon) \subset A_{\lambda_0} $
	$$
		x \in X \subset \overline{E}
	$$

	Por el principio Arquimediano $$\exists j_0 \in \N \quad \text{tal que}\quad \frac{1}{j_0} < \frac{\varepsilon}{2}
	$$
	Como $x  \in X \subset \overline{E} \rightarrow B\pqty{x, \frac{1}{j_0}} \cap E \neq \varnothing$
	%A reference to %Theorem~\ref{thm:mybigthm}
}



}
