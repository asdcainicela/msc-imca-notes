\chapter{Clase 7}
\clasedate{21 de abril de 2025}
\section{Conjuntos compactos}

\[
	K \subset \mathbb{R}^n \text{ es compacto si y solo si } K \text{ es cerrado y acotado}
\]
\[
	\overline{K} = K
\]

\propp{
Sea \( X \subset \mathbb{R}^n \). \( X \) es \textbf{sucesionalmente compacto} \(\Leftrightarrow\)

{	\( X \) es compacto si y solo si toda sucesión \( (x_k)_{k \in \mathbb{N}} \) en \( X \) posee alguna subsucesión convergente a un punto de \( X \).
		\label{eq:class7-1}}

}{
(\( \Rightarrow \))


Supongamos que \( X \) es compacto \( \Rightarrow X \) es cerrado y acotado.

Sea \( (x_k)_{k \in \mathbb{N}} \subset X \)

\[
	\Rightarrow (x_k)_{k \in \mathbb{N}} \text{ es acotada}
\]

Por el Teorema de Bolzano–Weierstrass:

\[
	\exists i : \mathbb{N} \to \mathbb{N} \text{ estrictamente creciente y } \exists w_0 \in \mathbb{R}^n \text{ tal que }
\]

\[
	\lim_{k \to \infty} x_{i(k)} = w_0 \in \mathbb{R}^n
\]

Como \( X \) es cerrado y \( (x_k) \subset X \)

\[
	\Rightarrow w_0 \in   \overline{X} = X
\]

(\( \Leftarrow \)) Supongamos \eqref{eq:class7-1}

Sea $w \in \overline{X} \to \exists  (x_k)_{k \in \N } \subset X$ tal que
$\lim_{k \to \infty} x_k = w \in \Rn \label{eq:class7-2}$

Como $(x_k)_{k \in \N} \subset X,$ por \eqref{eq:class7-1}.

$\exists j \colon \N \rightarrow \N$ estricatamente creciente y $\exists z \in X$ tal que $\lim_{k \to \infty}  x_{j(k)} = z \in X$

De \eqref{eq:class7-2}, $\lim_{k \to \infty} x_{j(k)} = w$

Por la unicidad del límite $w = z \in X$

$$
	\therefore \overline{X}  \subset X
$$
Así, $X$ es cerrado

Mostremos que $X$ es acotado $\rightarrow \exists c>0, \forall z \in X, \abs{z} \geq c$
Supongamos que $X$ no es acotado, $\rightarrow \forall c>0, \exists z_k \in X, \abs{Z_k} >k \label{eq:class7-3}$

Hemos construido $(z_k)_{k \in \N} \subset X$ tal que $$
	\lim_{k \to \infty} \abs{z_k} = +\infty$$

Dado  $i \colon \N \to \N$ estrictamente creciente. $i(k) \leq k, \forall k \in \N$

$$
	\forall k \in \N, \quad \abs{Z_{i(k)}} > i(k)
$$
$$
	\lim_{k \to \infty} \abs{Z_{i(k)}} = +\infty
$$
Así $(z_{i(k)})_{k \in \N} $ no converge, esto contradice a \eqref{eq:class7-1}.

}



\propp{
Sea \( X \subset \mathbb{R}^n \). Entonces, \( X \) es \textbf{sucesionalmente compacto} si y solo si es \textbf{compacto}, es decir:
\[
	X \text{ es compacto} \;\Leftrightarrow\; \text{toda sucesión } (x_k)_{k \in \mathbb{N}} \subset X \text{ tiene una subsucesión convergente en } X.
\]
}{
\textbf{(\( \Rightarrow \))} Supongamos que \( X \) es compacto.

\rmkb{Por el teorema de Heine–Borel en \(\mathbb{R}^n\), esto implica que \(X\) es cerrado y acotado.}

Sea \( (x_k)_{k \in \mathbb{N}} \subset X \) una sucesión arbitraria.

Como \(X\) es acotado, entonces \( (x_k) \) también lo es.
Por el \textbf{teorema de Bolzano–Weierstrass}, existe una subsucesión \( (x_{i(k)})_{k \in \mathbb{N}} \subset X \) tal que
\[
	\lim_{k \to \infty} x_{i(k)} = w_0 \in \mathbb{R}^n.
\]

Como \( X \) es cerrado y \( x_{i(k)} \in X \), se sigue que \( w_0 \in \overline{X} = X \).
\[
	\Rightarrow (x_{i(k)}) \text{ converge en } X.
\]

\textbf{(\( \Leftarrow \))} Supongamos que toda sucesión en \( X \) posee una subsucesión convergente en \( X \). Queremos probar que \(X\) es compacto.

\textbf{1. \(X\) es cerrado:}
Sea \( w \in \overline{X} \). Entonces existe una sucesión \( (x_k)_{k \in \mathbb{N}} \subset X \) tal que
\[
	\lim_{k \to \infty} x_k = w.
\]

Por hipótesis, \( (x_k) \) tiene una subsucesión convergente \( x_{j(k)} \to z \in X \).
Pero como \( x_{j(k)} \subset (x_k) \) y la sucesión original converge a \(w\), se sigue que también:
\[
	\lim_{k \to \infty} x_{j(k)} = w.
\]

Por la unicidad del límite:
\[
	w = z \in X.
\]
\[
	\Rightarrow \overline{X} \subset X \Rightarrow X \text{ es cerrado}.
\]

\textbf{2. \(X\) es acotado:}
Supongamos que \(X\) no es acotado.

Entonces, para todo \(k \in \mathbb{N}\), existe \( z_k \in X \) tal que
\[
	\abs{z_k} > k.
\]

Esto define una sucesión \( (z_k)_{k \in \mathbb{N}} \subset X \) tal que:
\[
	\lim_{k \to \infty} \abs{z_k} = +\infty.
\]

Sea \( i \colon \mathbb{N} \to \mathbb{N} \) estrictamente creciente.
Entonces, para todo \(k\),
\[
	\abs{z_{i(k)}} > i(k) \geq k \Rightarrow \lim_{k \to \infty} \abs{z_{i(k)}} = +\infty.
\]

\rmkb{Toda subsucesión de \( (z_k) \) también diverge, por lo tanto, \( (z_k) \) no tiene subsucesión convergente.}

Esto contradice la hipótesis de sucesionalmente compacto.
\[
	\Rightarrow X \text{ debe ser acotado}.
\]

Por lo tanto, \(X\) es cerrado y acotado, es decir, \textbf{compacto}.
}

\rmkb{En espacios métricos, el hecho de ser cerrado y acotado \textbf{no implica} ser sucesionalmente compacto. Esta propiedad es equivalente a la compacidad solo en \(\mathbb{R}^n\).}


\propp{
Sea \( \{K_p\}_{p \in \N} \) una colección de subconjuntos compactos de \( \R^n \) tal que:
\[
	K_p \supset K_{p+1}, \quad \forall p \in \N, \qquad K_p \neq \varnothing
\]
Entonces:
\[
	K := \bigcap\limits_{p \in \N} K_p \text{ es compacto y } K \neq \varnothing
\]
}{
\begin{itemize}
	\item \( \bigcap\limits_{p \in \N} K_p \subset K_1 \Rightarrow \bigcap\limits_{p \in \N} K_p \) es acotado.
	\item \( \bigcap\limits_{p \in \N} K_p \) es cerrado, pues cada \( K_p \) es cerrado.
\end{itemize}
\[
	\Rightarrow \bigcap\limits_{p \in \N} K_p \text{ es compacto}
\]

\textbf{Mostremos que \( K \neq \varnothing \colon \)}

Como \( K_p \neq \varnothing, \ \forall p \in \N \), fijamos \( x_p \in K_p \subset K_1 \Rightarrow (x_p)_{p \in \N} \subset K_1 \)

\[
	\Rightarrow (x_p)_{p \in \N} \text{ es acotada}
	\Rightarrow \exists i \colon \N \to \N \text{ estrictamente creciente tal que }
	\lim\limits_{p \to \infty} x_{i(p)} = z_0 \in K_1
\]

Tomemos \( (x_{i(p)})_{p \geq 2} \). Como \( x_{i(p)} \in K_{i(p)} \subset K_2 \) y \( i(p) \leq i(2) \leq 2 \), para todo \( p \leq 2 \), entonces:

\[
	\lim_{\substack{p \to \infty \\ p \leq 2}} x_{i(p)} = z_0, \quad x_{i(p)} \in K_2
\]

Fijemos \( k \in \N \). Como \( x_{i(p)} \in K_{i(p)} \subset K_k \) para \( p \leq k \), se tiene:

\[
	\lim_{\substack{p \to \infty \\ p \leq k}} x_{i(p)} = z_0 \in \overline{K_k} = K_k
\]

\[
	\therefore \forall k \in \N, \quad z_0 \in K_k
	\Rightarrow z_0 \in \bigcap\limits_{k \in \N} K_k = K
	\Rightarrow K \neq \varnothing
\]
}


\section{Cubrimiento}

Sea \( \{C_{\lambda}\}_{\lambda \in L} \) una familia de subconjuntos de \( \mathbb{R}^n \).

Decimos que \( \{C_{\lambda}\}_{\lambda \in L} \) es un \textbf{cubrimiento} de un conjunto \( X \subset \mathbb{R}^n \) si:
\[
	X \subset \bigcup\limits_{\lambda \in L} C_{\lambda}
\]

Un \textbf{subcubrimiento finito} de \( \{C_{\lambda}\}_{\lambda \in L} \) es una subfamilia \( \{C_{\lambda}\}_{\lambda \in L'} \), donde \( L' \subset L \) es un conjunto finito, tal que:
\[
	X \subset \bigcup\limits_{\lambda \in L'} C_{\lambda}
\]

\thmrpf{}{}{
Sea \( K \subset \mathbb{R}^n \) un conjunto compacto.

Entonces, todo cubrimiento abierto de \( K \) admite un subcubrimiento finito.
}{
Sea \( \{A_{\lambda}\}_{\lambda \in L} \) un cubrimiento abierto de \( K \), es decir:
\[
	K \subset \bigcup\limits_{\lambda \in L} A_{\lambda}, \quad A_{\lambda} \subset \mathbb{R}^n \text{ abierto } \forall \lambda \in L.
\]

Como \( K \subset \mathbb{R}^n \) es compacto y \( \{A_{\lambda}\}_{\lambda \in L} \) es una familia de abiertos que cubre a \( K \), por definición de compacidad:

\[
	\exists L' \subset L, \ \abs{L'} < \infty \quad \text{tal que} \quad K \subset \bigcup\limits_{\lambda \in L'} A_{\lambda}
\]

\[
	\therefore \{A_{\lambda}\}_{\lambda \in L'} \text{ es un subcubrimiento finito de } K.
\]
}


\lem{Lindelöf}{
Sea \( X \subset \mathbb{R}^n \) y sea \( \{C_{\lambda}\}_{\lambda \in \Lambda} \) un cubrimiento abierto de \( X \), es decir,
\[
	X \subset \bigcup\limits_{\lambda \in \Lambda} C_{\lambda}, \quad C_{\lambda} \text{ abierto } \forall \lambda \in \Lambda.
\]
Entonces, existe un subconjunto numerable \( \Lambda_0 \subset \Lambda \) tal que
\[
	X \subset \bigcup\limits_{\lambda \in \Lambda_0} C_{\lambda}.
\]
}

Por (Lindelöf), $\exists \Lambda_0 = \{\lambda_i \in \Lambda \colon i \in \N\} \subset\Lambda$, $\Lambda$ es numerable.

Tal que
\begin{equation}
	K \subset \bigcuplim{\lambda \in \Lambda_0} A_{\lambda} = \bigcuplim{i \in \N} A_{\lambda_{i}}
	\label{eq:class7-4}
\end{equation}
Sea
\begin{align*}
	F_0    & = K \supset F_1 \supset F_2 \supset \ldots                                                                  \\
	F_1    & = K-A_{\lambda_1} = K \cap \bqty{A_{\lambda_1}^{C}}                                                         \\
	F_2    & = K-\bqty{A_{\lambda_1} \cup A_{\lambda_2}} = K \cap \bqty{A_{\lambda_1} \cup A_{\lambda_2}}^{C}            \\
	\vdots & =\qquad \vdots\hspace{3cm} \vdots                                                                           \\
	F_m    & =  K-\bqty{ \bigcuplim{i = 1 }{m} A_{\lambda_i} } = K \cap \bqty{ \bigcuplim{i = 1 }{m} A_{\lambda_i} }^{C}
\end{align*}

Supongamos que $\forall m \in \N, F_m \neq \varnothing $ y compactos tal que $F_m \supset F_{m+1}, \forall m \in \N$
$$
	\rightarrow \bigcuplim{m \in \N} F_m  \neq \varnothing
$$
pero,
\begin{align*}
	\bigcaplim{m \in \N} F_m & = \bigcaplim{m \in \N} \pqty{K \cap  \bqty{\bigcuplim{i = 1}{m} A_{\lambda_i}}^C }  \\
	                         & =K \cap \bigcaplim{m \in \N} \pqty{   \bqty{\bigcuplim{i = 1}{m} A_{\lambda_i}}^C } \\
	                         & =K \cap   \bqty{ \bigcup   \bqty{\bigcuplim{i = 1}{m} A_{\lambda_i}} }^C            \\
	                         & =K \cap   \bqty{    \bigcuplim{i = 1}{m} A_{\lambda_i} }^C                          \\\\
	                         & =K - \bigcuplim{i = 1}{m} A_{\lambda_i}                                             \\
	                         & = K-K^C                                                                             \\
\end{align*}
$$
	\Rightarrow \bigcaplim{m \in \N} F_m = \varphi
$$
$$
	\rightarrow \exists m_0 \in \N \quad \text{tal que}\quad F_{m_0} -0 \varnothing
$$
$$
	\rightarrow K- \bqty{\bigcuplim{i = 1}{m_0} A_{\lambda_i}} = \varnothing
$$
$$
	\rightarrow K \subset \bigcuplim{i = 1}{m_0}  A_{\lambda_i}
$$

\thmrpf{}{}{}{}
