\chapter{Clase I}

\section{Topología}

Topología en $\Real{n}$, con $n \in \N$

$$
\Real{n} =  \underbrace{\R \times \R \times \cdots \R }_{n \text{veces}}
$$
$$
\Real{n} = \{ (x_1, x_2, \cdots, x_n); x_1 \in \R \wedge  x_2 \in \R \wedge
 \ldots  \wedge  x_n \in \R       \}
$$

Al $(x_1, x_2, \cdots, x_n) $  se conoce como (n-upla), (vector), (punta) y a $\Real{n}$ es el n-ésimo espacio vectorial.

$\Real{0} = \{\Ou\}$  espacio vectorial de dimensión $0$.

En $\Rn$ tenemos:

\begin{itemize}
    \item \textbf{Adición:}
    \begin{align*}
        +\colon \Rn \times \Rn &\to \Rn \\
        (x, y) &\mapsto x + y = (x_1 + y_1, \dots, x_n + y_n),
    \end{align*}
    donde $x = (x_1, \dots, x_n)$ e $y = (y_1, \dots, y_n)$, $(x+y)$ es el vector suma.

    \item \textbf{Multiplicación por escalar:}
    \begin{align*}
        \cdot\colon \R \times \Rn &\to \Rn \\
        (\lambda, x) &\mapsto \lambda x = (\lambda x_1, \dots, \lambda x_n),
    \end{align*}
    donde $x = (x_1, \dots, x_n) \text{ y } \lambda \in \R$ .
\end{itemize}



\ex{
    Verificar que $(\Rn, +, \cdot)$ es un $\R-\text{espacio vectorial}$. 
    Esto es $(\R, +)$ es un grupo conmutativo y además.
    \begin{itemize}
        \item $(\lambda+u)x = \lambda x + u x$
        \item $\lambda(x+y) = \lambda x + \lambda y$
        \item $\lambda(\mu x) = (\lambda \mu)x$
        \item $1 x = x$
    \end{itemize}

    \rmk{$\Ou = (0,0, \cdots , 0) \in \Rn$ es el vector nulo.}

    Si $x=(x_1, x_2, \cdots, x_n)$
    $\rightarrow  -x = (-x_1, -x_2, \cdots, -x_n) $.  
    Además $ -1(x) \text{ es el inverso aditivo de } x  \text{ y  es denotado por  } -x$
    }

   En $\Rn$, la base canónica es $B = \{ e_1, e_2, \ldots, e_n \}$ donde:
   \[
   \begin{aligned}
   	e_1 &= (1,0,\cdots,0) \in \Rn \\
   	e_2 &= (0,1,\cdots,0) \in \Rn \\
   	e_i &= (0,\cdots, \underset{\substack{\uparrow \\ i\text{-ésimo}}}{1},\cdots,0) \in \Rn, \quad \forall i \in \{1,2,\ldots,n\} \\
   	e_n &= (0,0,\cdots,1) \in \Rn
   \end{aligned}
   \]
    
\ex{ 
   	Mostrar que $B$ es una base de $\Rn$.\\
   	
   	Sea $x \in \Rn$ con
   	\[
   	\begin{aligned}
        x &= (x_1, x_2, \cdots, x_n)\\
   		x &= (x_1,0,\cdots,0)+(0,x_2,0,\cdots,0)+\cdots+(0,0,\cdots,x_n) \\
   		x &= x_1 e_1 + x_2 e_2 + \cdots + x_n e_n
   	\end{aligned}
   	\]
    ¿$B$ es linealmente independiente?\\
    Sea $\lambda_1, \cdots, \lambda_n \in \Rn$ tal que
    $$
    \lambda_1 e_1 + \lambda_2 e_2 + \cdots + \lambda_n e_n = \Ou
    $$
    $$
     (\lambda_1, \cdots, \lambda_n) = \Ou = (0,0,\cdots,0)    
    $$
    $$
    \rightarrow \lambda_1 = 0 \wedge \lambda_2 = 0 \wedge \cdots \wedge \lambda_n = 0
    $$
   }
   
 $\Lineal{\Real{m}}{\Real{n}} = \{T: \Rm \longrightarrow \Rn : T 
 \text{ es una transformación lineal}\}$

 $M(n\times m)$  conjunto de matrices de orden $n\times m $ con entradas reales.
\begin{align*}
    \psi \colon \Lineal{\Rm}{\Rn} &\longrightarrow M(n\times m) \\
    A & \longmapsto (A)
\end{align*}
Considere \begin{align*} 
    B =& \{ e_1, e_2, \cdots, e_m\} &\subset \Rm \text{base canónica}\\
    B' =& \{ \overline{e_1} , \overline{e_2}, \cdots, \overline{e_n}\} &\subset \Rn \text{base canónica}\\
    \end{align*}
$$
Ae_j = a_{1j}\overline{e_1}+a_{2j}\overline{e_2}+\cdots+a_{nj}\overline{e_n}
$$

\NiceMatrixOptions%
 {code-for-last-row = \scriptstyle \rotate ,
 code-for-last-col = \scriptstyle }

\[
\psi(A)=
\begin{bNiceMatrix}[last-row=5]
    \Block[fill=red!15,rounded-corners]{4-1}{}
    a_{11} & a_{12} & \cdots & 
    \Block[fill=blue!15,rounded-corners]{4-1}{}
    a_{1m} \\
    a_{21} & a_{22} & \cdots & a_{2m} \\
    \vdots & \vdots & \ddots & \vdots \\
    a_{n1} & a_{n2} & \cdots & a_{nm}\\
    \text{Coef. de} Ae_1& & & \text{Coef. de} Ae_m
\end{bNiceMatrix}
\]

\ex{ Mostrar que $\psi$ es una biyección. más aún, e sun isomorfismo entre espacios vectoriales.

Decimos que  $\psi \colon \Rm \times \Rn \longrightarrow \Rp$ es bilineal si
\begin{align*}
    \psi(x+x', y) &= \psi(x,y) + \psi(x',y) ; \, &\forall x, x' \in \Rm, \forall y \in \Rn \\
    \psi(\lambda x, y) &= \lambda \psi(x,y) ; &\forall x \in \Rm, \forall y \in \Rn, \forall \lambda \in \R \\
    \psi(x, y+y') &= \psi(x,y) + \psi(x,y') ; & \\
    \psi(x, \lambda y) &= \lambda \psi(x,y) ; &
\end{align*}
  
}

Dados \begin{align*}
    x = (x_1,\ldots,x_m) \in \Rm \\
    y= (y_1,\ldots,y_n) \in \Rn
\end{align*} 

\begin{align*}
    \psi(x,y) & = \psi\pqty{\sum_{i=1}^{m} x_i e_i, \sum_{j=1}^{n} y_j \overline{e_j}} \\
    & = \sum_{i=1}^m x_i \psi\pqty{ e_i, \sum_{j=1}^n y_j \overline{e_j}} \\
    & = \sum_{i=1}^m x_i \sum_{j=1}^n y_j \psi\pqty{e_i, \overline{e_j}} \\
    & = \sum_{i=1}^m \sum_{j=1}^n x_i y_j \psi\pqty{e_i, \overline{e_j}} \text{ (por bilinealidad)}
\end{align*}

\section{Producto Interno y Norma en $\Rn$}
Sea $E$ un espacio vectorrial sobre $\R$ 
\subsection{Producto Interno}
Un producto interno sobre $E$ es una función
 
\defn{Producto interno}{
Un \textbf{producto interno} sobre un espacio vectorial \( E \) es una función

\begin{align*}
    \langle \cdot, \cdot \rangle \colon E \times E &\longrightarrow \mathbb{R} \\
    (x, y) &\longmapsto \langle x, y \rangle 
\end{align*}

que satisface las siguientes propiedades:

\begin{description}
    \item[(i) Linealidad en la primera componente:]
    \begin{align}
        \langle x + x', y \rangle &= \langle x, y \rangle + \langle x', y \rangle, &&\forall x, x', y \in E \label{eq:linealidad1} \\
        \langle \lambda x, y \rangle &= \lambda \langle x, y \rangle, &&\forall x, y \in E, \, \forall \lambda \in \mathbb{R} \label{eq:linealidad2}
    \end{align}

    \item[(ii) Simetría:]
    \begin{equation}
        \langle x, y \rangle = \langle y, x \rangle, \quad \forall x, y \in E \label{eq:simetria}
    \end{equation}

    \item[(iii) Positividad definida:]
    \begin{equation}
        \langle x, x \rangle > 0, \quad \forall x \in E. \forall x \neq \Ou \label{eq:positividad}
    \end{equation}
\end{description}
}

Como consecuencia de las propiedades \eqref{eq:linealidad1} y \eqref{eq:simetria}, también se cumple:

\begin{itemize}
    \item \(\langle x, y + y' \rangle = \langle x, y \rangle + \langle x, y' \rangle\), \quad \(\forall x, y, y' \in E\)
    \item \(\langle x, \lambda y \rangle = \lambda \langle x, y \rangle\), \quad \(\forall x, y \in E, \, \forall \lambda \in \mathbb{R}\)
\end{itemize}

\rmk{
En otras palabras, el producto interno \( \langle \cdot, \cdot \rangle \) es \textbf{bilineal} y \textbf{simétrico}.
}

\ex{
Sea $\langle . , . \rangle  \colon \Rn \times \Rn \rightarrow \R$\\
Si  $	x = (x_1,\ldots,x_m) \in \Rm$ e $	y= (y_1,\ldots,y_n) \in \Rn $

$$
\rightarrow \langle x, y \rangle = \sum_{i = 1}^{n} x_i y_i  \text{ (Producto interno euclideano)}
$$

}

\defn{Ortogonalidad}{
	Sean \( x, y \in \mathbb{R}^n \). Decimos que \( x \) y \( y \) son \textbf{ortogonales} si
	\begin{equation}
		\langle x, y \rangle = 0. \label{eq:def-ortogonalidad}
	\end{equation}
}


\subsection{Norma}

\defn{Norma}{
	Una \textbf{norma} sobre el espacio vectorial \( E \) es una función
	\begin{align}
		\|\cdot\| \colon E &\longrightarrow \mathbb{R} \label{eq:norma-def} \\
		x &\longmapsto \|x\| \nonumber
	\end{align}
	que satisface las siguientes propiedades:
	\begin{description}
		\item[(i) Desigualdad triangular:]
		\begin{equation}
			\|x + y\| \leq \|x\| + \|y\|, \quad \forall x, y \in E
			 \label{eq:norma-triangular-i}
		\end{equation}
		
		\item[(ii) Homogeneidad absoluta:]
		\begin{equation}
			\|\alpha x\| = |\alpha| \cdot \|x\|, \quad \forall \alpha \in \mathbb{R}, \, \forall x \in E \label{eq:norma-homogeneidad-ii}
		\end{equation}
		
		\item[(iii) Positividad definida:]
		\begin{equation}
			\|x\| > 0 \quad \Longleftrightarrow \quad x \neq 0 \label{eq:norma-positividad-iii}
		\end{equation}
	\end{description}
}


\begin{itemize}
\item De \eqref{eq:norma-homogeneidad-ii} 
$$
\norm{\Ou} = \norm{0 x}=\abs{0}\norm{x} = 0
$$ 
\item De \eqref{eq:norma-homogeneidad-ii} 
$$
\norm{-x}=\norm{(-1)x}=\abs{-1}\norm{x}=\norm{x}
$$
\item De \eqref{eq:norma-triangular-i} 
$$
0 = \norm{\Ou} =\norm{x+(-x)} \leq \norm{x}+\underbrace{\norm{-x}}_{\norm{x}}
$$
\begin{equation}
	\therefore 0\leq \norm{x}, \quad \forall x \in E 
	\label{eq:norma-positividad2}
\end{equation}
\item  De \eqref{eq:norma-positividad2} y \eqref{eq:norma-positividad-iii} es equivalente a
$$
\norm{x} = 0 \longleftrightarrow x =\Ou
$$
\end{itemize}

\ex{
La norma euclideana en $\Rn$,  
\begin{align*}
	\norm{\cdot} \colon \mathbb{R}^n &\longrightarrow \mathbb{R} \\
	x = (x_1, \ldots, x_n) &\longmapsto \norm{x} = \sqrt{x_1^2 + \cdots + x_n^2} = \sqrt{\langle x, x \rangle}
\end{align*}


Mostemos que $\norm{.}$ es una norma en $\Rn$\\
Para \eqref{eq:norma-homogeneidad-ii} es inmediato, \eqref{eq:norma-positividad-iii} hay que negar $x \neq 0$, o sea $x = 0$\\
Solo falta mostrar \eqref{eq:norma-triangular-i}
Sean $x,y \in \R, y \neq 0 \rightarrow \inner{y}{y} >0 $\\
¿Para qué $\lambda \in \R$, se tiene que  $\inner{y}{x-\lambda y } = 0$

\begin{align*}
	&\leftrightarrow \inner{y}{x} - \lambda \inner{y}{y} = 0\\
	&\leftrightarrow \frac{\inner{y}{x}}{ \inner{y}{y}} = \lambda
\end{align*}
\begin{align*}
	\norm{x}^2 &= \norm{\lambda y +x-\lambda y}^2\\
	& =  \inner{xy+ x-\lambda y}{xy+ x-\lambda y}\\
	& = \abs{\lambda}^2 \norm{y}^2+\underbrace{\norm{x-\lambda y}^2 }_{\geq 0}
\end{align*}
$$
\rightarrow \norm{x}^2 \geq \abs{\lambda}^2 \norm{y}^2 = \frac{\abs{\inner{x}{y}}^2}{\norm{y}^2}
$$
$$
\rightarrow \norm{x}^2 \norm{y}^2 \geq \abs{\inner{x}{y}}^2
$$
$$
\rightarrow \norm{x}\norm{y} \geq \abs{\inner{x}{y}}, \quad \forall x,y \in \Rn
$$
Desigualdad de Cauchy-Schwarz.\\
Además, la igualdad se da cuando uno de los vectores es multiplo del otro o $x, y$ es linealmente dependiente.
}

\clmp{}{
	$\forall\,  x, y \in \Rn, \norm{x+y} \leq \norm{x}+ \norm{y}$
}{
 Sean $x, y \in \Rn$
 \begin{align*}
 	\norm{x+y}^2 & = \inner{x+y}{x+y}\\
 	& = \norm{x}^2+2\inner{x}{y}+\norm{y}^2\\
 	& \leq  \norm{x}^2+2\norm{x}\norm{y}+\norm{y}^2  = (\norm{x}+\norm{y})^2
 \end{align*}
 $$
 \rightarrow \norm{x+y} \leq \norm{x}+\norm{y}
 $$
}


Dados $x,y \in \Rn, $   $d(x,y)=\norm{x-y}$
\begin{align*}
	d \colon \Rn \times \Rn &\longrightarrow \R\\
	(x,y) &\longmapsto d(x,y)= \norm{x-y}; \quad d(x,y) \text{ es una métrica}
\end{align*}

\begin{itemize}
	\item[(i)] $\forall \, x, y \in \Rn, d(x,y)\geq 0$
	\item[(ii)]  $\forall \, x, y \in \Rn, d(x,y) = d(y,x)  $ simetría
	\item[(iii)]   $\forall \,  x, y, z  \in \Rn, d(x,y) \leq d(x,z) + d(z,y)    $ 
	\item[(iv)]   $\forall \,  x, y   \in \Rn, d(x,z) =0 \, \leftrightarrow x=y  $
\end{itemize}


\ex{
Sea $x = (x_1, \ldots, x_n) \in \Rn$,
\begin{align}	
	\norm{x}_1 & = \abs{x_1}+\abs{x_2}+\cdots +\abs{x_n} \label{eq:ex-norma-1}\\
	\norm{x}_{\infty} &= \max\{ \abs{x_1}, \abs{x_2}, \cdots, \abs{x_n}\} \nonumber\\
	\norm{x}_2&=\norm{x} = \sqrt{x_1^2+x_2^2+\cdots+ x_n^2} \nonumber
\end{align}
De \eqref{eq:ex-norma-1},
\begin{align*}
	 \norm{x+y}_1 &= \abs{x_1+y_1}+\cdots +\abs{x_n+y_n}\\
	 & \leq \abs{x_1}+\abs{y_1}+\cdots + \abs{x_n}+\abs{y_n} =  \norm{x}_1+\norm{y}_1
\end{align*}
$$
\norm{x} = \sqrt{x_1^2+\cdots x_1^n} \geq \abs{x_1} \rightarrow \norm{x}_{\infty} \leq \norm{x}
$$
$$
\norm{x}_1 \geq \norm{x}
$$
}

\clmp{Ley del paralelogramo}{
	 
	 Si $\norm{.} $ proviene de un producto interno en $\Rn$, entonces $\forall\, x,y \in \Rn$
	 $$
	 \norm{w} = \sqrt{\inner{w}{w}}
	 $$
	 $$
	 \Rightarrow \norm{x+y}^2 +\norm{x-y}^2 = 2(\norm{x}^2+\norm{y}^2)
	 $$
}{
 \begin{align}
 	 \norm{x+y}^2 = \inner{x+y}{x+y} = \norm{x}^2+2\inner{x}{y}+\norm{y}^2 \label{eq:paralelogramo-1}\\
 	 \norm{x-y}^2 = \inner{x+y}{x+y} = \norm{x}^2-2\inner{x}{y}+\norm{y}^2 \label{eq:paralelogramo-2} 
 \end{align}
 
 De \eqref{eq:paralelogramo-1} y \eqref{eq:paralelogramo-2}
 $$
  \norm{x+y}^2 +\norm{x-y}^2 = 2(\norm{x}^2+\norm{y}^2)
 $$
}

\ex{
En $\Real{2}, e_1=(1,0), e_2=(0,1)$\\
Si $\norm{.}_1$ en $\Real{2}$ proviene de un producto interno.
$$
\rightarrow \forall \, x, y \in \Real{2}, \norm{x+y}_1^2 +\norm{x-y}_1^2 = 2(\norm{x}_1^2+\norm{y}_1^2)
$$
Si $x=e_1, \, y= e_2 \longrightarrow 2^2+2^2 = 2(1^2+1^2)$ (absurdo)\\
Por lo tanto, $\norm{.}_1$ no proviene de un prducto interno.
}

Sean $\abs{.} \text{ y } \norm{.} $ dos normas en $\Rn$. Decimos que $\abs{.}$ y $\norm{.}$ son equivalentes si existen $\alpha, \beta \in \R^{+}$ tal que,
$$
\forall\, x \in \Rn, \, \alpha \abs{x} \leq \norm{x}\leq\beta \abs{x}
$$

dada $\norm{.}$ norma en $\Rn$ la \textbf{BOLA ABIERTA } con centro en $a \in \Rn$ y radio $r>0$ es

$$
B^{\norm{.}} (a,r) = \{z \in \Rn\colon \norm{z-a} < r\} 
$$

En $\Real{2}$

$$
B^{\norm{.}} (\Ou,1) = \{(z_1, z_2) \in \Real{2} \colon \norm{(z_1, z_2)} < 1\} 
$$

$$
B^{\norm{.}_{\infty}} (\Ou,1) = \{(z_1, z_2) \in \Real{2} \colon \max{\abs{z_1}, \abs{z_2}} < 1\} 
$$

$$
B^{\norm{.}_{1}} (\Ou,1) = \{(z_1, z_2) \in \Real{2} \colon   \abs{z_1}+ \abs{z_2}< 1\} 
$$

\thmr{ Equivalencia de normas en $\Rn$ }{thm:Teorema1}{
 	
 	En $\Rn$ todas las normas son equivalentes.
 }

\pf{ % Del teorema Theorem~\ref{thm:Teorema1}
	Sea $(x_k)_{k \in \N}$ una sucesión de puntos de $\Rn$
	\begin{align*}
		x \colon \N &\longrightarrow \Rn\\
		k & \longmapsto x(k)= x_k = (x_1^k, x_2^k, \ldots, x_n^k)
	\end{align*}
	Sea $a = (a_1, a_2, \ldots, a_n) \in \Rn $, decimos que $(x_k)_{k \, \in \, \N} $ converge a $a \, \in \Rn$ si,
	 
	$$
	\forall\, \varepsilon >0, \exists k_0 \in \, \N, \forall \, k \in \N , k \geq k_0 \longrightarrow \norm{x_k-a} <\varepsilon
	$$
	$$
	\lim\limits_{k \to \infty} x_k =a \text{ es respecto a la norma  }\quad \norm{x_k-a}_{\infty} \leq \abs{(x_k-a)}
	$$
	
	}
	
	
\rmkb{
	$$
	\norm{x}_{\infty} \leq \norm{x} \leq \norm{x}_1 \leq n\norm{x}_{\infty}\leq n \norm{x}_1
	$$
	$$
	\Rightarrow \norm{x} \leq \norm{x}_1 \leq n \norm{x}
	$$
}


\thmr{Criterio de convergencia en \(\mathbb{R}^n\)}{thm:Teorema2}{
	Sea \((x_k)_{k \in \mathbb{N}} \subset \mathbb{R}^n\), donde
	\[
	x_k = (x_1^k, x_2^k, \ldots, x_n^k) \quad \forall\, k \in \mathbb{N},
	\]
	y sea \(a = (a_1, a_2, \ldots, a_n) \in \mathbb{R}^n\).
	
	Entonces,
	\[
	\lim_{k \to \infty} x_k = a \quad \text{si y solo si} \quad \forall\, i \in \{1, \ldots, n\},\; \lim_{k \to \infty} x_i^k = a_i.
	\]
}

\pf{  $(\Rightarrow)$ Supongamos que $\lim\limits_{k\to \infty} x_k =a$\\
	Dado,
	 $
	 \varepsilon > 0, \exists\, k_0 \in \N, \forall\, k \in \N
	 $
	 
	 $$
	 k\geq k_0 \longrightarrow \abs{x_i^k-a} \leq  \norm{x_k-a}_{\infty} \leq \norm{x_{k}-a} < \varepsilon, \quad \forall \, i \in \{1, \ldots, n\}
	 $$
	 $$
	\therefore  \lim\limits_{k\to \infty} x_i^k =a_i,\quad \forall\, i \in \{1, \ldots, n\}
	 $$
	 
	 $(\Leftarrow)$ Suponga que $\forall \, i \in \{1,\ldots, n\}$
	 
	 $$
	 \lim\limits_{k\to \infty} x_i^k =a_i
	 $$
	 Dado $\varepsilon >0, \forall\, i \in \{1,\ldots, n\} \, \exists \, k_i^o \in \N$ tal que $\forall \, k \in \N, \varepsilon_0 =\frac{\varepsilon}{\sqrt{n}}$
	 
	  
	 \begin{align*}
	 	k \geq k_i^o &\rightarrow \abs{x_i^k -a_i} < \varepsilon_0\\
	 	&\rightarrow \abs{x_i^k -a_i}^2 < \varepsilon_0^2
	 \end{align*}
	 Sea $k_0 = \max \{k_1^0,\, k_2^0, \ldots, \,k_n^0\}$ si $	k \geq k_0 \geq k_i^0  \; \forall\; i \in \{1, 2, \ldots, n\}$
	 
	 $$
	 \rightarrow \sqrt{\sum_{i=1}^{n}\abs{x_i^k-a_i}^2}< \sqrt{n \varepsilon_0^2}
	 $$
	 $$
	 \rightarrow \norm{x_i^k -a} < \sqrt{n} \varepsilon_0 =\varepsilon
	 $$
	 
	%Theorem~\ref{{thm:Teorema1}
		
	}



























 