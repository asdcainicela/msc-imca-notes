\documentclass[oneside]{book}
\usepackage[left=1 in, right=1 in]{geometry}
\usepackage{amsmath, amsthm, amssymb, amsfonts}
\usepackage{ dsfont } % Usar \mathds{O}
\usepackage{thmtools}
\usepackage{graphicx}
\usepackage{setspace}
\usepackage{geometry}
\usepackage{float}
\usepackage{hyperref}
\usepackage[utf8]{inputenc}
\usepackage[english]{babel}
\usepackage{framed}
\usepackage[dvipsnames]{xcolor}
\usepackage{environ}
\usepackage{tcolorbox}
\usepackage{extpfeil}
%\usepackage[all]{xy} %Para flechas, muy antiguo
\usepackage{tikz}          % núcleo de TikZ
\usetikzlibrary{cd,matrix,arrows.meta} % tikz-cd + matrices + flechas
\usepackage{tikz-cd}
\usepackage{mathrsfs}  % letras cursivas L
\usepackage{nicematrix}
% ----------------- Configuración de Tipografía -----------------
\usepackage{mathpazo} % Fuente Palatino para matemáticas y texto
% \usepackage[utopia]{mathdesign}
% \usepackage[nomath,fulloldstylenums,fulloldstyle]{kpfonts} % Fuente alternativa
%\usepackage[nomath,fulloldstylenums]{kpfonts}
\usepackage{parskip} %% Evitar indentación en nuevos párrafos

\usepackage{physics}

\tcbuselibrary{theorems,skins,breakable}

\setstretch{1.2}
\geometry{
    textheight=9in,
    textwidth=5.5in,
    top=1in,
    headheight=12pt,
    headsep=25pt,
    footskip=30pt
}

% Variables
\def\notetitle{Análisis I}
\def\noteauthor{
    \textbf{Student} \\ 
     asdCain\\
    IMCA}
\def\notedate{Semester}

% The theorem system and user-defined commands
\input{theorems.tex}
\newcommand{\lcm}{\operatorname{lcm}}
\newcommand{\R}{\mathbb{R}} 
\newcommand{\real}[1]{\mathbb{R}^{#1}}
\newcommand{\N}{\mathbb{N}} 




% ------------------------------------------------------------------------------

\begin{document}
\title{\textbf{
    \LARGE{\notetitle} \vspace*{10\baselineskip}}
    }
\author{\noteauthor}
\date{\notedate}

%\maketitle
\newpage

\tableofcontents
\newpage

% ------------------------------------------------------------------------------
\chapter{Análisis I}

\section{Topología}

Topología en $\Real{n}$, con $n \in \N$

$$
\Real{n} =  \underbrace{\R \times \R \times \cdots \R }_{n \text{veces}}
$$
$$
\Real{n} = \{ (x_1, x_2, \cdots, x_n); x_1 \in \R \wedge  x_2 \in \R \wedge
 \ldots  \wedge  x_n \in \R       \}
$$

Al $(x_1, x_2, \cdots, x_n) $  se conoce como (n-upla), (vector), (punta) y a $\Real{n}$ es el n-ésimo espacio vectorial.

$\Real{0} = \{\Ou\}$  espacio vectorial de dimensión $0$.

En $\Rn$ tenemos:

\begin{itemize}
    \item \textbf{Adición:}
    \begin{align*}
        +\colon \Rn \times \Rn &\to \Rn \\
        (x, y) &\mapsto x + y = (x_1 + y_1, \dots, x_n + y_n),
    \end{align*}
    donde $x = (x_1, \dots, x_n)$ e $y = (y_1, \dots, y_n)$, $(x+y)$ es el vector suma.

    \item \textbf{Multiplicación por escalar:}
    \begin{align*}
        \cdot\colon \R \times \Rn &\to \Rn \\
        (\lambda, x) &\mapsto \lambda x = (\lambda x_1, \dots, \lambda x_n),
    \end{align*}
    donde $x = (x_1, \dots, x_n)$ .
\end{itemize}



\ex{
    Verificar que $(\Rn, +, \cdot)$ es un $\R-\text{espacio vectorial}$.\\
    Esto es $(\R, +)$ es un grupo conmutativo y además.
    \begin{itemize}
        \item $(\lambda+u)x = \lambda x + u x$
        \item $\lambda(x+y) = \lambda x + \lambda y$
        \item $\lambda(\mu x) = (\lambda \mu)x$
        \item $1 x = x$
    \end{itemize}

    \rmk{$\Ou = (0,0, \cdots , 0) \in \Rn$ es el vector nulo.}

    Si $x=(x_1, x_2, \cdots, x_n)$
    $\rightarrow  -x = (-x_1, -x_2, \cdots, -x_n) $. Además $ -1(x) \text{ es el inverso aditivo de } x  \text{ y se denota por } -x$
    }

   En $\Rn$, la base canónica es $B = \{ e_1, e_2, \ldots, e_n \}$ donde:
   \[
   \begin{aligned}
   	e_1 &= (1,0,\cdots,0) \in \Rn \\
   	e_2 &= (0,1,\cdots,0) \in \Rn \\
   	e_i &= (0,\cdots, \underset{\substack{\uparrow \\ i\text{-ésimo}}}{1},\cdots,0) \in \Rn, \quad \forall i \in \{1,2,\ldots,n\} \\
   	e_n &= (0,0,\cdots,1) \in \Rn
   \end{aligned}
   \]
    
\ex{ 
   	Mostrar que $B$ es una base de $\Rn$.\\
   	
   	Sea $x \in \Rn$ con
   	\[
   	\begin{aligned}
        x &= (x_1, x_2, \cdots, x_n)\\
   		x &= (x_1,0,\cdots,0)+(0,x_2,0,\cdots,0)+\cdots+(0,0,\cdots,x_n) \\
   		x &= x_1 e_1 + x_2 e_2 + \cdots + x_n e_n
   	\end{aligned}
   	\]
    ¿$B$ es linealmente independiente?\\
    Sea $\lambda_1, \cdots, \lambda_n \in \Rn$ tal que
    $$
    \lambda_1 e_1 + \lambda_2 e_2 + \cdots + \lambda_n e_n = \Ou
    $$
    $$
     (\lambda_1, \cdots, \lambda_n) = \Ou = (0,0,\cdots,0)    
    $$
    $$
    \rightarrow \lambda_1 = 0 \wedge \lambda_2 = 0 \wedge \cdots \wedge \lambda_n = 0
    $$
   }
   
 $\Lineal{\Real{m}}{\Real{n}} = \{T: \Rm \longrightarrow \Rn : T 
 \text{ es una transformación lineal}\}$

 $M(n\times m)$  conjunto de matrices de orden $n\times m $ con entradas reales.
\begin{align*}
    \psi \colon \Lineal{\Rm}{\Rn} &\longrightarrow M(n\times m) \\
    A & \longmapsto (A)
\end{align*}
Considere \begin{align*} 
    B =& \{ e_1, e_2, \cdots, e_m\} &\subset \Rm \text{base canónica}\\
    B' =& \{ \overline{e_1} , \overline{e_2}, \cdots, \overline{e_n}\} &\subset \Rn \text{base canónica}\\
    \end{align*}
$$
Ae_j = a_{1j}\overline{e_1}+a_{2j}\overline{e_2}+\cdots+a_{nj}\overline{e_n}
$$

\NiceMatrixOptions%
 {code-for-last-row = \scriptstyle \rotate ,
 code-for-last-col = \scriptstyle }

\[
\psi(A)=
\begin{bNiceMatrix}[last-row=5]
    \Block[fill=red!15,rounded-corners]{4-1}{}
    a_{11} & a_{12} & \cdots & 
    \Block[fill=blue!15,rounded-corners]{4-1}{}
    a_{1m} \\
    a_{21} & a_{22} & \cdots & a_{2m} \\
    \vdots & \vdots & \ddots & \vdots \\
    a_{n1} & a_{n2} & \cdots & a_{nm}\\
    \text{Coef. de} Ae_1& & & \text{Coef. de} Ae_m
\end{bNiceMatrix}
\]

\ex{ Mostrar que $\psi$ es una biyección. más aún, e sun isomorfismo entre espacios vectoriales.

Decimos que  $\psi \colon \Rm \times \Rn \longrightarrow \Rp$ es bilineal si
\begin{align*}
    \psi(x+x', y) &= \psi(x,y) + \psi(x',y) ; \, &\forall x, x' \in \Rm, \forall y \in \Rn \\
    \psi(\lambda x, y) &= \lambda \psi(x,y) ; &\forall x \in \Rm, \forall y \in \Rn, \forall \lambda \in \R \\
    \psi(x, y+y') &= \psi(x,y) + \psi(x,y') ; & \\
    \psi(x, \lambda y) &= \lambda \psi(x,y) ; &
\end{align*}
  
}

Dados \begin{align*}
    x = (x_1,\ldots,x_m) \in \Rm \\
    y= (y_1,\ldots,y_n) \in \Rn
\end{align*} 

\begin{align*}
    \psi(x,y) & = \psi\pqty{\sum_{i=1}^{m} x_i e_i, \sum_{j=1}^{n} y_j \overline{e_j}} \\
    & = \sum_{i=1}^m x_i \psi\pqty{ e_i, \sum_{j=1}^n y_j \overline{e_j}} \\
    & = \sum_{i=1}^m x_i \sum_{j=1}^n y_j \psi\pqty{e_i, \overline{e_j}} \\
    & = \sum_{i=1}^m \sum_{j=1}^n x_i y_j \psi\pqty{e_i, \overline{e_j}} \text{ (por bilinealidad)}
\end{align*}

\section{Producto Interno y Normal en $\Rn$}
Sea $E$ un espacio vectorrial sobre $\R$ 
\subsection{Producto Interno}
Un producto interno sobre $E$ es una función

 
\begin{align*}
    <\, ,\, > \colon E \times E &\longrightarrow \R \text{  tal que satisface}\\
    (x,y) & \longmapsto <x,y> \\
\end{align*}
\begin{itemize}
    \item $\forall x, x', y \in E, <x+x',y> = <x,y> + <x',y>$
    \item $\forall x, x', y \in E, <\lambda x,y> = \lambda <x,y>$
\end{itemize}
\chapter{Clase 2}

\section{Equivalante Norms}
Teníamos que, $\forall\, x \in \Rn$,
$$
\norm{x}_{\infty} \leq \norm{x}_2 \leq \norm{x}_1 \leq n \norm{x}_{\infty}
$$
Estas tres normas son equivalentes.

Dada $(x_k)_{k \in \N} \subset \Rn$, con $\forall \, k\in \N, x_k = (k_1^k, \, x_2^k, \ldots, x_n^k) \in \Rn, \; a = (a_1, \ldots ,a_n)\in \Rn$

$
\lim\limits_{k \to \infty} x_k = a $ significa que $\forall\; \varepsilon>0,\, \exists k_0 \in \N, \, \forall k \in \N
$
\begin{align*}
	k \geq k_0 \longrightarrow &\norm{x_k-a} < \varepsilon\\
	& \abs{ \norm{x_k-a}-0} < \varepsilon,  \quad \norm{x_k-a} \in \R 
\end{align*}

\thmr{Equivalencia de la convergencia en \(\mathbb{R}^n\)}{thm:Teorema3}{
	Sea \( (x_k)_{k \in \N} \subset \mathbb{R}^n \) y \( a \in \mathbb{R}^n \). Entonces:
	$$
	\lim_{k \to \infty} x_k = a \quad \text{en } \norm{\cdot}_2 
	\iff 
	\forall i \in \{1, \ldots, n\}, \quad \lim_{k \to \infty} x_i^k = a_i
	$$
}

\prop{
	$X \subset \Rn $ es acotado $\longleftrightarrow \Pi_i(X) \subset \R$ es acotada $\forall\; i \in \{1, 2, \ldots, n\}$
}
\pf{
}

\thmr{}{thm:Teorema4}{
	Sea \( (x_k)_{k \in \mathbb{N}} \subset \mathbb{R}^n \) una sucesión acotada. Entonces, existe una subsucesión \( (x_{k_j})_{j \in \mathbb{N}} \) que converge en \(\mathbb{R}^n\).
}


%\input{class/_template.tex}

\end{document}
