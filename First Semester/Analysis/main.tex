\documentclass[oneside]{book}

\usepackage{amsmath, amsthm, amssymb, amsfonts}
\usepackage{ dsfont } % Usar \mathds{O}
\usepackage{thmtools}
\usepackage{graphicx}
\usepackage{setspace}
\usepackage{geometry}
\usepackage{float}
\usepackage{hyperref}
\usepackage[utf8]{inputenc}
\usepackage[english]{babel}
\usepackage{framed}
\usepackage[dvipsnames]{xcolor}
\usepackage{environ}
\usepackage{tcolorbox}
\usepackage{extpfeil}


\tcbuselibrary{theorems,skins,breakable}

\setstretch{1.2}
\geometry{
    textheight=9in,
    textwidth=5.5in,
    top=1in,
    headheight=12pt,
    headsep=25pt,
    footskip=30pt
}

% Variables
\def\notetitle{Análisis I}
\def\noteauthor{
    \textbf{Student} \\ 
     asdCain\\
    IMCA}
\def\notedate{Semester}

% The theorem system and user-defined commands
\input{theorems.tex}
\newcommand{\lcm}{\operatorname{lcm}}
\newcommand{\R}{\mathbb{R}} 
\newcommand{\real}[1]{\mathbb{R}^{#1}}
\newcommand{\N}{\mathbb{N}} 




% ------------------------------------------------------------------------------

\begin{document}
\title{\textbf{
    \LARGE{\notetitle} \vspace*{10\baselineskip}}
    }
\author{\noteauthor}
\date{\notedate}

%\maketitle
\newpage

%\tableofcontents
\newpage

% ------------------------------------------------------------------------------

\chapter{Análisis I}

\section{Introducción}

Topología en $\real{n}$, con $n \in \N$

$$
\real{n} =  \underbrace{\R \times \R \times \cdots \R }_{n \text{veces}}
$$
$$
\real{n} = \{ (x_1, x_2, \cdots, x_n); x_1 \in \R \wedge  x_2 \in \R \wedge
 \ldots  \wedge  x_n \in \R       \}
$$

Al $(x_1, x_2, \cdots, x_n) $  se conoce como (n-upla), (vector), (punta) y a $\real{n}$ es el n-ésimo espacio vectorial.

$\real{0} = \{\Ou\}$  espacio vectorial de dimensión $0$.

En $\Rn$ tenemos:

\begin{itemize}
    \item \textbf{Adición:}
    \begin{align*}
        +\colon \Rn \times \Rn &\to \Rn \\
        (x, y) &\mapsto x + y = (x_1 + y_1, \dots, x_n + y_n),
    \end{align*}
    donde $x = (x_1, \dots, x_n)$ e $y = (y_1, \dots, y_n)$, $(x+y)$ es el vector suma.

    \item \textbf{Multiplicación por escalar:}
    \begin{align*}
        \cdot\colon \R \times \Rn &\to \Rn \\
        (\lambda, x) &\mapsto \lambda x = (\lambda x_1, \dots, \lambda x_n),
    \end{align*}
    donde $x = (x_1, \dots, x_n)$ .
\end{itemize}



\ex{
    Verificar que $(\Rn, +, \cdot)$ es un $\R-\text{espacio vectorial}$.\\
    Esto es $(\R, +)$ es un grupo conmutativo y además.
    \begin{itemize}
        \item $(\lambda+u)x = \lambda x + u x$
        \item $\lambda(x+y) = \lambda x + \lambda y$
        \item $\lambda(\mu x) = (\lambda \mu)x$
        \item $1 x = x$
    \end{itemize}

    \rmk{$\Ou = (0,0, \cdots , 0) \in \Rn$ es el vector nulo.}

    Si $x=(x_1, x_2, \cdots, x_n)$
    $\rightarrow  -x = (-x_1, -x_2, \cdots, -x_n) $. Además $ -1(x) \text{ es el inverso aditivo de } x  \text{ y se denota por } -x$
    }

   En $\Rn$, la base canónica es $B = \{ e_1, e_2, \ldots, e_n \}$ donde:
   \[
   \begin{aligned}
   	e_1 &= (1,0,\cdots,0) \in \Rn \\
   	e_2 &= (0,1,\cdots,0) \in \Rn \\
   	e_i &= (0,\cdots,1,\cdots,0) \in \Rn, \quad \forall i \in \{1,2,\ldots,n\} \\
   	e_n &= (0,0,\cdots,1) \in \Rn
   \end{aligned}
   \]
   
\ex{ 
   	Mostrar que $B$ es una base de $\Rn$.\\
   	
   	Sea $x \in \Rn$ con
   	\[
   	\begin{aligned}
        x &= (x_1, x_2, \cdots, x_n)\\
   		x &= (x_1,0,\cdots,0)+(0,x_2,0,\cdots,0)+\cdots+(0,0,\cdots,x_n) \\
   		x &= x_1 e_1 + x_2 e_2 + \cdots + x_n e_n
   	\end{aligned}
   	\]
    ¿$B$ es linealmente independiente?\\
    Sea $\lambda_1, \cdots, \lambda_n \in \Rn$ tal que
    $$
    \lambda_1 e_1 + \lambda_2 e_2 + \cdots + \lambda_n e_n = \Ou
    $$
    $$
     (\lambda_1, \cdots, \lambda_n) = \Ou = (0,0,\cdots,0)    
    $$
    $$
    \rightarrow \lambda_1 = 0 \wedge \lambda_2 = 0 
    $$

   }
   



\newpage

\defn{Definition Name}{
    A defintion.
}


\thmr{Theorem Name}{mybigthm}{
    A theorem.
}


\lem{Lemma Name}{
    A lemma.
}

\fact{
    A fact.
}

\cor{
    A corollary.
}

\prop{
    A proposition.
}

\clmp{}{
    A claim.
}{
    A reference to Theorem~\ref{thm:mybigthm}
}

\pf{
    Veniam velit incididunt deserunt est proident consectetur non velit ipsum voluptate nulla quis. Ea ullamco consequat non ad amet cupidatat cupidatat aliquip tempor sint ea nisi elit dolore dolore. 

    Laboris labore magna dolore eiusmod ea ex et eiusmod laboris. Et aliquip cupidatat reprehenderit id officia pariatur. 
}

\ex{
    Nostrud esse occaecat Lorem dolore laborum exercitation adipisicing eu sint sunt et. Excepteur voluptate consectetur qui ex amet esse sunt ut nostrud qui proident non. Ipsum nostrud ut elit dolor. Incididunt voluptate esse et est labore cillum proident duis.
}

\rmk{
    Some remark.
}

\rmkb{
    Some more remark.
}

\section{Pictures}

\begin{figure}[H]
    \center
    \includegraphics[scale=0.1]{img/loo.jpg}
    \caption{Waterloo, ON}
\end{figure}

\end{document}
