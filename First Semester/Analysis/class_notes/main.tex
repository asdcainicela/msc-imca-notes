\documentclass[oneside]{book}
\usepackage[left=1 in, right=1 in]{geometry}
\usepackage{amsmath, amsthm, amssymb, amsfonts}
\usepackage{mathtools}
\usepackage{dsfont} % Usar \mathds{O}
\usepackage{thmtools}
\usepackage{graphicx}
\usepackage{setspace}
\usepackage{geometry}
\usepackage{float}
\usepackage{hyperref}
\usepackage[utf8]{inputenc}
\usepackage[english]{babel}
\usepackage{framed}
\usepackage[dvipsnames]{xcolor}
\usepackage{environ}
\usepackage{tcolorbox}
\usepackage{extpfeil}
%\usepackage[all]{xy} %Para flechas, muy antiguo
\usepackage{tikz}          % núcleo de TikZ
\usetikzlibrary{cd,matrix,arrows.meta} % tikz-cd + matrices + flechas
\usepackage{tikz-cd}
\usepackage{mathrsfs}  % letras cursivas L
\usepackage{nicematrix}
% ----------------- Configuración de Tipografía -----------------
\usepackage{mathpazo} % Fuente Palatino para matemáticas y texto
% \usepackage[utopia]{mathdesign}
% \usepackage[nomath,fulloldstylenums,fulloldstyle]{kpfonts} % Fuente alternativa
%\usepackage[nomath,fulloldstylenums]{kpfonts}
\usepackage{parskip} %% Evitar indentación en nuevos párrafos

\usepackage{physics}
\usepackage{pgfplots} % dibujar las bolas
\usetikzlibrary{angles,quotes}
\usetikzlibrary{intersections, backgrounds, fillbetween}
\tcbuselibrary{theorems,skins,breakable}

\setstretch{1.2}
\geometry{
    textheight=9in,
    textwidth=5.5in,
    top=1in,
    headheight=12pt,
    headsep=25pt,
    footskip=30pt
}

% Variables
\def\notetitle{Análisis I}
\def\noteauthor{
    \textbf{Student} 
     Gerald Cainicela\\
    Master's program in mathematics}
\def\notedate{Semester}

% The theorem system and user-defined commands  
% Theorem System
% The following boxes are provided:
%   Definition:     \defn 
%   Theorem:        \thm 
%   Lemma:          \lem
%   Corollary:      \cor
%   Proposition:    \prop   
%   Claim:          \clm
%   Fact:           \fact
%   Proof:          \pf
%   Example:        \ex
%   Remark:         \rmk (sentence), \rmkb (block)
% Suffix
%   r:              Allow Theorem/Definition to be referenced, e.g. thmr
%   p:              Add a short proof block for Lemma, Corollary, Proposition or Claim, e.g. lemp
%                   For theorems, use \pf for proof blocks

% Definition
\newtcbtheorem[number within=section]{mydefinition}{Definition}
{
    enhanced,
    frame hidden,
    titlerule=0mm,
    toptitle=1mm,
    bottomtitle=1mm,
    fonttitle=\bfseries\large,
    coltitle=black,
    colbacktitle=green!20!white,
    colback=green!10!white,
}{defn}

\NewDocumentCommand{\defn}{m+m}{
    \begin{mydefinition}{#1}{}
        #2
    \end{mydefinition}
}

\NewDocumentCommand{\defnr}{mm+m}{
    \begin{mydefinition}{#1}{#2}
        #3
    \end{mydefinition}
}

% Theorem
\newtcbtheorem[use counter from=mydefinition]{mytheorem}{Theorem}
{
    enhanced,
    frame hidden,
    titlerule=0mm,
    toptitle=1mm,
    bottomtitle=1mm,
    fonttitle=\bfseries\large,
    coltitle=black,
    colbacktitle=cyan!20!white,
    colback=cyan!10!white,
}{thm}

\NewDocumentCommand{\thm}{m+m}{
    \begin{mytheorem}{#1}{}
        #2
    \end{mytheorem}
}

\NewDocumentCommand{\thmr}{mm+m}{
    \begin{mytheorem}{#1}{#2}
        #3
    \end{mytheorem}
}

\newenvironment{thmpf}{
	{\noindent{\it \textbf{Proof for Theorem.}}}
	\tcolorbox[blanker,breakable,left=5mm,parbox=false,
	before upper={\parindent15pt},
	after skip=10pt,
	borderline west={1mm}{0pt}{cyan!20!white}]
}{
	\textcolor{cyan!20!white}{\hbox{}\nobreak\hfill$\blacksquare$} 
	\endtcolorbox
}

\NewDocumentCommand{\thmrpf}{m+m+m+m}{
	\begin{mytheorem}{#1}{#2}
		#3
	\end{mytheorem}
	
	\begin{thmpf}
		#4
	\end{thmpf}
}

% Lemma
\newtcbtheorem[use counter from=mydefinition]{mylemma}{Lemma}
{
    enhanced,
    frame hidden,
    titlerule=0mm,
    toptitle=1mm,
    bottomtitle=1mm,
    fonttitle=\bfseries\large,
    coltitle=black,
    colbacktitle=violet!20!white,
    colback=violet!10!white,
}{lem}

\NewDocumentCommand{\lem}{m+m}{
    \begin{mylemma}{#1}{}
        #2
    \end{mylemma}
}

\newenvironment{lempf}{
	{\noindent{\it \textbf{Proof for Lemma}}}
	\tcolorbox[blanker,breakable,left=5mm,parbox=false,
    before upper={\parindent15pt},
    after skip=10pt,
	borderline west={1mm}{0pt}{violet!20!white}]
}{
    \textcolor{violet!20!white}{\hbox{}\nobreak\hfill$\blacksquare$} 
    \endtcolorbox
}

\NewDocumentCommand{\lemp}{m+m+m}{
    \begin{mylemma}{#1}{}
        #2
    \end{mylemma}

    \begin{lempf}
        #3
    \end{lempf}
}

% Corollary
\newtcbtheorem[use counter from=mydefinition]{mycorollary}{Corollary}
{
    enhanced,
    frame hidden,
    titlerule=0mm,
    toptitle=1mm,
    bottomtitle=1mm,
    fonttitle=\bfseries\large,
    coltitle=black,
    colbacktitle=orange!20!white,
    colback=orange!10!white,
}{cor}

\NewDocumentCommand{\cor}{+m}{
    \begin{mycorollary}{}{}
        #1
    \end{mycorollary}
}

\newenvironment{corpf}{
	{\noindent{\it \textbf{Proof for Corollary.}}}
	\tcolorbox[blanker,breakable,left=5mm,parbox=false,
    before upper={\parindent15pt},
    after skip=10pt,
	borderline west={1mm}{0pt}{orange!20!white}]
}{
    \textcolor{orange!20!white}{\hbox{}\nobreak\hfill$\blacksquare$} 
    \endtcolorbox
}

\NewDocumentCommand{\corp}{m+m+m}{
    \begin{mycorollary}{}{}
        #1
    \end{mycorollary}

    \begin{corpf}
        #2
    \end{corpf}
}

% Proposition
\newtcbtheorem[use counter from=mydefinition]{myproposition}{Proposition}
{
    enhanced,
    frame hidden,
    titlerule=0mm,
    toptitle=1mm,
    bottomtitle=1mm,
    fonttitle=\bfseries\large,
    coltitle=black,
    colbacktitle=yellow!30!white,
    colback=yellow!20!white,
}{prop}

\NewDocumentCommand{\prop}{+m}{
    \begin{myproposition}{}{}
        #1
    \end{myproposition}
}

\newenvironment{proppf}{
	{\noindent{\it \textbf{Proof for Proposition.}}}
	\tcolorbox[blanker,breakable,left=5mm,parbox=false,
    before upper={\parindent15pt},
    after skip=10pt,
	borderline west={1mm}{0pt}{yellow!30!white}]
}{
    \textcolor{yellow!30!white}{\hbox{}\nobreak\hfill$\blacksquare$} 
    \endtcolorbox
}

\NewDocumentCommand{\propp}{+m+m}{
    \begin{myproposition}{}{}
        #1
    \end{myproposition}

    \begin{proppf}
        #2
    \end{proppf}
}

% Claim
\newtcbtheorem[use counter from=mydefinition]{myclaim}{Claim}
{
    enhanced,
    frame hidden,
    titlerule=0mm,
    toptitle=1mm,
    bottomtitle=1mm,
    fonttitle=\bfseries\large,
    coltitle=black,
    colbacktitle=pink!30!white,
    colback=pink!20!white,
}{clm}


\NewDocumentCommand{\clm}{m+m}{
    \begin{myclaim*}{#1}{}
        #2
    \end{myclaim*}
}

\newenvironment{clmpf}{
	{\noindent{\it \textbf{Proof for Claim.}}}
	\tcolorbox[blanker,breakable,left=5mm,parbox=false,
    before upper={\parindent15pt},
    after skip=10pt,
	borderline west={1mm}{0pt}{pink!30!white}]
}{
    \textcolor{pink!30!white}{\hbox{}\nobreak\hfill$\blacksquare$} 
    \endtcolorbox
}

\NewDocumentCommand{\clmp}{m+m+m}{
    \begin{myclaim*}{#1}{}
        #2
    \end{myclaim*}

    \begin{clmpf}
        #3
    \end{clmpf}
}

% Fact
\newtcbtheorem[use counter from=mydefinition]{myfact}{Fact}
{
    enhanced,
    frame hidden,
    titlerule=0mm,
    toptitle=1mm,
    bottomtitle=1mm,
    fonttitle=\bfseries\large,
    coltitle=black,
    colbacktitle=purple!20!white,
    colback=purple!10!white,
}{fact}

\NewDocumentCommand{\fact}{+m}{
    \begin{myfact}{}{}
        #1
    \end{myfact}
}


% Proof
\NewDocumentCommand{\pf}{+m}{
    \begin{proof}
        [\noindent\textbf{Proof.}]
        #1
    \end{proof}
}

% Example
\newenvironment{example}{
    \par
    \vspace{5pt}
    \noindent\textbf{Example.}
    \begin{tcolorbox}[
        blanker, breakable, left=5mm, parbox=false,
        before upper={\parindent15pt},
        after skip=10pt,
        borderline west={1mm}{0pt}{orange!30!white}
        ]
}{
    \end{tcolorbox}
    \vspace{5pt}
}

\NewDocumentCommand{\ex}{+m}{
    \begin{example}
        #1
    \end{example}
}

\NewDocumentCommand{\expf}{m+m}{
	\begin{example}
		#1
	\end{example}
	
	\begin{tcolorbox}[
		blanker, breakable, left=5mm, parbox=false,
		before upper={\parindent15pt},
		after skip=10pt,
		borderline west={1mm}{0pt}{orange!30!white}
		]
		\noindent{\it \textbf{Proof for Example.}} #2
		\textcolor{orange!30!white}{\hfill$\blacksquare$}
	\end{tcolorbox}
}


% Remark
\NewDocumentCommand{\rmk}{+m}{
    {\it \color{blue!50!white}#1}
}

\newenvironment{remark}{
    \par
    \vspace{5pt}
    \noindent\textbf{Remark.}
    \begin{tcolorbox}[
        blanker, breakable, left=5mm, parbox=false,
        before upper={\parindent15pt},
        after skip=10pt,
        borderline west={1mm}{0pt}{gray!30!white}
        ]
}{
    \end{tcolorbox}
    \vspace{5pt}
}

\NewDocumentCommand{\rmkb}{+m}{
    \begin{remark}
        #1
    \end{remark}
}

%ejercicios
\newenvironment{exercise}{
	\par\vspace{5pt}
	\noindent\textbf{Exercise.}
	\begin{tcolorbox}[
		blanker, breakable, left=5mm, parbox=false,
		before upper={\parindent15pt},
		after skip=10pt,
		borderline west={1mm}{0pt}{red!30!white}
		]
	}{
	\end{tcolorbox}
	\vspace{5pt}
}

\NewDocumentCommand{\exer}{+m}{
	\begin{exercise}
		#1
	\end{exercise}
}

\NewDocumentCommand{\exerpf}{m+m}{
	\begin{exercise}
		#1
	\end{exercise}
	
	\begin{tcolorbox}[
		blanker, breakable, left=5mm, parbox=false,
		before upper={\parindent15pt},
		after skip=10pt,
		borderline west={1mm}{0pt}{red!30!white}
		]
		\noindent{\it \textbf{Proof / Solution.}} #2
		\textcolor{red!30!white}{\hfill$\blacksquare$}
	\end{tcolorbox}
}



\newcommand{\lcm}{\operatorname{lcm}}
\newcommand{\R}{\mathbb{R}} 
\newcommand{\Rn}{\mathbb{R}} 
\newcommand{\real}[1]{\mathbb{R}^{#1}}
\newcommand{\N}{\mathbb{N}} 
\newcommand{\Ou}{\mathds{O}} %Requiere \usepackage{dsfont}
\newcommand{\Ll}{\mathscr{L}} %Requiere \usepackage{mathrsfs}
\newcommand{\Lineal}[2]{\Ll{L}\left( #1, #2 \right)}


% ------------------------------------------------------------------------------

\begin{document}
\title{\textbf{
		\LARGE{\notetitle} \vspace*{10\baselineskip}}
}
\author{\noteauthor}
\date{\notedate}

%\maketitle
\newpage

%\tableofcontents
\newpage

% ------------------------------------------------------------------------------

%\chapter{Clase I}

\section{Topología}

Topología en $\Real{n}$, con $n \in \N$

$$
\Real{n} =  \underbrace{\R \times \R \times \cdots \R }_{n \text{veces}}
$$
$$
\Real{n} = \{ (x_1, x_2, \cdots, x_n); x_1 \in \R \wedge  x_2 \in \R \wedge
 \ldots  \wedge  x_n \in \R       \}
$$

Al $(x_1, x_2, \cdots, x_n) $  se conoce como (n-upla), (vector), (punta) y a $\Real{n}$ es el n-ésimo espacio vectorial.

$\Real{0} = \{\Ou\}$  espacio vectorial de dimensión $0$.

En $\Rn$ tenemos:

\begin{itemize}
    \item \textbf{Adición:}
    \begin{align*}
        +\colon \Rn \times \Rn &\to \Rn \\
        (x, y) &\mapsto x + y = (x_1 + y_1, \dots, x_n + y_n),
    \end{align*}
    donde $x = (x_1, \dots, x_n)$ e $y = (y_1, \dots, y_n)$, $(x+y)$ es el vector suma.

    \item \textbf{Multiplicación por escalar:}
    \begin{align*}
        \cdot\colon \R \times \Rn &\to \Rn \\
        (\lambda, x) &\mapsto \lambda x = (\lambda x_1, \dots, \lambda x_n),
    \end{align*}
    donde $x = (x_1, \dots, x_n) \text{ y } \lambda \in \R$ .
\end{itemize}



\ex{
    Verificar que $(\Rn, +, \cdot)$ es un $\R-\text{espacio vectorial}$. 
    Esto es $(\R, +)$ es un grupo conmutativo y además.
    \begin{itemize}
        \item $(\lambda+u)x = \lambda x + u x$
        \item $\lambda(x+y) = \lambda x + \lambda y$
        \item $\lambda(\mu x) = (\lambda \mu)x$
        \item $1 x = x$
    \end{itemize}

    \rmk{$\Ou = (0,0, \cdots , 0) \in \Rn$ es el vector nulo.}

    Si $x=(x_1, x_2, \cdots, x_n)$
    $\rightarrow  -x = (-x_1, -x_2, \cdots, -x_n) $.  
    Además $ -1(x) \text{ es el inverso aditivo de } x  \text{ y  es denotado por  } -x$
    }

   En $\Rn$, la base canónica es $B = \{ e_1, e_2, \ldots, e_n \}$ donde:
   \[
   \begin{aligned}
   	e_1 &= (1,0,\cdots,0) \in \Rn \\
   	e_2 &= (0,1,\cdots,0) \in \Rn \\
   	e_i &= (0,\cdots, \underset{\substack{\uparrow \\ i\text{-ésimo}}}{1},\cdots,0) \in \Rn, \quad \forall i \in \{1,2,\ldots,n\} \\
   	e_n &= (0,0,\cdots,1) \in \Rn
   \end{aligned}
   \]
    
\ex{ 
   	Mostrar que $B$ es una base de $\Rn$.\\
   	
   	Sea $x \in \Rn$ con
   	\[
   	\begin{aligned}
        x &= (x_1, x_2, \cdots, x_n)\\
   		x &= (x_1,0,\cdots,0)+(0,x_2,0,\cdots,0)+\cdots+(0,0,\cdots,x_n) \\
   		x &= x_1 e_1 + x_2 e_2 + \cdots + x_n e_n
   	\end{aligned}
   	\]
    ¿$B$ es linealmente independiente?\\
    Sea $\lambda_1, \cdots, \lambda_n \in \Rn$ tal que
    $$
    \lambda_1 e_1 + \lambda_2 e_2 + \cdots + \lambda_n e_n = \Ou
    $$
    $$
     (\lambda_1, \cdots, \lambda_n) = \Ou = (0,0,\cdots,0)    
    $$
    $$
    \rightarrow \lambda_1 = 0 \wedge \lambda_2 = 0 \wedge \cdots \wedge \lambda_n = 0
    $$
   }
   
 $\Lineal{\Real{m}}{\Real{n}} = \{T: \Rm \longrightarrow \Rn : T 
 \text{ es una transformación lineal}\}$

 $M(n\times m)$  conjunto de matrices de orden $n\times m $ con entradas reales.
\begin{align*}
    \psi \colon \Lineal{\Rm}{\Rn} &\longrightarrow M(n\times m) \\
    A & \longmapsto (A)
\end{align*}
Considere \begin{align*} 
    B =& \{ e_1, e_2, \cdots, e_m\} &\subset \Rm \text{base canónica}\\
    B' =& \{ \overline{e_1} , \overline{e_2}, \cdots, \overline{e_n}\} &\subset \Rn \text{base canónica}\\
    \end{align*}
$$
Ae_j = a_{1j}\overline{e_1}+a_{2j}\overline{e_2}+\cdots+a_{nj}\overline{e_n}
$$

\NiceMatrixOptions%
 {code-for-last-row = \scriptstyle \rotate ,
 code-for-last-col = \scriptstyle }

\[
\psi(A)=
\begin{bNiceMatrix}[last-row=5]
    \Block[fill=red!15,rounded-corners]{4-1}{}
    a_{11} & a_{12} & \cdots & 
    \Block[fill=blue!15,rounded-corners]{4-1}{}
    a_{1m} \\
    a_{21} & a_{22} & \cdots & a_{2m} \\
    \vdots & \vdots & \ddots & \vdots \\
    a_{n1} & a_{n2} & \cdots & a_{nm}\\
    \text{Coef. de} Ae_1& & & \text{Coef. de} Ae_m
\end{bNiceMatrix}
\]

\ex{ Mostrar que $\psi$ es una biyección. más aún, e sun isomorfismo entre espacios vectoriales.

Decimos que  $\psi \colon \Rm \times \Rn \longrightarrow \Rp$ es bilineal si
\begin{align*}
    \psi(x+x', y) &= \psi(x,y) + \psi(x',y) ; \, &\forall x, x' \in \Rm, \forall y \in \Rn \\
    \psi(\lambda x, y) &= \lambda \psi(x,y) ; &\forall x \in \Rm, \forall y \in \Rn, \forall \lambda \in \R \\
    \psi(x, y+y') &= \psi(x,y) + \psi(x,y') ; & \\
    \psi(x, \lambda y) &= \lambda \psi(x,y) ; &
\end{align*}
  
}

Dados \begin{align*}
    x = (x_1,\ldots,x_m) \in \Rm \\
    y= (y_1,\ldots,y_n) \in \Rn
\end{align*} 

\begin{align*}
    \psi(x,y) & = \psi\pqty{\sum_{i=1}^{m} x_i e_i, \sum_{j=1}^{n} y_j \overline{e_j}} \\
    & = \sum_{i=1}^m x_i \psi\pqty{ e_i, \sum_{j=1}^n y_j \overline{e_j}} \\
    & = \sum_{i=1}^m x_i \sum_{j=1}^n y_j \psi\pqty{e_i, \overline{e_j}} \\
    & = \sum_{i=1}^m \sum_{j=1}^n x_i y_j \psi\pqty{e_i, \overline{e_j}} \text{ (por bilinealidad)}
\end{align*}

\section{Producto Interno y Norma en $\Rn$}
Sea $E$ un espacio vectorrial sobre $\R$ 
\subsection{Producto Interno}
Un producto interno sobre $E$ es una función
 
\defn{Producto interno}{
Un \textbf{producto interno} sobre un espacio vectorial \( E \) es una función

\begin{align*}
    \langle \cdot, \cdot \rangle \colon E \times E &\longrightarrow \mathbb{R} \\
    (x, y) &\longmapsto \langle x, y \rangle 
\end{align*}

que satisface las siguientes propiedades:

\begin{description}
    \item[(i) Linealidad en la primera componente:]
    \begin{align}
        \langle x + x', y \rangle &= \langle x, y \rangle + \langle x', y \rangle, &&\forall x, x', y \in E \label{eq:linealidad1} \\
        \langle \lambda x, y \rangle &= \lambda \langle x, y \rangle, &&\forall x, y \in E, \, \forall \lambda \in \mathbb{R} \label{eq:linealidad2}
    \end{align}

    \item[(ii) Simetría:]
    \begin{equation}
        \langle x, y \rangle = \langle y, x \rangle, \quad \forall x, y \in E \label{eq:simetria}
    \end{equation}

    \item[(iii) Positividad definida:]
    \begin{equation}
        \langle x, x \rangle > 0, \quad \forall x \in E. \forall x \neq \Ou \label{eq:positividad}
    \end{equation}
\end{description}
}

Como consecuencia de las propiedades \eqref{eq:linealidad1} y \eqref{eq:simetria}, también se cumple:

\begin{itemize}
    \item \(\langle x, y + y' \rangle = \langle x, y \rangle + \langle x, y' \rangle\), \quad \(\forall x, y, y' \in E\)
    \item \(\langle x, \lambda y \rangle = \lambda \langle x, y \rangle\), \quad \(\forall x, y \in E, \, \forall \lambda \in \mathbb{R}\)
\end{itemize}

\rmk{
En otras palabras, el producto interno \( \langle \cdot, \cdot \rangle \) es \textbf{bilineal} y \textbf{simétrico}.
}

\ex{
Sea $\langle . , . \rangle  \colon \Rn \times \Rn \rightarrow \R$\\
Si  $	x = (x_1,\ldots,x_m) \in \Rm$ e $	y= (y_1,\ldots,y_n) \in \Rn $

$$
\rightarrow \langle x, y \rangle = \sum_{i = 1}^{n} x_i y_i  \text{ (Producto interno euclideano)}
$$

}

\defn{Ortogonalidad}{
	Sean \( x, y \in \mathbb{R}^n \). Decimos que \( x \) y \( y \) son \textbf{ortogonales} si
	\begin{equation}
		\langle x, y \rangle = 0. \label{eq:def-ortogonalidad}
	\end{equation}
}


\subsection{Norma}

\defn{Norma}{
	Una \textbf{norma} sobre el espacio vectorial \( E \) es una función
	\begin{align}
		\|\cdot\| \colon E &\longrightarrow \mathbb{R} \label{eq:norma-def} \\
		x &\longmapsto \|x\| \nonumber
	\end{align}
	que satisface las siguientes propiedades:
	\begin{description}
		\item[(i) Desigualdad triangular:]
		\begin{equation}
			\|x + y\| \leq \|x\| + \|y\|, \quad \forall x, y \in E
			 \label{eq:norma-triangular-i}
		\end{equation}
		
		\item[(ii) Homogeneidad absoluta:]
		\begin{equation}
			\|\alpha x\| = |\alpha| \cdot \|x\|, \quad \forall \alpha \in \mathbb{R}, \, \forall x \in E \label{eq:norma-homogeneidad-ii}
		\end{equation}
		
		\item[(iii) Positividad definida:]
		\begin{equation}
			\|x\| > 0 \quad \Longleftrightarrow \quad x \neq 0 \label{eq:norma-positividad-iii}
		\end{equation}
	\end{description}
}


\begin{itemize}
\item De \eqref{eq:norma-homogeneidad-ii} 
$$
\norm{\Ou} = \norm{0 x}=\abs{0}\norm{x} = 0
$$ 
\item De \eqref{eq:norma-homogeneidad-ii} 
$$
\norm{-x}=\norm{(-1)x}=\abs{-1}\norm{x}=\norm{x}
$$
\item De \eqref{eq:norma-triangular-i} 
$$
0 = \norm{\Ou} =\norm{x+(-x)} \leq \norm{x}+\underbrace{\norm{-x}}_{\norm{x}}
$$
\begin{equation}
	\therefore 0\leq \norm{x}, \quad \forall x \in E 
	\label{eq:norma-positividad2}
\end{equation}
\item  De \eqref{eq:norma-positividad2} y \eqref{eq:norma-positividad-iii} es equivalente a
$$
\norm{x} = 0 \longleftrightarrow x =\Ou
$$
\end{itemize}

\ex{
La norma euclideana en $\Rn$,  
\begin{align*}
	\norm{\cdot} \colon \mathbb{R}^n &\longrightarrow \mathbb{R} \\
	x = (x_1, \ldots, x_n) &\longmapsto \norm{x} = \sqrt{x_1^2 + \cdots + x_n^2} = \sqrt{\langle x, x \rangle}
\end{align*}


Mostemos que $\norm{.}$ es una norma en $\Rn$\\
Para \eqref{eq:norma-homogeneidad-ii} es inmediato, \eqref{eq:norma-positividad-iii} hay que negar $x \neq 0$, o sea $x = 0$\\
Solo falta mostrar \eqref{eq:norma-triangular-i}
Sean $x,y \in \R, y \neq 0 \rightarrow \inner{y}{y} >0 $\\
¿Para qué $\lambda \in \R$, se tiene que  $\inner{y}{x-\lambda y } = 0$

\begin{align*}
	&\leftrightarrow \inner{y}{x} - \lambda \inner{y}{y} = 0\\
	&\leftrightarrow \frac{\inner{y}{x}}{ \inner{y}{y}} = \lambda
\end{align*}
\begin{align*}
	\norm{x}^2 &= \norm{\lambda y +x-\lambda y}^2\\
	& =  \inner{xy+ x-\lambda y}{xy+ x-\lambda y}\\
	& = \abs{\lambda}^2 \norm{y}^2+\underbrace{\norm{x-\lambda y}^2 }_{\geq 0}
\end{align*}
$$
\rightarrow \norm{x}^2 \geq \abs{\lambda}^2 \norm{y}^2 = \frac{\abs{\inner{x}{y}}^2}{\norm{y}^2}
$$
$$
\rightarrow \norm{x}^2 \norm{y}^2 \geq \abs{\inner{x}{y}}^2
$$
$$
\rightarrow \norm{x}\norm{y} \geq \abs{\inner{x}{y}}, \quad \forall x,y \in \Rn
$$
Desigualdad de Cauchy-Schwarz.\\
Además, la igualdad se da cuando uno de los vectores es multiplo del otro o $x, y$ es linealmente dependiente.
}

\clmp{}{
	$\forall\,  x, y \in \Rn, \norm{x+y} \leq \norm{x}+ \norm{y}$
}{
 Sean $x, y \in \Rn$
 \begin{align*}
 	\norm{x+y}^2 & = \inner{x+y}{x+y}\\
 	& = \norm{x}^2+2\inner{x}{y}+\norm{y}^2\\
 	& \leq  \norm{x}^2+2\norm{x}\norm{y}+\norm{y}^2  = (\norm{x}+\norm{y})^2
 \end{align*}
 $$
 \rightarrow \norm{x+y} \leq \norm{x}+\norm{y}
 $$
}


Dados $x,y \in \Rn, $   $d(x,y)=\norm{x-y}$
\begin{align*}
	d \colon \Rn \times \Rn &\longrightarrow \R\\
	(x,y) &\longmapsto d(x,y)= \norm{x-y}; \quad d(x,y) \text{ es una métrica}
\end{align*}

\begin{itemize}
	\item[(i)] $\forall \, x, y \in \Rn, d(x,y)\geq 0$
	\item[(ii)]  $\forall \, x, y \in \Rn, d(x,y) = d(y,x)  $ simetría
	\item[(iii)]   $\forall \,  x, y, z  \in \Rn, d(x,y) \leq d(x,z) + d(z,y)    $ 
	\item[(iv)]   $\forall \,  x, y   \in \Rn, d(x,z) =0 \, \leftrightarrow x=y  $
\end{itemize}


\ex{
Sea $x = (x_1, \ldots, x_n) \in \Rn$,
\begin{align}	
	\norm{x}_1 & = \abs{x_1}+\abs{x_2}+\cdots +\abs{x_n} \label{eq:ex-norma-1}\\
	\norm{x}_{\infty} &= \max\{ \abs{x_1}, \abs{x_2}, \cdots, \abs{x_n}\} \nonumber\\
	\norm{x}_2&=\norm{x} = \sqrt{x_1^2+x_2^2+\cdots+ x_n^2} \nonumber
\end{align}
De \eqref{eq:ex-norma-1},
\begin{align*}
	 \norm{x+y}_1 &= \abs{x_1+y_1}+\cdots +\abs{x_n+y_n}\\
	 & \leq \abs{x_1}+\abs{y_1}+\cdots + \abs{x_n}+\abs{y_n} =  \norm{x}_1+\norm{y}_1
\end{align*}
$$
\norm{x} = \sqrt{x_1^2+\cdots x_1^n} \geq \abs{x_1} \rightarrow \norm{x}_{\infty} \leq \norm{x}
$$
$$
\norm{x}_1 \geq \norm{x}
$$
}

\clmp{Ley del paralelogramo}{
	 
	 Si $\norm{.} $ proviene de un producto interno en $\Rn$, entonces $\forall\, x,y \in \Rn$
	 $$
	 \norm{w} = \sqrt{\inner{w}{w}}
	 $$
	 $$
	 \Rightarrow \norm{x+y}^2 +\norm{x-y}^2 = 2(\norm{x}^2+\norm{y}^2)
	 $$
}{
 \begin{align}
 	 \norm{x+y}^2 = \inner{x+y}{x+y} = \norm{x}^2+2\inner{x}{y}+\norm{y}^2 \label{eq:paralelogramo-1}\\
 	 \norm{x-y}^2 = \inner{x+y}{x+y} = \norm{x}^2-2\inner{x}{y}+\norm{y}^2 \label{eq:paralelogramo-2} 
 \end{align}
 
 De \eqref{eq:paralelogramo-1} y \eqref{eq:paralelogramo-2}
 $$
  \norm{x+y}^2 +\norm{x-y}^2 = 2(\norm{x}^2+\norm{y}^2)
 $$
}

\ex{
En $\Real{2}, e_1=(1,0), e_2=(0,1)$\\
Si $\norm{.}_1$ en $\Real{2}$ proviene de un producto interno.
$$
\rightarrow \forall \, x, y \in \Real{2}, \norm{x+y}_1^2 +\norm{x-y}_1^2 = 2(\norm{x}_1^2+\norm{y}_1^2)
$$
Si $x=e_1, \, y= e_2 \longrightarrow 2^2+2^2 = 2(1^2+1^2)$ (absurdo)\\
Por lo tanto, $\norm{.}_1$ no proviene de un prducto interno.
}

Sean $\abs{.} \text{ y } \norm{.} $ dos normas en $\Rn$. Decimos que $\abs{.}$ y $\norm{.}$ son equivalentes si existen $\alpha, \beta \in \R^{+}$ tal que,
$$
\forall\, x \in \Rn, \, \alpha \abs{x} \leq \norm{x}\leq\beta \abs{x}
$$

dada $\norm{.}$ norma en $\Rn$ la \textbf{BOLA ABIERTA } con centro en $a \in \Rn$ y radio $r>0$ es

$$
B^{\norm{.}} (a,r) = \{z \in \Rn\colon \norm{z-a} < r\} 
$$

En $\Real{2}$

$$
B^{\norm{.}} (\Ou,1) = \{(z_1, z_2) \in \Real{2} \colon \norm{(z_1, z_2)} < 1\} 
$$

$$
B^{\norm{.}_{\infty}} (\Ou,1) = \{(z_1, z_2) \in \Real{2} \colon \max{\abs{z_1}, \abs{z_2}} < 1\} 
$$

$$
B^{\norm{.}_{1}} (\Ou,1) = \{(z_1, z_2) \in \Real{2} \colon   \abs{z_1}+ \abs{z_2}< 1\} 
$$

\thmr{ Equivalencia de normas en $\Rn$ }{thm:Teorema1}{
 	
 	En $\Rn$ todas las normas son equivalentes.
 }

\pf{ % Del teorema Theorem~\ref{thm:Teorema1}
	Sea $(x_k)_{k \in \N}$ una sucesión de puntos de $\Rn$
	\begin{align*}
		x \colon \N &\longrightarrow \Rn\\
		k & \longmapsto x(k)= x_k = (x_1^k, x_2^k, \ldots, x_n^k)
	\end{align*}
	Sea $a = (a_1, a_2, \ldots, a_n) \in \Rn $, decimos que $(x_k)_{k \, \in \, \N} $ converge a $a \, \in \Rn$ si,
	 
	$$
	\forall\, \varepsilon >0, \exists k_0 \in \, \N, \forall \, k \in \N , k \geq k_0 \longrightarrow \norm{x_k-a} <\varepsilon
	$$
	$$
	\lim\limits_{k \to \infty} x_k =a \text{ es respecto a la norma  }\quad \norm{x_k-a}_{\infty} \leq \abs{(x_k-a)}
	$$
	
	}
	
	
\rmkb{
	$$
	\norm{x}_{\infty} \leq \norm{x} \leq \norm{x}_1 \leq n\norm{x}_{\infty}\leq n \norm{x}_1
	$$
	$$
	\Rightarrow \norm{x} \leq \norm{x}_1 \leq n \norm{x}
	$$
}


\thmr{Criterio de convergencia en \(\mathbb{R}^n\)}{thm:Teorema2}{
	Sea \((x_k)_{k \in \mathbb{N}} \subset \mathbb{R}^n\), donde
	\[
	x_k = (x_1^k, x_2^k, \ldots, x_n^k) \quad \forall\, k \in \mathbb{N},
	\]
	y sea \(a = (a_1, a_2, \ldots, a_n) \in \mathbb{R}^n\).
	
	Entonces,
	\[
	\lim_{k \to \infty} x_k = a \quad \text{si y solo si} \quad \forall\, i \in \{1, \ldots, n\},\; \lim_{k \to \infty} x_i^k = a_i.
	\]
}

\pf{  $(\Rightarrow)$ Supongamos que $\lim\limits_{k\to \infty} x_k =a$\\
	Dado,
	 $
	 \varepsilon > 0, \exists\, k_0 \in \N, \forall\, k \in \N
	 $
	 
	 $$
	 k\geq k_0 \longrightarrow \abs{x_i^k-a} \leq  \norm{x_k-a}_{\infty} \leq \norm{x_{k}-a} < \varepsilon, \quad \forall \, i \in \{1, \ldots, n\}
	 $$
	 $$
	\therefore  \lim\limits_{k\to \infty} x_i^k =a_i,\quad \forall\, i \in \{1, \ldots, n\}
	 $$
	 
	 $(\Leftarrow)$ Suponga que $\forall \, i \in \{1,\ldots, n\}$
	 
	 $$
	 \lim\limits_{k\to \infty} x_i^k =a_i
	 $$
	 Dado $\varepsilon >0, \forall\, i \in \{1,\ldots, n\} \, \exists \, k_i^o \in \N$ tal que $\forall \, k \in \N, \varepsilon_0 =\frac{\varepsilon}{\sqrt{n}}$
	 
	  
	 \begin{align*}
	 	k \geq k_i^o &\rightarrow \abs{x_i^k -a_i} < \varepsilon_0\\
	 	&\rightarrow \abs{x_i^k -a_i}^2 < \varepsilon_0^2
	 \end{align*}
	 Sea $k_0 = \max \{k_1^0,\, k_2^0, \ldots, \,k_n^0\}$ si $	k \geq k_0 \geq k_i^0  \; \forall\; i \in \{1, 2, \ldots, n\}$
	 
	 $$
	 \rightarrow \sqrt{\sum_{i=1}^{n}\abs{x_i^k-a_i}^2}< \sqrt{n \varepsilon_0^2}
	 $$
	 $$
	 \rightarrow \norm{x_i^k -a} < \sqrt{n} \varepsilon_0 =\varepsilon
	 $$
	 
	%Theorem~\ref{{thm:Teorema1}
		
	}



























 
%\chapter{Clase 2}

\section{Equivalante Norms}
Teníamos que, $\forall\, x \in \Rn$,
$$
\norm{x}_{\infty} \leq \norm{x}_2 \leq \norm{x}_1 \leq n \norm{x}_{\infty}
$$
Estas tres normas son equivalentes.

Dada $(x_k)_{k \in \N} \subset \Rn$, con $\forall \, k\in \N, x_k = (k_1^k, \, x_2^k, \ldots, x_n^k) \in \Rn, \; a = (a_1, \ldots ,a_n)\in \Rn$

$
\lim\limits_{k \to \infty} x_k = a $ significa que $\forall\; \varepsilon>0,\, \exists k_0 \in \N, \, \forall k \in \N
$
\begin{align*}
	k \geq k_0 \longrightarrow &\norm{x_k-a} < \varepsilon\\
	& \abs{ \norm{x_k-a}-0} < \varepsilon,  \quad \norm{x_k-a} \in \R 
\end{align*}

\thmr{Equivalencia de la convergencia en \(\mathbb{R}^n\)}{teorema3}{
	Sea \( (x_k)_{k \in \N} \subset \mathbb{R}^n \) y \( a \in \mathbb{R}^n \). Entonces:
	$$
	\lim_{k \to \infty} x_k = a \quad \text{en } \norm{\cdot}_2 
	\iff 
	\forall i \in \{1, \ldots, n\}, \quad \lim_{k \to \infty} x_i^k = a_i
	$$
}

\pf{Prueba del teorema \ref{thm:teorema3}\\
	Sea una subsucesión $(x_k)_{k\in \, \N}=(x_1, x_2, x_3, x_4, x_5, x_6, x_7, \ldots) \subset \Rn $,
	
	\begin{itemize}
		\item Una subsucesión de \((x_k)_{k \in \mathbb{N}}\) es una sucesión de la forma:
		\[
		(x_{i_k})_{k \in \mathbb{N}} = (x_{i_1}, x_{i_2}, x_{i_3}, \ldots)
		\]
		tal que $(i_k)_{k \in \mathbb{N}} \subset \mathbb{N}, \{i_1< i_2< i_3< \ldots\} = \N' \subset \N$, es estrictamente creciente.
	  
		\begin{itemize}
			\item Función de índices:
			 \begin{align*}
			 	i \colon \N & \longrightarrow \N\\
			 	k &\longmapsto i(k)=i_k
			 \end{align*}
			
			\item Sucesión original como función:
			 \begin{align*}
			 	x \colon \N & \longrightarrow \Rn\\
			 	k &\longmapsto x(k)=x_k
			 \end{align*}
			\item Subsucesión como composición:
			\begin{align*}
				 x \circ i \colon \N &\longrightarrow \Rn\\
				  k &\longmapsto (x \circ i)(k) = x(i(k)) = x_{i_k}
			\end{align*}
			\item Notación alternativa para la subsucesión:
			\[
			(x_{i_k})_{k \in \N} = (x_p)_{p \in \N'}, \quad \text{donde } \N' = \{i_1, i_2, i_3, \ldots\} \subset \mathbb{N}
			\]
		\end{itemize}
		\item   Sea \(X \subset \mathbb{R}^n\). Decimos que \(X\) es \textbf{acotado con respecto a la norma infinito}  
		
		\begin{equation}
			\exists \; c>0 \text{ tal que } X \subset B^{\norm{\cdot}_\infty}(\vb{0}, c) = \left\lbrace z \in \mathbb{R}^n : \norm{z - \vb{0}}_\infty \leq c \right\rbrace
			\label{eq:norm-infty-ball}
		\end{equation}
		\[
		B^{\norm{\cdot}_\infty}(\vb{0}, c) = \left\lbrace z = (z_1, \ldots, z_n) \in \mathbb{R}^n : \max_{1 \leq i \leq n} |z_i| \leq c \right\rbrace
		\]
		\[
		B^{\norm{\cdot}_\infty}(\vb{0}, c) = [-c, c]^n = \prod_{i=1}^n [-c, c].
		\]
		
		  Por lo tanto, la condición de acotamiento se puede expresar también como:
		
		\[
		\exists\, c > 0 \text{ tal que } \forall\, x \in X,\quad \norm{x}_\infty \leq c.
		\]
		
	  %Es decir, todos los vectores \(x = (x_1, \ldots, x_n) \in X\) tienen cada componente dentro del intervalo \([-c, c]\), o equivalentemente:
		\[
		|x_i| \leq c,\quad \text{para todo } i = 1, \ldots, n.
		\]
		\begin{figure}[H]
			\centering
			\begin{tikzpicture}[scale=0.7]
				\begin{axis}[
					width=7cm, height=7cm,
					axis lines=middle,
					xtick=\empty, ytick=\empty,
					axis equal,
					enlargelimits,
					xmin=-1.2, xmax=1.2,
					ymin=-1.2, ymax=1.2,
					thick,
					domain=0:90
					]
					
					%p=inf
					\draw[ultra thick, red!100!blue, fill=red!100!blue, opacity=0.3 ] (axis cs:-1,-1) rectangle (axis cs:1,1);
					
					% Add +1 and -1 labels
					\node[font=\small] at (axis cs:1.1,-0.1) {c};
					\node[font=\small] at (axis cs:-1.1,-0.1) {-c};
					\node[font=\small] at (axis cs:0.1,1.1) {c};
					\node[font=\small] at (axis cs:0.1,-1.1) {-c};		
					
				\end{axis} 
			\end{tikzpicture}
		\end{figure}
	\end{itemize}
}


\prop{
	Sea \(X \subset \mathbb{R}^n\). Entonces:
	
	\[
	X \text{ es acotado (en } \norm{\cdot}_\infty) \iff \forall\, i \in \{1, \ldots, n\},\ \pi_i(X) \text{ está acotado en } \mathbb{R},
	\]
	donde \(\pi_i\colon \mathbb{R}^n \to \mathbb{R}\) es la proyección sobre la \(i\)-ésima coordenada.
	
}
\pf{   \((\Rightarrow)\)  Supongamos que \(X \subset \mathbb{R}^n\) es acotado en la norma infinito.  
 
 Entonces, existe \(c > 0\) tal que:
 \[
 X \subset B^{\norm{\cdot}_\infty}[0, c] = \{x \in \mathbb{R}^n : \norm{x}_\infty \leq c \} = \prod_{i=1}^n [-c, c].
 \]
 
 De esto se deduce que para cada \(i \in \{1, \ldots, n\}\), la proyección \(\pi_i(X)\) satisface:
 \[
 \pi_i(X) \subset [-c, c],
 \]
 es decir, cada coordenada de los vectores en \(X\) está acotada por \(c\).
 
   \((\Leftarrow)\)  (Ejercicio)
 
}

\thmr{Bolzano–Weierstrass en $\Rn$}{teorema4}{
	Sea \( (x_k)_{k \in \mathbb{N}} \subset \mathbb{R}^n \) una sucesión acotada.  
	Entonces, existe una subsucesión \( (x_{k_j})_{j \in \mathbb{N}} \) y un punto \( x \in \mathbb{R}^n \) que  convergen 
	\[
	\lim_{j \to \infty} x_{k_j} = x.
	\]
}

\pf{ Prueba del Theorem \ref{thm:teorema4}\\
	\underline{Caso $n=3$:} 
	\begin{itemize}
		\item Como $(x_k)_{k \in \N}$ es acotada $\rightarrow \exists \, c>0$ tal que $x_k = (x_1^k, \, x_2^k, \, x_3^k) \in \Real{3}$
		
		$(x_1^k)_{k \in \N} \subset \R $ es acotado, pues 
		
		\begin{equation} 
			\forall\, k \in \N, \norm{x_k}_{\infty} \leq c\rightarrow \abs{x_i^k} \leq c, \quad \forall\, i \in \qty{1, 2, 3} 
			\label{eq:teorema4_1}
		\end{equation}
		
		Por \textbf{Bolzano-Weierstrass en $\R$}, $\exists \, (x_1^{i_k})_{k \in \N} \subset ( x_1^k)_{k \in \N}$ tal que $\lim\limits_{k \to \infty} x_1^{i_K} = b_1 \in \R $\\
		 donde $i_1<i_2< i_3<\ldots$
		
		\item $(x_2^{i_k}) \subset (x_2^k)_{k\in \N}$ es acotada, pues $\abs{x_2^k} \leq c, \forall\, k \in \N$
		
		Por \textbf{Bolzano-Weierstrass en $\R$}, $\exists \, (x_2^{i_{j_k}})_{k \in \N} \subset ( x_2^{i_k})_{k \in \N}$ tal que $\lim\limits_{k \to \infty} x_2^{i_{j_K}} = b_2 \in \R $
		
		\item  $(x_3^{i_{j_k}})   $ es acotada por \eqref{eq:teorema4_1} 
		
		Por \textbf{Bolzano-Weierstrass en $\R$}, $\exists \, (x_3^{i_{j_{p_k}}})_{k \in \N} \subset ( x_3^{i_{j_k}})_{k \in \N}$ tal que $\lim\limits_{k \to \infty} x_3^{i_{j_{p_k}}} = b_3 \in \R $ 
	\end{itemize}
	Como $(x_1^{i_{j_{p_k}}})_{k \in \N} \subset (x_1^{i_k})_{k \in \N}$ y $(x_2^{i_{j_{p_k}}})_{k \in \N} \subset (x_2^{i_k})_{k \in \N}$ entonces
	$$
	\lim\limits_{k \to \infty} x_1^{i_{j_{p_k}}}=b_1 \in \R
	$$
	$$
	\lim\limits_{k \to \infty} x_2^{i_{j_{p_k}}}=b_2 \in \R
	$$
	$$
	\lim\limits_{k \to \infty} x_3^{i_{j_{p_k}}}=b_3 \in \R
	$$
	Del theorem \ref{thm:teorema3}
	$$
	 \pqty{x_{i_{j_{p_k}}}} _{k\in \N} \text{ converge } (b_1, b_2, b_3)
	$$
	donde,
	$$
	x_{i_{j_{p_k}}}=\pqty{x_1^{i_{j_{p_k}}}, x_2^{i_{j_{p_k}}}, x_3^{i_{j_{p_k}}}}, \forall k \in \N
	$$
}
 
Sabemos que \( (\mathbb{R}, |\cdot|) \) es un espacio métrico completo.

\begin{itemize}
	\item Sea \( (x_k)_{k \in \mathbb{N}} \subset \mathbb{R}^n \) una sucesión.  
	Entonces, \( (x_k) \) es de Cauchy en \( \mathbb{R}^n \) (respecto a alguna norma \( \|\cdot\| \)) si y solo si:
	\[
	\forall\, \varepsilon > 0,\ \exists\, k_0 \in \mathbb{N}\ \text{tal que}\ \forall\, m, p \in \mathbb{N},\ m, p \geq k_0 \Rightarrow \|x_m - x_p\| < \varepsilon.
	\]
\end{itemize}

\ex{Si $(x_{k})_{k \in \N} \subset  \Rn$ es convergente $\rightarrow (x_k)_{k\in \N}$ es de Cauchy.
	\pf{
		Sea $(x_k)_{k\in \N}$ sucesión convergente en $\Rn$
		$$
		\rightarrow \exists \, a \in \Rn \text{ tal que } \lim\limits_{k \to \infty } x_k =a 
		$$
	Dado $\varepsilon >0, \exists\, k_0 \in \N$
	\begin{align*}
		k\geq k_0 & \rightarrow \norm{x_k-a} < \frac{\varepsilon}{2}\\
		p\geq k_0 & \rightarrow \norm{x_p-a} < \frac{\varepsilon}{2}
	\end{align*}
	\begin{align*} 
	\text{Si }	k, p \geq k_0 \rightarrow \norm{x_k-k_p} &\leq \norm{x_k-a}+\norm{a-x_p}\\
	&< \frac{\varepsilon}{2}+ \frac{\varepsilon}{2}\\
	\norm{x_k-k_p}&<\varepsilon
	\end{align*}
	$\therefore (x_{k})_{k \in \N} $ es de Cauchy.
	}
}

 
\thmr{}{teorema5}{
	El espacio \( (\mathbb{R}^n, \|\cdot\|_{\infty}) \) es completo.
	
	Es decir, si \( (x_k)_{k \in \mathbb{N}} \subset \mathbb{R}^n \) es una sucesión de Cauchy respecto a \( \|\cdot\|_{\infty} \), entonces \( (x_k) \) converge en \( \mathbb{R}^n \).
}

\pf{
Prueba del teorema \ref{thm:teorema5} \\
Sea $x_k = \pqty{x_1^k, x_2^k, \ldots, x_n^k}, \forall\, k\in \N$ tal que $(x_k)_{k \in \mathbb{N}}$ es de Cauchy.

Dado $\varepsilon >0,  \exists \, k_0 \in \N \quad \forall k, p \in \N$
\begin{align*}
	k, p \geq k_0 \longrightarrow \norm{x_k-x_p}_{\infty} &<\varepsilon\\
	\max\limits_{1 \leq i \leq n} \abs{x_i^k-x_i^p}& \varepsilon
\end{align*}
$\forall i \in \{1, 2, \ldots, n\}, \quad k, p \geq k_0 \rightarrow \abs{x_{i}^{k}-x_i^p}<\varepsilon$

$\therefore \pqty{x_i^k}_{k\in \N} \subset \R$ es de cauchy en $\R$ completo $\forall i \in \{1, \ldots, n\} \quad \exists\, a_i \in \R$ tal que $\lim\limits_{k \to \infty} x_i^k = a_i$

del teorema \ref{thm:teorema3} \quad $\lim\limits_{k \to \infty} x_k = a = (a_1, \ldots, a_n)$

}

\thmr{Equivalencia de normas en \( \mathbb{R}^n \)}{teorema6}{
	En \( \mathbb{R}^n \), todas las normas son equivalentes.
	
	Es decir, dadas dos normas \( \abs{.} \) y \( \norm{.} \) en \( \mathbb{R}^n \), existen constantes \( \alpha, \beta \in \mathbb{R}^+ \) tales que:
	\[
	\forall\, x \in \mathbb{R}^n,\quad \alpha  \abs{x} \leq \norm{x}  \leq \beta \abs{x}
	\]
}

\pf{
	Prueba del teorema \ref{thm:teorema5} \\
	Basta con mostrar que cualquier norma $\abs{.}$ es equivalente a $\norm{.}_1$
	
	Sea $\abs{.}$ una norma en $\Rn, \forall \, x \in \Rn$,$\forall x=(x_1, \ldots, x_n)=\pqty{x_1 e_1+\cdots +x_n e_n} \in \Rn$
	\begin{align*}
		\abs{x} = \abs{\sum_{i=1}^{n} x_i e_i} \leq \sum_{i=1}^{n}\abs{x_i}\abs{e_i}&\leq \beta \sum_{i=1}^{n} \abs{x_i}\\
		&\leq \beta \norm{x_i}_1
	\end{align*}
	donde $\max\limits_{1 \leq i \leq n} \abs{e_i}=\beta$
	\begin{equation}
		\forall \, x \in \Rn, \abs{x} \leq \beta\norm{x}_1 
		\label{eq:norma_equival}
	\end{equation}
	
	Solo falta demostrar que $\exists \,\alpha>0, \forall\, x \in \Rn, \quad\alpha \norm{x}_1 \leq \abs{x}$
	
	Por contradicción, supongamos que 
	$$
	\forall \,\alpha>0, \exists\, x \in \Rn, \quad\alpha \norm{x_{\alpha}}_1 > \abs{x_\alpha}
	$$
	para  
	$$
	\alpha = \frac{1}{k}, \exists\, x_k \in \Rn, \frac{1}{k} \norm{x_k}_1>\abs{x_k}
	$$
	$$
	\forall k \in \N,    \exists\, x_k \in \Rn, \frac{1}{k} \norm{x_k}_1>\abs{x_k} \rightarrow x_k \neq 0
	$$
	Hemos construido $(x_k)_{k\in \N } \subset \Rn$ tal que $\norm{x_k}_1>0 \wedge \abs{x_k}>0$
	
	$$
	\forall k \in \N, \frac{1}{k} > \frac{1}{\norm{x_k}_1}\cdot \abs{x_k} = \abs{\frac{1}{\norm{x_k}_1}\cdot x_k} = \abs{z_k}
	$$
	
	Tenemos,
	\begin{align}
		 \forall k \in \N, \quad\norm{z_k}_1 =1 \label{eq:equi-7} \\
		  \forall k \in \N, \quad\abs{z_k}  =\frac{1}{k} \label{eq:equi-8}
	\end{align}
	Como $(z_k)_{k \in \N} \subset \Rn$, de \ref{eq:equi-7}  es acotada respecto a $\norm{.}_1$

	Por \textbf{Bolzano-Weierstrass en $\Rn$}, $\exists (z_{i_k}) \subset (z_k)_{k\in \N}$

Para  $\norm{.}_1 \approx \norm{.}_{\infty}$ y  $\exists \, a \in \Rn$ tales que
$$
\lim\limits_{k\to \infty} z_{i_k}=a
$$
\begin{equation}
 \lim\limits_{k\to \infty} \norm{ z_{i_k} -a }=0 \label{eq:norma_equival-1}
\end{equation}

\rmkb{
$$
\norm{b} = \norm{b-c+c}\leq \norm{b-c}+\norm{c}
$$
$$
\norm{b} -\norm{c}  \leq \norm{b-c} 
$$
$$
\rightarrow \abs{\norm{b} -\norm{c} } \leq \norm{b-c} 
$$

$$
\rightarrow 0 \leq \abs{\norm{z_{i_K}}_1 -\norm{a}_1 }  \leq \norm{ z_{i_k} -a } 
$$
$$
0 \leq \lim\limits_{k\to \infty} \abs{\norm{z_{i_K}}_1 -\norm{a}_1 }  \leq \lim\limits_{k\to \infty} \norm{ z_{i_k} -a } =0
$$
$$
\rightarrow   \lim\limits_{k\to \infty} \abs{\norm{z_{i_K}}_1 -\norm{a}_1 }    =0
$$
$$
\therefore   \lim\limits_{k\to \infty}  \norm{z_{i_K}}_1 =\norm{a}_1    \rightarrow a \neq 0  
$$
}

De \eqref{eq:norma_equival} y \eqref{eq:norma_equival-1}, como $\forall\, k \in \N, \quad \abs{z_{i_k}-a}\leq \beta\norm{z_{i_k}-a}_1 $

$$
\rightarrow  \lim\limits_{k\to \infty} \abs{z_{i_k}-a} = 0
$$
como $   \abs{\abs{z_{i_k} -\abs{a} }}\leq \abs{z_{i_k}-a}$
$$
\rightarrow  \lim\limits_{k\to \infty} \abs{z_{i_k}}=\abs{a}
$$
De \eqref{eq:equi-8}, $\abs{z_{i_k} }< \frac{1}{i_k}$
$$
\lim\limits_{k \to \infty} \abs{z_{i_k}} = 0 \rightarrow \abs{a}=0
$$
$\rightarrow a=0$ (Contradicción)

Por lo tanto, existe \( \alpha > 0 \) tal que:

\[
\alpha \|x\|_1 \leq \abs{x}, \quad \forall\, x \in \mathbb{R}^n.
\]
}

\cor{
	Sean $(x_k)_{k\in \N}$ y $(y_k)_{k\in \N}$ sucesiones en $\R^n$ tales que
	$$
	\lim_{k \to \infty} x_k = a \in \R^n \quad \text{y} \quad \lim_{k \to \infty} y_k = b \in \R^n.
	$$
	Sea $(\lambda_k)_{k\in \N} \subset \R$ una sucesión tal que $\lim_{k \to \infty} \lambda_k = \lambda_0$.
	
	Entonces, se cumple:
	\begin{description}
		\item[(i)] 
		\[
		\lim_{k \to \infty} (x_k + y_k) = \lim_{k \to \infty} x_k + \lim_{k \to \infty} y_k = a + b.
		\]
		
		\item[(ii)] 
		\[
		\lim_{k \to \infty} (\lambda_k x_k) = \lambda_0 \cdot \lim_{k \to \infty} x_k = \lambda_0 a.
		\]
		
		\item[(iii) Producto interno euclidiano] 
		\[
		\lim_{k \to \infty} \inner{x_k}{y_k} = \inner{\lim_{k \to \infty} x_k}{\lim_{k \to \infty} y_k} = \inner{a}{b}.
		\]
		
		\item[(iv) Para toda norma $\|\cdot\|$ en $\R^n$] 
		\[
		\lim_{k \to \infty} \|x_k\| = \|\lim_{k \to \infty} x_k\| = \|a\|.
		\]
	\end{description}
	Donde $a = (a_1, \ldots, a_n)$, $b = (b_1, \ldots, b_n)$, $x_k = (x_1^k, \ldots, x_n^k)$ y $y_k = (y_1^k, \ldots, y_n^k)$.
}
		
					
					
\pf{Prueba del ítem (i):\\
	Como $\lim\limits_{k \to \infty} x_k = a$, se tiene que, para cada $i \in \{1, \ldots, n\}$, $
	\lim\limits_{k \to \infty} x_i^k = a_i $
	
	Analogamente, $\lim\limits_{k \to \infty} y_i^k = b_i \quad \text{para todo } i \in \{1, \ldots, n\}$
	
	Por tanto, para cada $i \in \{1, \ldots, n\}$:
	\[
	\lim_{k \to \infty} (x_i^k + y_i^k) = \lim_{k \to \infty} x_i^k + \lim_{k \to \infty} y_i^k = a_i + b_i.
	\]
	 Por el teorema \eqref{thm:teorema3}, 
	\[
	\lim_{k \to \infty} (x_k + y_k) = a + b.
	\]
}




%\chapter{Clase 3}
\clasedate{07 de abril de 2025}
\section{Unicidad en sucesiones acotadas}

Dado $\varepsilon > 0$, $\exists\, k_0 \in \mathbb{N}$,
tal que $k \geq k_0 \Rightarrow \|x_k - a\| < \varepsilon$

\begin{figure}[H]
	\centering
	\includegraphics[width=0.45\linewidth]{img/class3_figure1.png}
\end{figure}
\thmr{}{teorema3-1}{


	Sea $(x_k)_{k\in \N} \subset \Rn$ una sucesión acotada. Entonces, $(x_k)_{k\in \N}$ es convergente si y sólo si toda subsucesión convergente de $(x_k)_{k\in \N}$ converge al mismo punto de $\Rn$.

}
\pf{Prueba del teorema \ref{thm:teorema3-1}

$(\Rightarrow)$  Supongamos que $\lim\limits_{k \to \infty} x_k = a \in \Rn$. Sea $(x_{i_k})_{k \in \N} \subset (x_k)_{k \in \N}$ una subsucesión.

\begin{equation}
	\forall \varepsilon > 0,\, \exists\, k_0 \in \N:\, \forall k \in \N,\; k \geq k_0 \Rightarrow \|x_k - a\| < \varepsilon    \label{eq:class3-1}
\end{equation}

Como \((i_k)_{k \in \N} \subset \N\) es estrictamente creciente \(\Rightarrow i_k \geq k,\, \forall k \in \N\).

Entonces, \(k \geq k_0 \Rightarrow i_k \geq k \geq k_0 \Rightarrow \|x_{i_k} - a\| < \varepsilon\) por \eqref{eq:class3-1}.

\(\therefore \lim\limits_{k \to \infty} x_{i_k} = a\).


($\Leftarrow$) Supongamos que toda subsucesión de $(x_k)_{k\in \N}$ que converge lo hace al mismo punto.

Sea $A = \qty{a \in \Rn, \exists \, (x_{i_k})_{k\in \N} \subset (x_k)_{k\in \N} : \lim\limits_{k \to \infty } x_{i_k}=a}$ (Conjunto de valores de adherencia de la sucesión $(x_k)_{k\in\N}$)

Como $(x_k)_{k\in \N}$ es acotada, del teorema de \textbf{Bolzano-Weierstrass}, $\exists (x_{i_k})_k \subset (x_k)_k$ tal que $\lim\limits_{k \to \infty } x_{i_k}=a \in \Rn$. Así $ a\in A, A \neq  \varnothing $

Queremos mostrar que $\lim\lim_{k \to \infty} x_k=a$

Por reducción al absurdo, supongamos que \underline{no ocurra} $\lim\limits_{k \to \infty} x_k=a$

$$
	\sim( \forall \varepsilon > 0,\, \exists\, k_0 \in \N:\, \forall k \in \N,\; k \geq k_0 \Rightarrow \|x_k - a\| < \varepsilon )
$$
$$
	\exists\, \varepsilon > 0,\, \forall\, k_0 \in \N:\, \exists\, k \in \N,\; k \geq k_0 \wedge \|x_k - a\| \geq \varepsilon
$$

Podemos construir $(x_{j_k})_{k\in\N} \subset (x_k)_{k\in \N}$ tal que
\begin{equation}
	\abs{x_{i_j}-a}\geq \varepsilon_0 \quad\forall k\in \N \label{eq:class3-eq2}
\end{equation}

Por el teorema de \textbf{Bolzano-Weierstrass}, \(\exists\, \pqty{x_{j_{p_k}}}_{k \in \N} \subset (x_{j_k})_{k \in \N}\) tal que
\[
	\lim\limits_{k \to \infty} x_{j_{p_k}} = b \in \Rn
\]
De \eqref{eq:class3-eq2} $\abs{ x_{j_{p_k}} -a} \geq \varepsilon_0, \quad \forall k \in \N$

Así, $\abs{b-a} = \lim\limits_{k \to \infty}\abs{ x_{j_{p_k}}-a} = \varepsilon_0 >0 \rightarrow b\neq a$
}

\section{Puntos de acumulación}

\begin{itemize}
	\item Sea \( X \subset \mathbb{R}^n \). Decimos que \( a \in \mathbb{R}^n \) es \underline{punto de acumulación} de \( X \), \( a \in X' \), si:
	      \[
		      \forall \varepsilon > 0, \quad \left( B(a, \varepsilon) \setminus \{a\} \right) \cap X \neq \emptyset
	      \]
	      \ex{
		      Si \( B(a, \varepsilon) = X \), entonces
		      \[
			      \forall \delta > 0, \quad \left( B(b, \delta) \setminus \{b\} \right) \cap X \neq \emptyset
		      \]
		      \vspace{-2em}
		      \begin{figure}[H]
			      \centering
			      \includegraphics[width=0.25\linewidth]{img/class3_figure2.png}
		      \end{figure}
		      \vspace{-3em}
		      \[
			      \Rightarrow b \in X'
		      \]
	      }

	      \ex{
		      Mostrar que si \( X = B(a, \varepsilon) = \{ z \in \mathbb{R}^n : |z - a| < \varepsilon \} \), entonces:
		      \[
			      X' = B[a, \varepsilon] = \{ z \in \mathbb{R}^n : |z - a| \leq \varepsilon \}
		      \]
	      }
	\item Sea \( a \in X \). Decimos que \( a \in X \) es un punto aislado de \( X \) si \( a \notin X' \). Esto es,
	      \[
		      \exists \varepsilon_0 >0, (B(a, \varepsilon_0) \setminus \{a\}) \cap X = \varnothing
	      \]

	      \vspace{-2em}
	      \begin{figure}[H]
		      \centering
		      \includegraphics[width=0.25\linewidth]{img/class3_figure3.png}
	      \end{figure}
	      \vspace{-1em}
\end{itemize}


\thmr{}{teorema3-1}{
	Sea $X\subset \Rn$ son equivalentes:
	\begin{description}
		\item[(i)] $a \in X'$
		\item[ (ii)] $\exists (x_k)_{k\in \N} \subset X\setminus\{a\}$, es decir $x_k \neq a \, \forall k\in \N$, tal que $\lim\limits_{k \to \infty} x_k =a$
	\end{description}
}
\pf{ Prueba del teorema \ref{thm:teorema3-1}\\

	\begin{description}
		\item[(ii) $\Rightarrow$ (i)]  Supongamos que $\exists (x_k)_{k \in \N} \subset X\setminus \{a\}$ tal que $\lim\limits_{k \to \infty} x_k =a$

		      Dado $\varepsilon>0, \, \exists \, k_0\in \N, \forall k\in \N$
		      \begin{align*}
			      k \geq k_0 & \rightarrow \abs{x_k-a}<\varepsilon                                                \\
			      k \geq k_0 & \rightarrow  x_k \in \pqty{B(a,\varepsilon)\setminus\{a\}} \cap X \neq \varnothing
		      \end{align*}
		      Por tanto $a \in X'$
		\item[(i) $\Rightarrow$ (ii)] Sea $a \in X'$
		      $$
			      \forall \varepsilon>0,  \pqty{B(a,\varepsilon)\setminus\{a\}} \cap X \neq \varnothing
		      $$
		      \begin{itemize}
			      \item para $\varepsilon= 1, \exists x_1 \in  \pqty{B(a,1)\setminus\{a\}} \cap X \neq \varnothing $

			            $$
				            \exists x_1 \in X \text{ tal que } 0<\abs{x_1-a}<1
			            $$
			            Por el \underline{Principio  Arquimediano}, $\exists i_2 \in \N$ tal que $\frac{1}{i_2} < \abs{x_1-a}$
			      \item para $\varepsilon= \frac{1}{i_2} , \exists x_{i_2} \in  \pqty{B\qty(a,\frac{1}{i_2})\setminus\{a\}} \cap X \neq \varnothing $

			            $$
				            \exists x_{i_2} \in X \text{ tal que } 0<\abs{x_{i_2}-a}<\frac{1}{i_2} \text{ con } i_2>1=i_1
			            $$
			            \textbf{\textcolor{red}{$\ldots$ Tarea-hacerlo inductivo  }}

		      \end{itemize}
		      De este modo hemos construido $(x_{i_k}) \subset X\setminus \{a\}$ tal que $\abs{x_{i_k}-a} \frac{1}{i_k} \, \forall k \in \N$ donde $(i_k)_{k\in \N} \subset \N$ estrictamente creciente $i_1<i_2<i_3<\ldots$.

		      Por lo tanto $\lim\limits_{k \to \infty} \abs{x_{i_k}-a} = 0 \longrightarrow \lim\limits_{k \to \infty} x_{i_k} =a$
	\end{description}
}

\section{Funciones continuas}

Sean $X \in \Rm$ y $f : X \rightarrow \Rn$,
\begin{itemize}
	\item  Decimos que $f$ es \underline{continua} en $a\in X$ si
	      \begin{align*}
		      \forall \varepsilon >0, \exists \delta >0, \forall x \in X, \abs{x-a}<\delta & \longrightarrow \abs{f(x)-f(a)}<\varepsilon       \\
		      \forall x \in X, x\in B(a, \delta)                                           & \longrightarrow f(x) \in B(f(a), \varepsilon)     \\
		      \text{si }\quad x \in X \cap B(a, \delta)                                    & \longrightarrow f(x) \in B(f(a), \varepsilon)     \\
		                                                                                   & \rightarrow x \in f^{-1}\qty(B(f(a),\varepsilon)) \\
		      f(X\cap B(a, \delta ))                                                       & \subset B(f(a), \varepsilon)                      \\
		      X\cap B(a, \delta )                                                          & \subset f^{-1}\qty(B(f(a), \varepsilon))
	      \end{align*}
	      $\delta$ depende de $\varepsilon$ y de $a \in X$, es decir $\delta=\delta(\varepsilon, a)$
	\item $f$ es discontinua (no continua) en $a\in X$ si
	      $$
		      \forall \varepsilon_0 >0, \forall \delta>0, \exists x \in X, \abs{x-a}<\varepsilon \wedge \abs{f(x)-f(a)} \geq \varepsilon_0
	      $$
	\item $f \colon X \to \Rn$ es continua, significa que $f$ es continua en todo punto de $X$.

	\item Sea \( f \colon X \to \mathbb{R}^n \) una función cualquiera. Si $a \in X$ y $a \in X'$, es decir,  si \( a \in X \) es un punto aislado de \( X \), entonces \( f \) es continua en \( a \).

	      \begin{align*}
		      \exists \delta_0>0 \text{ tal que } & \pqty{B(a, \delta_0) \setminus\{a\}}\cap X = \varnothing \\
		      \to                                 & B(a, \delta_0) \cap X =\{a\}
	      \end{align*}
	      $$
		      \text{Dado } \varepsilon>0, \exists \delta_0>0, \in A  \forall x \in X, \abs{x-a} < \delta_0 \rightarrow \abs{f(x)-f(a)}<\varepsilon
	      $$
	      Si $x=a,$ entonces, $ 0<\delta_0 \rightarrow \abs{f(a)-f(a)}=0 < \varepsilon$

	      Por lo tanto, se cumple la condición de continuidad en \( a \):
	      \[
		      \forall \varepsilon > 0,\ \exists \delta > 0\ \text{tal que}\ \forall x \in X,\ \abs{x - a} < \delta \Rightarrow \abs{f(x) - f(a)} < \varepsilon.
	      \]
	      \[
		      \therefore\ f \text{ es continua en } a\in X.
	      \]

	      \ex{
		      Sea $A \colon \Rm \to \Rn $ una transformación lineal, tenemos que A es continua, pues
		      \begin{align*}
			      \forall x = (x_1, \ldots, x_m) \in \Rm, \abs{Ax} & =\abs{A\qty(\sum_{i=1}^{m} x_i e_i)}                                                                       \\
			                                                       & = \abs{\sum_{i=1}^{m} x_i Ae_i}                                                                            \\
			                                                       & \leq \sum_{i=1}^{m} \abs{x_i} \underbrace{\abs{A e_i}}_{\leq c = \max\limits_{1\leq i \leq m} \abs{A e_i}} \\
			                                                       & \leq c \cdot  \sum_{i=1}^{m} \abs{x_i}                                                                     \\
			                                                       & = c \norm{x}_1
		      \end{align*}
		      $	\forall x \in \Rm, \abs{Ax} \leq c \norm{x}$\\
		      En particular, $\forall x, w \in \Rm, \abs{Ax-Aw}\leq c \norm{x-w}_1$, \underline{A es LIPSCHITZIANA}

	      }
	      \rmkb{
		      Sea \( A \colon \mathbb{R}^m \to \mathbb{R}^n \) una aplicación lineal. Decimos que \( A \) es \textbf{Lipschitz} (o que satisface una condición de Lipschitz) si existe una constante \( c > 0 \) tal que
		      \[
			      \forall x \in \mathbb{R}^m,\quad \norm{Ax} \leq c \norm{x}.
		      \]
		      En particular, dado que \( A \) es lineal, para cualesquiera \( x, w \in \mathbb{R}^m \), se cumple:
		      \[
			      \norm{Ax - Aw} = \norm{A(x - w)} \leq c \norm{x - w}.
		      \]
		      Esto significa que \( A \) es una función Lipschitz con constante de Lipschitz \( c \).
	      }

	\item $f: X \subset \Rm \to \Rn$, $f$ es continua en $X$. Decimos que \( f \) es \textbf{uniformemente continua} si:
	      $$
		      \forall \varepsilon > 0,\ \exists \delta > 0\ \text{tal que}\ \forall x, w \in X,\ \abs{x - w} < \delta \Rightarrow \abs{f(x) - f(w)} < \varepsilon
	      $$
	      En este caso, \(\delta\) depende únicamente de \(\varepsilon\), y no de los puntos \(x, w\).
\end{itemize}

\ex{Toda función $f: X \rightarrow \Rn$ lipschitziana es uniformemente continua.\\
	$$
		\exists c>0, \text{ tal que } \forall x, w \in X, \abs{f(x)-f(w)} \leq c \abs{x-w}
	$$
	Dado $\varepsilon >0$, existe $\delta = \frac{\varepsilon}{c}>0, \forall x, w \in X$
	$$
		\abs{x-w} <\delta = \frac{\varepsilon}{c} \rightarrow \abs{f(x)-f(w)} \leq c \abs{x-w} < \varepsilon
	$$
	$$
		\therefore f \text{ es uniformemente continua}
	$$
}

En particular las proyecciones canónicas
\begin{align*}
	\pi_i \colon \Rn          & \longrightarrow \R          \\
	x=(x_1, x_2, \ldots, x_n) & \longmapsto \pi_i (x) = x_i
\end{align*}
$\pi_i$ es una transformación lineal\\
$\therefore $ uniformemente continua.

\ex{Ejercicio\\
	Mostrar que si $\varphi \colon \Rm \times \Rn \to \Rp$ bilineal, entonces $\varphi$ es continua.
}

\thmr{}{teorema3-2}{
	Sean \( X \subset \mathbb{R}^m \), \( Y \subset \mathbb{R}^n \), y funciones \( f \colon X \to \mathbb{R}^n \), \( g \colon Y \to \mathbb{R}^p \) tales que \( f(X) \subset Y \), es decir, la composición \( g \circ f \colon X \to \mathbb{R}^p \) está bien definida.

	Si \( f \) es continua en \( x_0 \in X \) y \( g \) es continua en \( f(x_0) \in Y \), entonces \( g \circ f \) es continua en \( x_0 \).

	En particular, si \( f \) es continua en todo \( X \) y \( g \) es continua en todo \( Y \), entonces \( g \circ f \colon X \to \mathbb{R}^p \) es continua.
}

\pf{ Prueba del teorema \ref{thm:teorema3-2}\\
	Sea \( f \colon X \to Y \subset \Rn \) y \( g \colon Y \to \Rp \). Supongamos que \( f \) es continua en \( x_0 \in X \), y que \( g \) es continua en \( f(x_0) \in Y \).
	Queremos probar que \( g \circ f \colon X \to \Rp \) es continua en \( x_0 \).

	Como \( g \) es continua en \( f(x_0) \), dado \( \varepsilon > 0 \), existe \( n > 0 \) tal que
	\begin{equation} \label{eq:class3-eq3}
		\forall y \in Y, \quad \abs{y - f(x_0)} < n \ \Rightarrow \ \abs{g(y) - g(f(x_0))} < \varepsilon.
	\end{equation}

	Como \( f \) es continua en \( x_0 \), existe \( \delta > 0 \) tal que
	\begin{equation} \label{eq:class3-eq4}
		\forall x \in X, \quad \abs{x - x_0} < \delta \ \Rightarrow \ \abs{f(x) - f(x_0)} < n.
	\end{equation}

	Dado que \( f(x) \in f(X) \subset Y \), se cumple \( f(x) \in Y \), por lo que al aplicar \eqref{eq:class3-eq3} se obtiene:
	\[
		\forall x \in X, \quad \abs{x - x_0} < \delta \ \Rightarrow \ \abs{g(f(x)) - g(f(x_0))} < \varepsilon.
	\]

	Es decir,
	\[
		\abs{(g \circ f)(x) - (g \circ f)(x_0)} < \varepsilon.
	\]

	Por lo tanto, \( g \circ f \) es continua en \( x_0 \in X \).



	\rmkb{Sea $f(x) \in \Rn$ y
		\begin{align*}
			f\colon X \subset \Rm & \longrightarrow \Rn                                 \\
			x                     & \longmapsto f(x) = (f_1(x), f_2(x), \ldots, f_n(x))
		\end{align*}
		decimos que $f1, f2, \ldots, f_n \colon \R $ son las \textcolor{red}{funciones coordenadas de $f$.}\\
		Notación: $f=(f_1, f_2, \ldots, f_n)$\\
		Es claro que $f_i = \pi_i \circ f, \forall i \in \{1, 2, \ldots, n\}$
	}
}


%\chapter{Clase 4}
%\chapter{Clase 5}

\expf{
Sea \( f \colon X \subset \mathbb{R}^m \to \mathbb{R}^n \). La función \( f \) es \emph{uniformemente continua} si, y solo si \(\forall  (x_k)_{k\in \N} , (y_k)_{k\in \N} \subset X \), se cumple que
\[
	\lim_{k \to \infty} \lvert x_k - y_k \rvert = 0 \quad \Rightarrow \quad \lim_{k \to \infty} \lvert f(x_k) - f(y_k) \rvert = 0
\]
}{\\
($\Rightarrow$) \quad Como $f$ es uniformemente continua.\\
Dado $\varepsilon >0, \exists \delta>0, \forall x, y \in X$
\begin{equation}
	\abs{x-y} <\delta \longrightarrow \abs{f(x)-f(y)} <\varepsilon
	\label{eq:class5-1}
\end{equation}
Sean $(x_k)_k, (y_k)_k \subset X$ tales que $ \lim_{k \to \infty}  \abs{f(x_k) - f(y_k)  }= 0$

Para el $\delta>0 $ anterior, $\exists k_0 \in \N, \forall k \in \N$
$$
	k\geq k_0 \rightarrow \abs{x_k-y_k} <\delta
$$
De \eqref{eq:class5-1},
$$
	\rightarrow \abs{f(x_k)-f(y_k)} < \varepsilon
$$
$$
	\therefore \lim_{k \to \infty} \abs{f(x_k)-f(y_k)} = 0
$$

($\Leftarrow$) \quad  Supongamos que $f$ no es uniformemente continua
$$
	\exists \varepsilon_0 >0, \forall \delta>0, \exists x_{\delta}, y_{\delta} \in X,\qquad \abs{x_{\delta}-y_{\delta}} <\delta \wedge \abs{f(x_{\delta})-f(y_{\delta})} \geq \varepsilon_0
$$
$$
	\forall k \in \N, \exists x_k, y_k \in X, 0\geq \abs{x_k-y_k} < \frac{1}{k} \wedge \abs{f(x_{\delta})-f(y_{\delta})} \geq \varepsilon_0
$$
$$
	\rightarrow \lim_{k \to \infty} \abs{f(x_{\delta})-f(y_{\delta})}=0 \wedge  \forall k \in \N, \abs{f(x_{\delta})-f(y_{\delta})} \geq \varepsilon_0
$$
Por hipótesis, $\lim\limits_{k \to \infty} \abs{f(x_{\delta})-f(y_{\delta})}=0$, pero $\lim_{k \to \infty} \abs{f(x_{\delta})-f(y_{\delta})}\geq \varepsilon_0 $\\
$\rightarrow 0\geq \varepsilon_0 >0 $ (Contradicción).
}

\expf{
	Sea \(f\colon X \subset \mathbb{R}^m \to \mathbb{R}^n\) uniformemente continua, si $
		\forall\, (x_k)_{k\in\mathbb{N}} \subset X,$
	si $(x_k)_{k \in \N}$ es de Cauchy, entonces la sucesión
	$	\bigl(f(x_k)\bigr)_{k\in\mathbb{N}} \subset \mathbb{R}^n$
	también es de Cauchy.\\
	¿El recíproco se cumple?
}{
	No se cumple, por contraejemplo.\\
	Sea $g \colon \R \to \R$ continua.\\
	$f(x) = \frac{1}{x}. \quad x \in \R -\{0\}$\\
	$g(x) = x^2$\\
	$g$ es sucesión de Cauchy en sucesión de Cauchy, pero no es uniformemente continua puesto que,

	$$
		x_k=k, y_k=k+\frac{1}{k}, \quad \forall k \in \N
	$$
	$$
		\lim\limits_{k \to \infty} \abs{x_k-y_k} =0 \wedge\lim_{k \to \infty} \abs{f(x_{\delta})-f(y_{\delta})} = 2
	$$
}


\thmrpf{}{}{
	Sea $f \colon X \in \Rm, \; X \to \Rn$ una función, $a \in X'$ y $b \in \Rn$. Tenemos que
	$$
		\lim\limits_{ x \to a} f(x)  = b = f(a)\in \Rn \Leftrightarrow \forall (x_k)_{k } \subset X-\{a\}, \; \lim_{k \to \infty} x_k =a \rightarrow \lim_{k \to \infty} f(x_k) = b = f(a)
	$$
}{
	($\Rightarrow$)\\
	Agregar graficos\\
	Sea $(x_k)_{k \in \N } \subset X-\{a\}$ tal que $\lim\limits_{k \to \infty} =a$

	Como $\lim\limits_{k\to \infty} f(x)=b$, dado $\varepsilon>0, \exists \delta>0$
	\begin{equation}
		\forall x \in X, 0< \abs{x-a} <\delta \rightarrow\abs{f(x)-b} < \varepsilon
		\label{eq:class5-2}
	\end{equation}

	$$
		k\geq k_0 \rightarrow 0< \abs{x_k-a}<\delta \rightarrow \abs{f(x_k)-b}< \varepsilon
	$$
	$$
		\therefore \lim_{k \to \infty} f(x_k) = b
	$$
	($\Leftarrow$)\\
	Supongamos que $\forall (x_k)_{k \in \N } \subset X-\{a\}$ tal que $\lim\limits_{k \to \infty} =a$

	Por contradicción supongtamos que
	$$
		\exists \varepsilon_0 >0, \forall \delta >0, \exists x_{\delta} \in X, 0<\abs{x_{\delta}-a}<\delta \wedge \abs{f(x_{\delta})-b}\geq \varepsilon_0
	$$

	$\forall k \in \N$, considere  $\delta_k = \frac{1}{k}$, $\exists x_k \in X; 0 <\abs{x_k-a} < \frac{1}{k} \wedge \abs{f(x_k)-b} \geq \varepsilon_0$.

	Hemos construido $(x_k)_{k \in \N} \subset X-\{a\}$ tal que $\lim\limits_{k\to \infty} =0$

	Por hipótesis $\rightarrow \lim\limits_{k \to \infty} f(x_k) = b$, pero $\forall k \in \N,  \abs{f(x_k)-b} \geq \varepsilon_0>0$
	$$
		\equiv \lim\limits_{k \to \infty} \abs{f(x_k)-b} =0 \qquad \rightarrow 0 \geq \varepsilon_0>0 \;(\rightarrow \; \leftarrow)
	$$
}

\propp{
	Sean \( f \colon X \subset \mathbb{R}^m \to \mathbb{R}^n \), \( a \in X' \) (punto de acumulación de \( X \)) y \( b \in \mathbb{R}^n \) tales que
	\[
		\lim_{x \to a} f(x) = b.
	\]
	Sea \( g \colon Y \subset \mathbb{R}^n \to \mathbb{R}^p \) una función continua en \( b \in Y \), y supongamos que \( f(X) \subset Y \), lo cual garantiza que la composición \( g \circ f \colon X \to \mathbb{R}^p \) está bien definida.

	Entonces,
	\[
		\lim_{x \to a} g(f(x)) = g(b).
	\]

}{
	Como \( g \) es continua en \( b \in Y \), dado \( \varepsilon > 0 \), existe \( \eta > 0 \) tal que
	\begin{equation}
		\lvert y - b \rvert < \eta \quad \Rightarrow \quad \lvert g(y) - g(b) \rvert < \varepsilon, \quad \forall y \in Y.
		\label{eq:class5-3}
	\end{equation}

	Por otro lado, como \( \lim\limits_{x \to a} f(x) = b \), para el \( \eta > 0 \) anterior, existe \( \delta > 0 \) tal que
	\[
		0 < \lvert x - a \rvert < \delta \quad \Rightarrow \quad \lvert f(x) - b \rvert < \eta, \quad \text{con } x \in X.
	\]
	Además, como \( f(X) \subset Y \), se cumple \( f(x) \in Y \) para todo \( x \in X \), por lo que podemos aplicar \eqref{eq:class5-3} y obtener:
	\[
		\lvert g(f(x)) - g(b) \rvert < \varepsilon.
	\]

	Por lo tanto,
	\[
		\lim_{x \to a} g(f(x)) = g\left( \lim_{x \to a} f(x) \right) = g(b),
	\]
}



\thmrpf{}{}{
	Sea
	\begin{align*}
		f \colon X \subset \mathbb{R}^m & \longrightarrow \mathbb{R}^n                                                                                          \\
		x                               & \longmapsto f(x) = (f_1(x), f_2(x), \ldots, f_n(x)) \quad \text{y} \quad b = (b_1, b_2, \ldots, b_n) \in \mathbb{R}^n
	\end{align*}
	Tenemos que
	$$
		\lim_{x \to a} f(x) = b \quad \Leftrightarrow \quad \forall i \in \{1, 2, \ldots, n\}, \quad \lim_{x \to a} f_i(x) = b_i
	$$

}{  ($\Rightarrow$) Sea \( f = (f_1, f_2, \dots, f_n) \colon X \to \mathbb{R}^n \), con \( f_i : X \to \mathbb{R} \) para cada \( i \in \{1, 2, \dots, n\} \), y supongamos que \( \lim_{x \to a} f(x) = b \).
	Como cada
	\begin{align*}
		\pi_i \colon \mathbb{R}^n & \longrightarrow \mathbb{R} \\
		(x_1, \dots, x_n)         & \longmapsto x_i
	\end{align*}
	es una proyección, y las proyecciones son funciones continuas, podemos concluir que
	\[
		\lim_{x \to a} \pi_i(f(x)) = \pi_i(b), \quad \forall i \in \{1, \dots, n\}
	\]
	En particular, esto implica que
	\[
		f_i(x) \to b_i \quad \text{cuando} \quad x \to a
	\]

	\noindent($\Leftarrow$) Sea \( (x_k)_{k \in \mathbb{N}} \subset X \setminus \{a\} \) tal que \( \lim_{k \to \infty} x_k = a \).
	Como \( \lim_{x \to a} f_i(x) = b_i \) para cada \( i \in \{1, \dots, n\} \), se cumple que
	\[
		\lim_{k \to \infty} f_i(x_k) = b_i, \quad \forall i \in \{1, \dots, n\}
	\]
	Por lo tanto, tenemos que
	\[
		\lim_{k \to \infty} \left( f_1(x_k), \dots, f_n(x_k) \right) = (b_1, \dots, b_n),
	\]
	lo que implica que
	\[
		\lim_{k \to \infty} f(x_k) = b
	\]
}

\rmkb{
	Sea \( z_k = (z_1^k, \ldots, z_n^k) \in \mathbb{R}^n \) y \( a = (a_1, \ldots, a_n) \in \mathbb{R}^n \). Entonces,
	\[
		\lim_{k \to \infty} z_k = a \quad \Longleftrightarrow \quad \lim_{k \to \infty} z_i^k = a_i, \quad \forall i \in \{1, \ldots, n\}.
	\]
}


\propp{Sea

}{

}

\thmr{}{}{
	Sea
}

\section{Punto interior}

Sea \( X \subset \mathbb{R}^n \).
\begin{itemize}
	\item Se dice que \( a \in X \) es un **punto interior** de \( X \) si
	      \[
		      \exists\, \delta > 0 \text{ tal que } B(a, \delta) \subset X.
	      \]

	\item El interior  de \( X \) es el conjunto de todos sus puntos interiores de $X$. Se denota por:
	      \begin{align*}
		      \interior{X} & = \{a \in X \colon a \text{ es punto interior de } X \}                                 \\
		                   & = \{a \in X \colon \exists\, \delta_a > 0 \text{ tal que } B(a, \delta_a) \subset X \}.
	      \end{align*}
	      Además, se cumple que \( \interior{X} \subset X \).

	\item El conjunto \( X \subset \mathbb{R}^n \) es \textbf{abierto} si todos sus puntos son interiores de \( X \), es decir,
	      \[
		      X = \interior{X}.
	      \]
	      \pf{
		      Supongamos que \( X \) es abierto. Entonces, por definición, todo punto \( a \in X \) es un punto interior, es decir, existe \( \delta_a > 0 \) tal que \( B(a, \delta_a) \subset X \).

		      Por lo tanto,
		      \[
			      X = \bigcup_{a \in X} B(a, \delta_a),
		      \]
		      es decir, \( X \) se puede expresar como unión de bolas abiertas contenidas en \( X \).

		      Recíprocamente, si existe \( \delta_a > 0 \) para cada \( a \in X \) tal que \( B(a, \delta_a) \subset X \), entonces todo punto de \( X \) es interior, por lo tanto \( X \subset \interior{X} \). Como siempre se tiene \( \interior{X} \subset X \), se concluye que \( X = \interior{X} \), y por tanto \( X \) es abierto.
	      }

\end{itemize}


\expf{
	Sea \( \delta > 0 \) y \( a \in \mathbb{R}^n \). Definimos la bola abierta como
	\[
		B(a,\delta) =  \{ z \in \mathbb{R}^n \colon \abs{z-a} < \delta \}.
	\]
	Sea \( x \in B(a, \delta) \). Mostremos que \( x \) es un punto interior de \( B(a,\delta) \); es decir, existe \( r > 0 \) tal que \( B(x, r) \subset B(a, \delta) \).
}{
	\begin{figure}[H]
		\centering
		% Aquí deberías incluir un dibujo que represente las bolas \( B(a,\delta) \) y \( B(x, r) \subset B(a,\delta) \)
		\caption{Representación de \( B(x, r) \subset B(a, \delta) \)}\label{figsa}
	\end{figure}

	Como \( x \in B(a,\delta) \), se tiene \( \abs{x - a} < \delta \). Definimos
	\[
		r = \delta - \abs{x - a} > 0.
	\]
	Afirmamos que \( B(x, r) \subset B(a, \delta) \), lo que probará que \( x \) es un punto interior.

	Sea \( w \in B(x, r) \), es decir, \( \abs{w - x} < r \). Entonces,
	\[
		\abs{w - a} \leq \abs{w - x} + \abs{x - a} < r + \abs{x - a} = \delta.
	\]
	Por lo tanto, \( w \in B(a, \delta) \), y se concluye que \( B(x, r) \subset B(a, \delta) \).
}


\corp{
	Dado $X \in \Rn$, tenemos que $\interior{X}$ es abierto.
}{
	Sea $x \in \interior{X} \rightarrow \exists \delta>0, B(x,\delta) \in X$
}


\thmrpf{}{}{}{}
\ex{
	Sea
}


%\chapter{Clase 6}
\clasedate{16 de abril de 2025}


%\chapter{Clase 7}
\clasedate{21 de abril de 2025}
\section{Conjuntos compactos}

\[
	K \subset \mathbb{R}^n \text{ es compacto si y solo si } K \text{ es cerrado y acotado}
\]
\[
	\overline{K} = K
\]

\propp{
Sea \( X \subset \mathbb{R}^n \). Entonces, \( X \) es \textbf{sucesionalmente compacto} \(\Leftrightarrow\)
\begin{equation}
	X \text{ es compacto} \iff \text{Toda sucesión } (x_k)_{k \in \mathbb{N}} \subset X \text{ tiene una subsucesión convergente a un punto de } X.
	\label{eq:class7-1}
\end{equation}
}{
(\( \Rightarrow \))


Supongamos que \( X \) es compacto \( \Rightarrow X \) es cerrado y acotado.

Sea \( (x_k)_{k \in \mathbb{N}} \subset X \)

\[
	\Rightarrow (x_k)_{k \in \mathbb{N}} \text{ es acotada}
\]

Por el Teorema de Bolzano–Weierstrass:

\[
	\exists i : \mathbb{N} \to \mathbb{N} \text{ estrictamente creciente y } \exists w_0 \in \mathbb{R}^n \text{ tal que }
\]

\[
	\lim_{k \to \infty} x_{i(k)} = w_0 \in \mathbb{R}^n
\]

Como \( X \) es cerrado y \( (x_k) \subset X \)

\[
	\Rightarrow w_0 \in   \overline{X} = X
\]

(\( \Leftarrow \)) Supongamos \eqref{eq:class7-1}

Sea \( w \in \overline{X} \). Entonces existe una sucesión \( (x_k)_{k \in \mathbb{N}} \subset X \) tal que
\begin{equation}
	\lim_{k \to \infty} x_k = w \in \mathbb{R}^n.
	\label{eq:class7-2}
\end{equation}


Como $(x_k)_{k \in \N} \subset X,$ por \eqref{eq:class7-1}.

$\exists j \colon \N \rightarrow \N$ estricatamente creciente y $\exists z \in X$ tal que $\lim_{k \to \infty}  x_{j(k)} = z \in X$

De \eqref{eq:class7-2}, $\lim_{k \to \infty} x_{j(k)} = w$

Por la unicidad del límite $w = z \in X$

$$
	\therefore \overline{X}  \subset X
$$
Así, $X$ es cerrado

Mostremos que $X$ es acotado $\rightarrow \exists c>0, \forall z \in X, \abs{z} \geq c$

Supongamos que \( X \) no es acotado. Entonces:
\begin{equation}
	\forall k \in \mathbb{N}, \exists z_k \in X \text{ tal que } \|z_k\| > k.
	\label{eq:class7-3}
\end{equation}


Hemos construido $(z_k)_{k \in \N} \subset X$ tal que $$
	\lim_{k \to \infty} \abs{z_k} = +\infty$$

Dado  $i \colon \N \to \N$ estrictamente creciente. $i(k) \leq k, \forall k \in \N$

$$
	\forall k \in \N, \quad \abs{Z_{i(k)}} > i(k)
$$
$$
	\lim_{k \to \infty} \abs{Z_{i(k)}} = +\infty
$$
Así $(z_{i(k)})_{k \in \N} $ no converge, esto contradice a \eqref{eq:class7-1}.

}

\rmkb{En espacios métricos, el hecho de ser cerrado y acotado \textbf{no implica} ser sucesionalmente compacto. Esta propiedad es equivalente a la compacidad solo en \(\mathbb{R}^n\).}

\propp{
Sea \( \{K_p\}_{p \in \N} \) una colección de subconjuntos compactos de \( \R^n \) tal que:
\[
	K_p \supset K_{p+1}, \quad \forall p \in \N, \qquad K_p \neq \varnothing
\]
Entonces:
\[
	K := \bigcap\limits_{p \in \N} K_p \text{ es compacto y } K \neq \varnothing
\]
}{
\begin{itemize}
	\item \( \bigcap\limits_{p \in \N} K_p \subset K_1 \Rightarrow \bigcap\limits_{p \in \N} K_p \) es acotado.
	\item \( \bigcap\limits_{p \in \N} K_p \) es cerrado, pues cada \( K_p \) es cerrado.
\end{itemize}
\[
	\Rightarrow \bigcap\limits_{p \in \N} K_p \text{ es compacto}
\]

\textbf{Mostremos que \( K \neq \varnothing \colon \)}

Como \( K_p \neq \varnothing, \ \forall p \in \N \), fijamos \( x_p \in K_p \subset K_1 \Rightarrow (x_p)_{p \in \N} \subset K_1 \)

\[
	\Rightarrow (x_p)_{p \in \N} \text{ es acotada}
	\Rightarrow \exists i \colon \N \to \N \text{ estrictamente creciente tal que }
	\lim\limits_{p \to \infty} x_{i(p)} = z_0 \in K_1
\]

Tomemos \( (x_{i(p)})_{p \geq 2} \). Como \( x_{i(p)} \in K_{i(p)} \subset K_2 \) y \( i(p) \leq i(2) \leq 2 \), para todo \( p \leq 2 \), entonces:

\[
	\lim_{\substack{p \to \infty \\ p \leq 2}} x_{i(p)} = z_0, \quad x_{i(p)} \in K_2
\]

Fijemos \( k \in \N \). Como \( x_{i(p)} \in K_{i(p)} \subset K_k \) para \( p \leq k \), se tiene:

\[
	\lim_{\substack{p \to \infty \\ p \leq k}} x_{i(p)} = z_0 \in \overline{K_k} = K_k
\]

\[
	\therefore \forall k \in \N, \quad z_0 \in K_k
	\Rightarrow z_0 \in \bigcap\limits_{k \in \N} K_k = K
	\Rightarrow K \neq \varnothing
\]
}

\section{Cubrimiento}

Sea \( \{C_{\lambda}\}_{\lambda \in L} \) una familia de subconjuntos de \( \mathbb{R}^n \).

Decimos que \( \{C_{\lambda}\}_{\lambda \in L} \) es un \textbf{cubrimiento} de un conjunto \( X \subset \mathbb{R}^n \) si:
\[
	X \subset \bigcup\limits_{\lambda \in L} C_{\lambda}
\]

Un \textbf{subcubrimiento finito} de \( \{C_{\lambda}\}_{\lambda \in L} \) es una subfamilia \( \{C_{\lambda}\}_{\lambda \in L'} \), donde \( L' \subset L \) es un conjunto finito, tal que:
\[
	X \subset \bigcup\limits_{\lambda \in L'} C_{\lambda}
\]

\thmrpf{}{}{
Sea \( K \subset \mathbb{R}^n \) un conjunto compacto.

Entonces, todo cubrimiento abierto de \( K \) admite un subcubrimiento finito.
}{
Sea \( \{A_{\lambda}\}_{\lambda \in L} \) un cubrimiento abierto de \( K \), es decir:
\[
	K \subset \bigcup\limits_{\lambda \in L} A_{\lambda}, \quad A_{\lambda} \subset \mathbb{R}^n \text{ abierto } \forall \lambda \in L.
\]

i.e   \( \forall \lambda \in L, \{A_{\lambda}\}_{\lambda \in L} \) ees abierto de $\Rn$, $$K \subset \bigcuplim{\lambda \in L} A_{\lambda}$$

\lem{Lindelöf}{
Sea \( X \subset \mathbb{R}^n \) y sea \( \{C_{\lambda}\}_{\lambda \in \Lambda} \) un cubrimiento abierto de \( X \), es decir,
\[
	X \subset \bigcup\limits_{\lambda \in \Lambda} C_{\lambda}, \quad C_{\lambda} \text{ abierto } \forall \lambda \in \Lambda.
\]
Entonces, existe un subconjunto numerable \( \Lambda_0 \subset \Lambda \) tal que
\[
	X \subset \bigcup\limits_{\lambda \in \Lambda_0} C_{\lambda}.
\]
}

Por (Lindelöf), $\exists \Lambda_0 = \{\lambda_i \in \Lambda \colon i \in \N\} \subset\Lambda$, $\Lambda$ es numerable.

Tal que
\begin{equation}
	K \subset \bigcuplim{\lambda \in \Lambda_0} A_{\lambda} = \bigcuplim{i \in \N} A_{\lambda_{i}}
	\label{eq:class7-4}
\end{equation}
Sea
\begin{align*}
	F_0    & = K \supset F_1 \supset F_2 \supset \ldots                                                                                       \\
	F_1    & = K-A_{\lambda_1} = K \cap \bqty{A_{\lambda_1}^{C}}                                                                              \\
	F_2    & = K-\bqty{A_{\lambda_1} \cup A_{\lambda_2}} = K \cap \bqty{A_{\lambda_1} \cup A_{\lambda_2}}^{C}                                 \\
	\vdots & =\qquad \vdots\hspace{3cm} \vdots                                                                                                \\
	F_m    & =  K-\bqty{ \bigcuplim{i = 1 }{m} A_{\lambda_i} } = K \cap \bqty{ \bigcuplim{i = 1 }{m} A_{\lambda_i} }^{C} \quad \text{Cerrado}
\end{align*}
Tenemos
\[
	F_0 \supset F_1 \supset F_2 \supset F_3 \supset \ldots \supset F_m \supset \ldots; \qquad F_0 \text{ acotado } \Rightarrow F_m \text{ es acotado } \forall m \in \mathbb{N}
\]
donde cada \( F_m \) es cerrado como intersección de cerrados, y acotado, luego \textbf{compacto}  en \( \mathbb{R}^n \).

Supongamos \( F_m \neq \varnothing \) para todo \( m \in \mathbb{N} \), y que \( F_m \supset F_{m+1} \), entonces, por el teorema del conjunto descendente de compactos no vacíos:
\[
	\bigcap_{m \in \mathbb{N}} F_m \neq \varnothing
\]
Pero:
\begin{align*}
	\bigcap_{m \in \mathbb{N}} F_m
	 & = \bigcap_{m \in \mathbb{N}} \left( K \cap \left( \bigcup_{i = 1}^{m} A_{\lambda_i} \right)^C \right) \\
	 & = K \cap \bigcap_{m \in \mathbb{N}} \left( \bigcup_{i = 1}^{m} A_{\lambda_i} \right)^C                \\
	 & = K \cap \left( \bigcup_{m \in \mathbb{N}} \bigcup_{i = 1}^{m} A_{\lambda_i} \right)^C                \\
	 & = K \cap \left( \bigcup_{i \in \mathbb{N}} A_{\lambda_i} \right)^C                                    \\
	 & = K - \bigcup_{i \in \mathbb{N}} A_{\lambda_i}                                                        \\
	 & = K - K \quad \text{(por \eqref{eq:class7-4})}                                                        \\
	 & = \varnothing
\end{align*}
\[
	\Rightarrow \bigcap_{m \in \mathbb{N}} F_m = \varnothing \quad \text{contradicción}
\]

Por lo tanto, existe \( m_0 \in \mathbb{N} \) tal que \( F_{m_0} = \varnothing \), es decir:
\[
	K - \bigcup_{i=1}^{m_0} A_{\lambda_i} = \varnothing \Rightarrow K \subset \bigcup_{i=1}^{m_0} A_{\lambda_i}
\]

\[
	\therefore \text{Existe un subcubrimiento finito de } \{A_\lambda\}_{\lambda \in L} \text{ que cubre } K.
\]
}

\thmrpf{}{}{
	\begin{align}
		\text{Si todo cubrimiento abierto de } \mathbb{R}^n \text{ admite algún subcubrimiento finito,} \label{eq:class7-5} \\
		\text{entonces } K \text{ es compacto.} \nonumber
	\end{align}
}{
	Sea $K \subset \Rn$ tales que verifica \eqref{eq:class7-5}.
	\begin{itemize}
		\item Mostremos que $K$ es acotado.
		      $$
			      K \subset \bigcuplim{a \in K} B(a,1), \quad \text{abierto en } \Rn
		      $$
		      Entonces, gracias a \eqref{eq:class7-5} existen $a_1, a_2, \ldots, a_n \in K$\\
		      tales que
		      $$
			      K \subset \bigcuplim{i =1}{m} B(a_i, 1)
		      $$
		      donde $z \in K \rightarrow \exists i \in \{1, \ldots, m\}$ tal que $z \in B(a_i, 1)$\\
		      Sea $L = \max\limits_{1 \leq i \leq m} \{\abs{a_1}\}+1$, además $\abs{z-a_i} <1$
		      $$
			      \abs{z } \leq \abs{z-a_i} +\abs{a_i} < L
		      $$
		      $$
			      \therefore K \text{ es acotado}
		      $$
		\item Mostremos que $K$ es cerrado, $\overline{K} \subset K$.

		      Suponga que $K$ no es acotado ( $\overline{K} \not\subset K $)
		      \begin{equation}
			      \Rightarrow \exists w_0 \in \overline{K}, \quad w_0 \not\in K
			      \label{eq:class7-6}
		      \end{equation}
		      $$
			      \rightarrow\{w_0\} \cap K = \varnothing
		      $$
		      $$
			      \rightarrow K \subset \Rn-\{w_0\}
		      $$
		      \rmkb{
			      $$
				      \{w_0\} = \bigcaplim{p \in  \N} B\qty[w_0, \frac{1}{p}]
			      $$
		      }
		      $$
			      \Rightarrow K \subset\Rn-\{w_0\} = \bigcuplim{p \in \N}\bqty{\Rn-B\bqty{w_0, \frac{1}{p}}}
		      $$
		      donde $\Rn-B\bqty{w_0, \frac{1}{p}}  $ es abierto.
		      $$
			      \Rightarrow K \subset  \bigcuplim{p \in \N}\bqty{\Rn-B\bqty{w_0, \frac{1}{p}}}
		      $$
		      Por \eqref{eq:class7-5}\; $\exists p_1, p_2, \ldots, p_l \in \N$, además podemos suponer $p_1 < p_2< \ldots < p_l$
		      $$
			      K \subset  \bigcuplim{p \in \N}{l}\bqty{\Rn-B\bqty{w_0, \frac{1}{p_l}}} = \Rn-B\bqty{w_0, \frac{1}{p_l}}
		      $$
		      $$
			      \Rightarrow K \cap B\bqty{w_0, \frac{1}{p_l}} = \varnothing \qquad (\rightarrow \leftarrow)
		      $$
		      Esto contradice \eqref{eq:class7-6}, pues $w_0 \in \overline{K}$.

	\end{itemize}
}

\thmrpf{Lindelöf}{}{
Sea $X \subset \Rn$. Todo cubrimiento abierto del conjunto $X$ admite algún subcubrimiento numerable.
}{
Por lo visto antes, existe $E \subset X$ numerable y dentro de $X$, $X \subset \overline{E}$.

\noindent Sea $E = \{x_i\}_{i \in \N}$\\
Sea $\{A_{\lambda}\}_{\lambda \in \Lambda} $ un cubrimiento abierto de $X \rightarrow X \subset \bigcuplim{\lambda \in \Lambda} A_{\lambda}$\\
Sea $\mathcal{F} = \qty{ B\pqty{x_i, \frac{1}{i}}}_{(i, j) \in \N \times \N \text{(numerable)}} $
\begin{align*}
	\exists \varphi \colon \N & \longrightarrow \N \times \N \text{ biyección} \\
	k                         & \longmapsto \varphi(k) = (i_k, i_k)
\end{align*}
Sea $\mathcal{F}$ una función numerable de bolas abiertas
$$
	\mathcal{F}  = \{B_k\}_{k \in \N} \quad\text{tal que}\quad B_k = B\pqty{x_{i_k}, \frac{1}{i_{k}}}
$$
$$
	\widehat{\N} = \{k \in \N \colon B_k \subset A_{\lambda}\; \text{para algún}\;  \lambda \in \Lambda\}
$$

\clmp{}{
Sea $B_k \subset A_{\lambda_k}$
$$
	X \subset \bigcuplim{k \in \widehat{\N}} B_k\subset \bigcuplim{k \in \widehat{\N}} A_{\lambda_k}
$$
}{
Sea $x \in X$. Como,
$$
	X \subset \bigcuplim{k \in \widehat{\N}} A_{\lambda_k} \rightarrow \exists \lambda_0 \in \Lambda \text{ tal que } x \in A_{\lambda_0}
$$
donde $A_{\lambda_0} $ es abierto.

Así $\exists \varepsilon >0$ tal que $B(x, \varepsilon) \subset A_{\lambda_0} $
$$
	x \in X \subset \overline{E}
$$

Por el principio Arquimediano $$\exists j_0 \in \N \quad \text{tal que}\quad \frac{1}{j_0} < \frac{\varepsilon}{2} \quad \pqty{\equiv \frac{2}{j}  < \varepsilon}
$$
Como $x  \in X \subset \overline{E} \rightarrow B\pqty{x, \frac{1}{j_0}} \cap E \neq \varnothing$
$$
	\abs{x_{i_0}-x} < \frac{1}{j_0}
$$
$x \in B\pqty{x_{i_0}, \frac{1}{j_0}} = B_{k_0}$ tal que $\exists k_0 \in \N, \; \phi(k_0) = $

Así, $x \in B_{k_0} \subset B(x, \varepsilon) \subset A_{\lambda_0}$. Así $k_0 \in \widehat{\N}$.\\
Sí $z \in B_{k_0} = B\pqty{x_{i_0}, \frac{1}{j_0}}$
$$
	\abs{z-x_{i_0}} < \frac{1}{j_0}
$$
$$
	\rightarrow \abs{z-x} \leq \abs{z-x_{i_0}} + \abs{x_{i_0} - x} <\frac{1}{j_0} + \frac{1}{j_0} = \frac{2}{j_0} < \varepsilon
$$
$$
	\rightarrow z \in B(x, \varepsilon)
$$
%Theorem~\ref{thm:mybigthm}
}
$\forall x \in X, \; \exists k \in \N $  tal que $x \in B_k \subset A_{\lambda_k}$ para algún $\lambda_k \in \Lambda$\\
$\exists k \in \widehat{\N}$ tal que $x \in B_k$
$$
	\therefore X \subset \bigcup{k \in \widehat{\N}} B_k
$$
}

\thmrpf{}{}{
Sea $f \colon X \subset \Rn \to \Rn$ una función continua.\\
Si $K \subset X$ es compacto, entonces $f(k)$ es compacto.
}{
Sea $K \subset X$ compacto.\\
Mostremos que $f(K)$ es compacto.\\
Sea $\{A_{\lambda}\}_{\lambda \in \Lambda}$ un cubrimiento abierto de $f(K)$,\\
$\rightarrow A_{\lambda}$ es abierto $\forall \lambda \in \Lambda$ y $f(K) \subset \bigcuplim{\lambda \in \Lambda} A_{\lambda}$
$$
	K \subset f^{-1}\pqty{f(k)} \subset f^{-1} \pqty{\bigcuplim{\lambda \in \Lambda} A_{\lambda}}
$$
$$
	K \subset \bigcuplim{\lambda \in \Lambda} f^{-1} \pqty{  A_{\lambda}}
$$
$f^{-1} (A_{\lambda}) $ es abierto $\forall \lambda \in \Lambda$.

Como $k$ es compacto, $\rightarrow \exists \lambda_1, \lambda_2, \ldots, \lambda_m \in \Lambda$ tales que

\begin{equation}
	K \subset \bigcuplim{i = 1}{m} f^{-1}\pqty{A_{\lambda_{i}}}
	\label{eq:class7-7}
\end{equation}
\clm{Afirmación}{
	$$
		f(K) \subset \bigcuplim{i = 1}{m} A_{\lambda_{i}}
	$$}
Sea $z \in f(K) \rightarrow z = f(w)$ para algún $w \in K$

De \eqref{eq:class7-7},  $\exists i \in \{1, \ldots, m\} $ tal que
$$
	w \in f^{-1} \pqty{A_{\lambda_{i}}}
$$
$$
	z = f(w) \in A_{\lambda_{i}}  \subset \bigcuplim{i = 1}{m} A_{\lambda_{i}}
$$
$$
	\therefore K \text{ is compact}
$$
}

\propp{(Teorema de Weierstrass)\\
	Sea \( f \colon K \subset \mathbb{R}^m \to \mathbb{R}^n \) una función continua definida sobre un conjunto \( K \) compacto, es decir, cerrado y acotado, con \( K \neq \varnothing \).\\
	Entonces existen puntos \( w_1, w_2 \in K \) tales que:
	\[
		f(w_1) \leq f(w) \leq f(w_2), \quad \forall w \in K,
	\]
	donde \( f(w_1) \) es el valor mínimo y \( f(w_2) \) el valor máximo que toma la función \( f \) en \( K \).
}{
	Como \( f \) es continua y \( K \) es compacto, entonces la imagen \( f(K) \subset \mathbb{R} \) es también un conjunto compacto y no vacío.\\

	Por ser compacto en \( \mathbb{R} \), \( f(K) \) está acotado y alcanza sus extremos. Es decir, existen:
	\[
		m_0 = \inf f(K), \quad M_0 = \sup f(K),
	\]
	y se cumple \( m_0, M_0 \in f(K) \) porque \( f(K) \) es cerrado.\\

	$\forall m \in \N$, $\exists y_m \in f(K)$ tal que
	$$
		M_0 -\frac{1}{m} < y_m \leq M_0
	$$
	$y_m = f(x_m)$ para algún $x_m \in K$. $(x_m)_m \subset K$ tal que $\forall m \in \N$,
	$$
		M_0 -\frac{1}{m} < f(x_m) \leq M_0
	$$
	Como $K$ es compacto, $\exists i \colon \N \to \N$ estrictamente creciente tal que,
	$$
		\lim\limits_{m \to \infty} x_{i(m)} = w_2 \in K
	$$
	Como $f$ es continua en $w2$,
	$$
		\lim\limits_{m \to \infty} f(x_{i(m)}) = f(w_2)
	$$


	Como \( f(x_{i(m)}) \to M_0 \) y también \( f(x_{i(m)}) \to f(w_2) \), por unicidad del límite se concluye que:
	\[
		f(w_2) = M_0.
	\]

	Por tanto, \( f \) alcanza su valor máximo en el punto \( w_2 \in K \).\\
	De manera análoga, se puede construir una sucesión que converge a un punto \( w_1 \in K \) tal que \( f(w_1) = m_0 \), el valor mínimo.

	\[
		\therefore \forall w \in K, \quad f(w_1) \leq f(w) \leq f(w_2).
	\]
}

%\chapter{Clase 8}
\clasedate{23 de abril de 2025}

\corrp{}{class8-coro1}{
	Sea \( K \subset \mathbb{R}^m \) compacto y \( f \colon K \to \mathbb{R}^n \) continua.\\
	Entonces, para todo subconjunto cerrado \( F \subset K \), la imagen \( f(F) \) es cerrada en \( \mathbb{R}^n \).
}{
	Sea \( F \subset K \) un conjunto cerrado. Como \( K \) es compacto, entonces \( F \), al ser un subconjunto cerrado de un compacto, también es compacto.
	
	Dado que \( f \) es continua y \( F \) es compacto, la imagen \( f(F) \) es compacta en \( \mathbb{R}^n \).
	
	Pero en \( \mathbb{R}^n \), todo conjunto compacto es cerrado. Por tanto, \( f(F) \) es cerrado.
} 

\corrp{}{}{
Sea \( f \colon K \subset \mathbb{R}^m \to L \subset \mathbb{R}^n \) una función continua y biyectiva, con \( K \) compacto y \( L = f(K) \).\\
Entonces \( f \) es un homeomorfismo entre \( K \) y \( L \).
}{
Como $f$ es biyección $\exists \colon f^{-1} \colon L \rightarrow K$.

Sea $F \subset K$ cerrado $\rightarrow (f^{-1})^{-1} (F) = f(F)$ es cerrado en $L$ por Corollary~\ref{cor:class8-coro1}. 

$$
\therefore f^{-1} \text{ es continua}
$$

Por tanto, \( f \) es un homeomorfismo entre \( K \) y \( L \).
}

\corp{}{
Sea $\varphi \colon K \subset \Rm \rightarrow L \subset \Rn$ continua, donde $K$ es compacto  y además $\varphi(K) = L$\\
Sea $f \colon L \rightarrow \Rp$. Tenemos que $f$ es continua si, y solo sí $f \circ \varphi \colon K \rightarrow \Rp$ continua.
}{
($\Rightarrow$)\quad Composición de continuas es continua.

\noindent ($\Leftarrow$)\quad 	Supongamos que $f \circ \varphi \colon K\rightarrow \Rp$ es continua.\\

Sea $F \subset \Rp$ cerrado. Mostremos que $f^{-1}(F)$ es cerrado en $L$.

Como $f\circ \varphi \colon K \subset \Rm \rightarrow \Rp$ es continua.
$$
\rightarrow (f \circ \varphi)^{-1} (F) \text{ es cerrado en } K
$$

Del corolario~\ref{cor:class8-coro1},  $\varphi$ lleva cerrados en cerrados.
\begin{align*}
	f^{-1}(F) &= \varphi\qty((f \circ\varphi)^{-1}(F)) \text{ es cerrado}\\
	&= \varphi\pqty{\varphi^{-1 (f^{-1}(F))}} \text{  Ejercicio}
\end{align*}
Como $\varphi \colon K \rightarrow L$ es sobreyectiva y $R \subset L$.
$$
\varphi\pqty{\varphi^{-1}(R)}=R
$$
}

\thmrpf{}{}{
Sea $f \colon K \subset \Rm \rightarrow\Rn$ continua definida en $K \subset \Rm$ compacto. Entonces $f$ es uniformemente continua.
}{

Supongamos que $f$ no es uniformemente continua,
$$
\sim \bqty{ \forall \varepsilon>0, \exists \delta>0, \forall x, y \in K, \abs{x-y}< \delta \rightarrow \abs{f(x)-f(y)}< \varepsilon }
$$
$$
\exists \varepsilon_0 >0, \forall \delta>0, \exists x_{\delta}, y_{\delta} \in K, \abs{x_{\delta}-y_{\delta}}<\delta \wedge \abs{f(x_{\delta}) -f(y_{\delta})} \geq \varepsilon
$$

$\forall k \in \N$, sea $\delta_k = \frac{1}{k}$
$$
\forall k \in \N, \exists x_k, y_k \in K, \abs{x_k-y_k}< \frac{1}{k} \wedge \abs{f(x_k)-f(y_k)} \geq \varepsilon_0
$$

Así $(x_k)_k \subset K$, $(y_k)_k \subset K$

$\exists j \colon \N \rightarrow \N$ estrictamente creciente tal que $\lim\limits_{x \to \infty} x_{j(k)} =x_0 \in K$

De (*), $\lim\limits_{ k \to \infty} y_{j(k)} =x_0 \in K$

Como $f$ es continua

$$
f(x_p) = \lim\limits_{k \to \infty} f(x_{j(k)}) = \lim\limits_{k \to \infty} f(y_{j(k)})
$$
esto contradice a $\abs{expression}$


}



\thmrpf{}{}{
	
}{
	
}



\thmrpf{}{}{

}{

}

\section{Distancia entre conjuntos}
Sean $X, Y \subset  \Rn $ no vacíos, la distancia entre $X \text{ e } Y$ está acotado inferiormente,
\begin{equation}
	d(X,Y)= \inf \qty{\abs{x-y } \in \R : x \in X, y \in Y}
\end{equation}

\rmkb{
	Si $X \subset  M \text{ e } Y \subset M$, entonces
	$$
	d(X,Y) \geq d(M,N)
	$$
	\pf{
	\begin{align*}
		\alpha\pqty{X, Y} & = \qty{ \abs{x-y} \colon x \in X, y \in Y}\\
		\alpha(M, N) & = \qty{ \abs{x-y} \colon x \in M, y \in N}
	\end{align*}
	Se tiene que $\alpha(X, Y ) \subset \alpha(M,N)$ 
	\begin{align*} 
		\rightarrow \inf(\alpha\pqty{X, Y}) &\geq \inf\qty(\alpha\pqty{M, N})\\
	  d\pqty{X, Y} &\geq d\pqty{M, N}
	\end{align*}
	}	
}

\thmrpf{}{}{
	Sean \( X, Y \subset \mathbb{R}^n \) conjuntos no vacíos. Entonces:
\[
d(X, Y) = d(\overline{X}, \overline{Y}),
\]
donde \( d(A, B) := \inf\{ \|a - b\| : a \in A, b \in B \} \) es la distancia entre conjuntos.
}{
\textbf{Primero:} Como \( X \subset \overline{X} \) y \( Y \subset \overline{Y} \), se tiene:
\[
d(\overline{X}, \overline{Y}) \leq d(X, Y).
\]



Mostremos que 
$$
d\pqty{\overline{X}, \overline{Y}} \geq d\pqty{X, Y} .
$$
Supongamos que $d\pqty{\overline{X}, \overline{Y}} < d\pqty{X, Y}  $,  además $d\pqty{X, Y}  $ no es cota inferior de $ \alpha\pqty{\overline{X}, \overline{Y}}$\\
Si $p \in \overline{X}, q \in \overline{Y}$
$$
\inf\{p-q\colon p \in \overline{X}, q \in \overline{Y}\} \quad \text{ mayor cota superior}
$$
$d(X,Y) \rightarrow \exists (y_k)_k \subset Y \text{ tal que } y_k \rightarrow y_0$\\
$\exists x_0 \in \overline{X}, \exists y_0 \in \overline{Y}$ tales que $\abs{x_0-y_0} < d(X,Y)$\\
$\exists (x_k)_k \subset X$ tal que $x_k \rightarrow x_0$,  $\exists (y_k)_k \subset Y$ tal que $y_k \rightarrow y_0$
$$
\abs{\abs{z}-\abs{w}} \geq \abs{z-w}, \varphi(z) = \abs{z}
$$ 
\begin{align*}
	\lim\limits_{k \to \infty} \abs{x_k-y_k} & = \lim\limits_{k \to \infty} \varphi \pqty{{x_k-y_k}} \\
	 & = \varphi\pqty{ \lim\limits_{k \to \infty} \pqty{x_k-y_k}}  \\
	  & =\varphi(x_0-y_0)\\
	  &=   \abs{x_0-y_0}< d(X,Y)
\end{align*}
$\exists k_0 \in \N, x_{k_0} \subset X, y_{k_0} \subset Y$ tal que,
 
\begin{align*}
	\abs{x_{k_0} - y_{k_0}} &< d(X, Y)\\
	&< \inf\{\abs{x-y} \colon x \in X \wedge y \in Y\} \qquad (\rightarrow \leftarrow)
\end{align*}
Entonces $d\pqty{\overline{X}, \overline{Y}} \geq d\pqty{X, Y} $

%------------------------------


\vspace{1ex}
\textbf{Veamos la desigualdad opuesta:} Supongamos, en busca de contradicción, que:
\[
d(\overline{X}, \overline{Y}) < d(X, Y).
\]
Entonces existen puntos \( x_0 \in \overline{X} \), \( y_0 \in \overline{Y} \) tales que:
\[
\|x_0 - y_0\| < d(X, Y).
\]

Como \( x_0 \in \overline{X} \), existe una sucesión \( (x_k) \subset X \) tal que \( x_k \to x_0 \).\\
Análogamente, existe una sucesión \( (y_k) \subset Y \) tal que \( y_k \to y_0 \).

Entonces:
\[
\lim_{k \to \infty} \|x_k - y_k\| = \|x_0 - y_0\| < d(X, Y).
\]

Por lo tanto, para algún \( k_0 \in \mathbb{N} \), se cumple:
\[
\|x_{k_0} - y_{k_0}\| < d(X, Y),
\]
con \( x_{k_0} \in X \), \( y_{k_0} \in Y \), lo cual contradice la definición de \( d(X, Y) \) como el ínfimo de todas las distancias entre puntos de \( X \) y \( Y \).

\vspace{1ex}
\textbf{Conclusión:} La suposición lleva a contradicción, por lo tanto:
\[
d(\overline{X}, \overline{Y}) \geq d(X, Y),
\]
y combinando con la desigualdad anterior, se concluye:
\[
d(X, Y) = d(\overline{X}, \overline{Y}).
\]
}

 

\thmrpf{}{}{

}{

}
 
 
 
%\chapter{Clase 9}
\clasedate{28 de abril de 2025}

 
%\chapter{Clase 10}
\clasedate{30 de abril de 2025}

 
%\chapter{Clase 11}
\clasedate{07 de mayo de 2025}

 
\chapter{Clase 12}
\clasedate{10 de mayo de 2025}

\thmrpf{Teorema del valor medio}{}{
	Sea $n\in\mathbb{N}$, con $n\geq 1$.
	Sea $f:[a,b]\to\mathbb{R}^n$ continua en $[a,b]$ y diferenciable
	en $]a,b[$. Si existe $M>0$ tal
	\[|f'(t)|\leq M,\;\forall t\in]a,b[,\]
	entonces \[|f(b)-f(a)|\leq M(b-a)\]
}{
	Si $|\cdot|$ es la norma euclidiana ya se probó.\\
	Si $\|\cdot\|$ es otra norma
	en $\mathbb{R}^n$, entonces
	existe $\alpha,\beta\in\mathbb{R}^+$ tal que
	\[\begin{cases}
		\forall w\in\mathbb{R}^n,\; \|w\|\leq \alpha|w|\\
		\forall w\in\mathbb{R}^n,\; |w|\leq \beta\|w\|
	\end{cases}\]
	
	Cuando $f$ es de clase $\mathcal{C}^1$ en $]a,b[$ (Se puede cambiar esto por
	$f'$ es integrable en cada
	subintervalo compacto $[c,d]\subset]a,b[$)\\
	Dado $[c,d]\subset]a,b[$, $f'$ es continua en $[c,d]$
	
	Del T.F. Cálculo
	\[\int_c^d f'(t)dt = f(d)-f(c)\]
	Por tanto, por desigualdad real
	\[|f(d)-f(c)| = \left|\int_c^d f'(t)dt\right| \leq \int_c^d |f'(t)|dt \leq \int_c^d M dt = M(d-c)\]
	ya que $\forall t\in]a,b[$, $|f'(t)|\leq M$
	
	$\forall$ subintervalo compacto $[c,d]\subset]a,b[$
	\begin{equation}
		|f(d)-f(c)|\leq M(d-c)
		\label{eq:class12-1}
	\end{equation}
	Considere $k_0\in\mathbb{N}$ tal $\frac{1}{k_0}<\frac{b-a}{2}$
	
	De \eqref{eq:class12-1}, $\forall k\in\mathbb{N}$, con $k\geq k_0$, $d_k = b-\frac{1}{k}, c_k = a+\frac{1}{k}$
	\[\abs{f\qty(b-\frac{1}{k})-f\qty(a+\frac{1}{k})}\leq M(d_k-c_k)\]
	Haciendo $k\to+\infty$, de la continuidad de $f:[a,b]\to\mathbb{R}^n$
	\[|f(b)-f(a)|\leq M(b-a)\]
}
\lemp{}{
	Sean $f:[c,d] \to\mathbb{R}$ y $\varphi \colon [a,b] \to \R$ continuas  y diferenciables $[a,b] $. Si $\abs{f'(t) \leq \varphi'(t)}$ y $\varphi(t)$ es estrictamente creciente, es decir, $\varphi'(t)>0$ para todo $t \in [a,b] $, entonces 
	$$
	\abs{f(b)-f(a) } \leq \varphi(b)-\varphi(c)
	$$
}{
	Sea  $[c, d] \subset ]a, b[$ subintervalo compacto fijo y arbitrario. 
	
	Mostremos que  $\abs{f(d)-f(c)} \leq \varphi(d)-\varphi(c)$ (basta con hacer $c \to a^{+}$ y $d \to b^{-}$) 
	
	Por contradicción, supongamos que 
	$$
	\abs{f(d)-f(c)} \geq \varphi(d)-\varphi(c)
	$$
	Elegimos $A \in \left] 1, \frac{\abs{f(d)-f(c)}}{\varphi(d)-\varphi(c)}\right[  $

$$
\rightarrow 1 < A < \frac{\abs{f(d)-f(c)}}{\varphi(d)-\varphi(c)}
$$
\begin{equation} 
\rightarrow A[ \varphi(d) -\varphi(c)] < \abs{f(d)-f(c)}
\label{eq:class12-2}
\end{equation}

Si 
\begin{align*}
	A\pqty{ \varphi\pqty{\frac{c+d}{2}} -\varphi(c)} &\geq \abs{ f\pqty{\frac{c+d}{2}} -f(c) }\\
	\text{ y }\quad A\pqty{ \varphi(d)- \varphi\pqty{\frac{c+d}{2}} } &\geq \abs{ f(d)-f\pqty{\frac{c+d}{2}}  }
\end{align*}

Sumando 

\begin{align*}
	A\pqty{  \varphi(d)  -\varphi(c)} &\geq \abs{ f\pqty{\frac{c+d}{2}} -f(c) } + \abs{ f(d)-f\pqty{\frac{c+d}{2}}}\\
 	&\geq \abs{ f\pqty{\frac{c+d}{2}} -f(c)+ f(d)-f\pqty{\frac{c+d}{2}}}\\
 	A\pqty{  \varphi(d)  -\varphi(c)} & \geq \abs{  f(d) -f(c)  } \qquad \text{Contradice a \eqref{eq:class12-2}}
\end{align*}

Entonces necesariamente,
\begin{equation}
A\cdot\qty[\varphi\qty(\frac{c+d}{2})-\varphi(c)] < \qty|f\qty(\frac{c+d}{2})-f(c)| \label{eq:class12-3}
\end{equation}
o,  
\begin{equation}
	A\cdot\qty[\varphi(d)-\varphi\qty(\frac{c+d}{2})] < \qty|f(d)-f\qty(\frac{c+d}{2})|
	\label{eq:class12-4}
\end{equation}


si ocurre \eqref{eq:class12-3}, elegimos $[c_1,d_1]:=\qty[c,\frac{c+d}{2}]$

y si ocurre \eqref{eq:class12-4}, elegimos $[c_1,d_1]:=\qty[\frac{c+d}{2},d]$

Tenemos que $A[\varphi(d_1)- \varphi(c_1)] < |f(d_1)-f(c_1)|$

Análogamente, se tienen dos posibilidades:

\begin{equation}
	A[\varphi(\frac{c_1+d_1}{2})-\varphi(c_1)] < |f(\frac{c_1+d_1}{2})-f(c_1)| 
		\label{eq:class12-5}
	\end{equation}
o
\begin{equation}
	A[\varphi(d_1)-\varphi(\frac{c_1+d_1}{2})] < |f(d_1)-f(\frac{c_1+d_1}{2})|
	\label{eq:class12-6}
\end{equation}
De este modo, continuando este argumento hemos construído una sucesión encajada de intervalos compactos
$\{[c_k,d_k]\}_{k\in\mathbb{N}}$ tal que:

$[c_k,d_k]\supset[c_{k+1},d_{k+1}]$ $\forall k\in\mathbb{N}\cup \{0\}$ con
$[c_0,d_0]\supset[c_1,d_1]\supset[c_2,d_2]\supset\cdots$ 

donde $\ell([c_k,d_k])=\frac{d-c}{2^k}$, $\forall k\in\mathbb{N}$
y $A[\varphi(d_k)-\varphi(c_k)]<|f(d_k)-f(c_k)|$

Así, $\bigcaplim{k\in\mathbb{N}}[c_k,d_k]=\{t_0\}\in[c,d] \subset ]a, b[$
$$
\therefore f \text{ y } \varphi \text{ son diferenciables en }   t_0\in]a,b[
$$

}


 \lem{Ejercicio}{
 	Sea $f : \mathbb{I} \to \mathbb{R}^n$ una función diferenciable en un punto $t_0 \in \interior{\mathbb{I}}$.  
 	Sean $\{c_k\}_{k\in\mathbb{N}}, \{d_k\}_{k\in\mathbb{N}} \subset \interior{\mathbb{I}}$ dos sucesiones tales que:
 	\begin{itemize}
 		\item $c_k < d_k$ para todo $k \in \mathbb{N}$,
 		\item $c_k \leq t_0 \leq d_k$ para todo $k \in \mathbb{N}$,
 		%\item $\lim_{k \to \infty} c_k = \lim_{k \to \infty} d_k = t_0$.
 	\end{itemize}
 	Entonces, se tiene que
 	\[
 	f'(t_0) = \lim_{k \to \infty} \frac{f(d_k) - f(c_k)}{d_k - c_k}.
 	\]
 }
 
 
	
	

	
	$$
	A\cdot\left(\frac{\varphi(d_k)-\varphi(c_k)}{d_k-c_k}\right)<\left|\frac{f(d_k)-f(c_k)}{d_k-c_k}\right|
	$$
	
	Haciendo $k\to\infty$, del Lema 1
	
	$$
	\varphi'(t_0)\leq A\cdot \varphi'(t_0)\leq|f'(t_0)|
	$$
	
	$\exists t_0\in[c,d]\subset]a,b[$ tal que $\varphi'(t_0)<|f'(t_0)|$
	
	Esto contradice la hipótesis!
	
	Por ende, $|f(d)-f(c)|\leq\varphi(d)-\varphi(c)$
	
	 
	\corrp{}{Teorema}{
		Sea $f \colon [a, b] \to \mathbb{R}^n$ una función continua en $[a,b]$ y diferenciable en el intervalo abierto $]a, b[$, tal que
		\[
		f'(t) = \Ou \in \mathbb{R}^n \quad \text{para todo } t \in ]a, b[.
		\]
		Entonces, $f$ es constante en $[a, b]$.
	}{
		Dado que $f'(t) = \Ou$ para todo $t \in ]a, b[$, se tiene:
		\[
		\|f'(t)\| = 0 \quad \text{para todo } t \in ]a, b[.
		\]
		Sea $[c,d] \subset ]a, b[$ un subintervalo compacto arbitrario. Como $f$ es continua en $[c,d]$ y derivable en $]c,d[$, podemos aplicar el teorema del valor medio (T.V.M.) :
		\[
		\exists \, \xi \in ]c,d[ \text{ tal que } f(d) - f(c) = f'(\xi)(d - c).
		\]
		Pero como $f'(\xi) = \mathbf{0}$, se deduce que
		\[
		f(d) - f(c) = \Ou \quad \Rightarrow \quad f(d) = f(c).
		\]
		Como esto vale para todo par de puntos $c, d \in ]a, b[$, se concluye que $f$ es constante en $]a, b[$.  
		Finalmente, como $f$ es continua en $[a,b]$, esta constancia se extiende al intervalo cerrado.
		
		\[
		\therefore \quad f \text{ es constante en } [a, b].
		\]
	}
	
	\section{Teorema Fundamental del Cálculo}
	
\thmrpf{Teorema Fundamental del Cálculo}{}{
\begin{align*}
	\text{Sea} \quad f: [a,b] \subset \mathbb{R} &\longrightarrow \mathbb{R}^n \text{ de clase }  C^1 \text{ (Se puede cambiar esto por } f' \text{ integrable)}\\
	t &\longmapsto f(t) = (f_1(t), \ldots, f_n(t))
\end{align*}
Entonces
\[\int_a^b f(t) dt = f(b) - f(a)\]
}{
	  Como $ f = (f_1, f_2, \ldots, f_n)$  es de clase $C^1$
	
	$\Rightarrow$
	$f$ es diferencial de $[a,b]$ y $f' = (f_1', f_2', \ldots, f_n)$ es continua
	
	$\Rightarrow$
	$f_1, f_2, \ldots, f_n$ es diferenciable.
	$[a,b]$ y $f_1', f_2', \ldots, f_n$ son continuas.
	
	$\Rightarrow$
	$f_1, f_2, \ldots, f_n : [a,b] \rightarrow \mathbb{R}$ son de clase $C^1$. ($f_i^{'}$ es continua $\forall i \in \{1, \ldots, n\}$ )
	
	Del T.F. Cálculo para funciones reales de variable
	real:
	\[\int_a^b f_i^{'}(t) dt = f_i(b) - f_i(a), \quad \forall i \in \{1, 2, \ldots, n\}\]
	
	Además, $f'$ es continua, $D_{f'} = \varnothing, m(D_{f'}) = 0$
	\begin{align*}
		\int_a^b f'(t) dt &= \pqty{ \int_a^b f_1^{'}(t) dt , \cdots , \int_a^b f_n^{'}(t) dt}\\
		&=\pqty{f_1(b) - f_1(a) , \cdots , f_n(b) - f_n(a)}\\
		& = f(b)-f(a)
	\end{align*}
	
}

\propp{
	Sea $f \colon [a,b]   \longrightarrow \mathbb{R}^n$ integrable. Entonces
\begin{align*}
	 \abs{f}\colon [a,b] & \longrightarrow \mathbb{R}^n \text{ es integrable } \\
	t &\longmapsto \abs{f}(t) = \abs{f(t)}=  \abs{\cdot} \circ f
\end{align*}
y además, 
$$
\abs{\int_{0}^{b} f(t) \dd{t}} \leq \int_{a}^{b} \abs{f(t)} \dd{t}
$$
\rmk{
Si $f$ es continua en $t_0 \rightarrow \abs{f}$ es continua en $t_0$. 
}
}{
$\forall x,y \in [a, b]$,
\begin{equation}
	\abs{\abs{f(x)}-\abs{f(y)}} \leq \abs{f(x)-f(y)}
	\label{eq:class12-7}
\end{equation}
\begin{align*}
	\abs{f(x)} &= \abs{f(x)-f(y)+f(y)}\\
	& \leq \abs{f(x)-f(y)}+\abs{f(y)}
\end{align*}
Si $f$ es continua en $x_0 \in  [a, b] \rightarrow \abs{f}$ es continua en $x_0 \in [a, b]$
$$
\forall \varepsilon>0, \exists \delta >0, \forall x \in [a, b], \abs{x-x_0} < \delta \rightarrow \abs{f(x)-f(x_0)}< \varepsilon
$$
De \eqref{eq:class12-7}
$$
\abs{\abs{f(x)}- \abs{f(x_0)}} \leq \abs{f(x)-f(x_0)}< \varepsilon
$$
$\rightarrow \abs{f}$ es continua en $x_0$

Si $f$ es continua en $x_0 \in  [a, b]$ equivale  a $D_{\abs{f}} \subset D_{f} = \{x \in [a, b]\colon f \text{ no es continua en } x \}$\\
$D_{\abs{f}}= \{x \in [a, b]\colon \abs{f} \text{ no es continua en } x \} $

Como $f$ es integrable $\stackrel{Lebesgue}{=} m\pqty{D_f} = 0$\\
Pero
 $$0 \leq m\pqty{D_{\abs{f}}} \leq m\pqty{D_{f}}$$
$$m\pqty{D_{\abs{f}}} = 0\stackrel{Lebesgue}{=} \abs{f} \quad \text{es integrable. }$$ 
}

Como
\begin{equation*}
	\int_a^b f = \int_a^b f(t) \dd{t} \in \mathbb{R}^n,
\end{equation*}
dado que \( f \) es integrable en \( [a, b] \), se tiene que:

Para toda partición puntillada \( P^* = (P, \xi) \), es decir, toda partición \( P = \{t_0 = a < t_1 < \dots < t_k = b\} \) del intervalo \([a,b]\), y toda puntillación \( \xi = (\xi_i)_{i=1}^k \) con \( \xi_i \in [t_{i-1}, t_i] \), se cumple que:

\begin{equation}
	\forall \varepsilon > 0,\ \exists \delta > 0\ \text{tal que si } \abs{P} < \delta,\ \text{entonces } \abs*{ \sum_{i=1}^k (t_i - t_{i-1}) f(\xi_i) - \int_a^b f } < \varepsilon.
	\label{eq:class12-8}
\end{equation}

Análogamente, dado que \( \abs{f} \) es integrable en \([a,b]\), es decir:
\begin{equation*}
	\int_a^b \abs{f} = \int_a^b \abs{f(t)} \dd{t}, 
\end{equation*}
entonces también se cumple que:
\begin{equation}
	\forall \varepsilon > 0,\ \exists \delta > 0\ \text{tal que si } \abs{P} < \delta,\ \text{entonces } \abs*{ \sum_{i=1}^k (t_i - t_{i-1}) \abs{f}(\xi_i) - \int_a^b \abs{f} } < \varepsilon.
	\label{eq:class12-9}
\end{equation}

Además, para cualquier partición y puntillación, por desigualdad del valor absoluto en \( \mathbb{R}^n \), se cumple:
\begin{equation}
	\abs*{ \sum_{i=1}^k (t_i - t_{i-1}) f(\xi_i) } \leq \sum_{i=1}^k (t_i - t_{i-1}) \abs{f(\xi_i)}.
	\label{eq:class12-10}
\end{equation}

Ahora, sea \( \varepsilon > 0 \). Por \eqref{eq:class12-8} y \eqref{eq:class12-9}, existe \( \delta > 0 \) tal que si \( \abs{P} < \delta \), entonces:
\begin{align}
	\abs*{ \sum_{i=1}^k (t_i - t_{i-1}) f(\xi_i) - \int_a^b f } &< \frac{\varepsilon}{2}, \label{eq:class12-11} \\
	\abs*{ \sum_{i=1}^k (t_i - t_{i-1}) \abs{f}(\xi_i) - \int_a^b \abs{f} } &< \frac{\varepsilon}{2}. \label{eq:class12-12}
\end{align}

Luego, usando la desigualdad \eqref{eq:class12-10}, se tiene:
\[
\abs*{ \sum_{i=1}^k (t_i - t_{i-1}) f(\xi_i) }
\leq \sum_{i=1}^k (t_i - t_{i-1}) \abs{f(\xi_i)}.
\]

Aplicando \eqref{eq:class12-11} y \eqref{eq:class12-12}, se obtiene:
\begin{align*}
	\abs*{ \int_a^b f }
	&\leq \abs*{ \sum_{i=1}^k (t_i - t_{i-1}) f(\xi_i) } + \abs*{ \int_a^b f - \sum_{i=1}^k (t_i - t_{i-1}) f(\xi_i) } \\
	&< \sum_{i=1}^k (t_i - t_{i-1}) \abs{f(\xi_i)} + \frac{\varepsilon}{2} \\
	&< \abs*{ \int_a^b \abs{f} } + \frac{\varepsilon}{2} + \frac{\varepsilon}{2} = \int_a^b \abs{f} + \varepsilon.
\end{align*}

Como \( \varepsilon > 0 \) fue arbitrario, se concluye que:
\begin{equation*}
	\abs*{ \int_a^b f } \leq \int_a^b \abs{f}.
	\label{eq:final-ineq}
\end{equation*}

\qed





 
\chapter{Clase 13}
\clasedate{12 de mayo de 2025}

\lemp{}{
Sea \( f \colon I \to \mathbb{R} \) una función definida en un intervalo \( I \subset \mathbb{R} \), y sea \( x_0 \in \operatorname{int}(I) \) un punto interior.

Si \( f \) es diferenciable en \( x_0 \), entonces para toda sucesión \( (t_k)_{k \in \mathbb{N}}, (s_k)_{k \in \mathbb{N}} \subset I \) tal que:
\begin{itemize}
	\item \( t_k \neq s_k \) para todo \( k \in \mathbb{N} \),
	\item \( \lim\limits_{k \to \infty} t_k = x_0 = \lim\limits_{k \to \infty} s_k \),
	\item \( t_k \leq x_0 \leq s_k \) para todo \( k \in \mathbb{N} \),
\end{itemize}
se cumple que:
\[
\lim_{k \to \infty} \frac{f(s_k) - f(t_k)}{s_k - t_k} = f'(x_0).
\]
}{
Sean $(t_k)_{k \in \N}, (s_k)_{k \in \N} \subset \I$, con $t_k \neq s_k \quad \forall k \in \N$, tales que
\begin{equation} 
	t_k \to x_0 \text{ y } s_k \to x_0\\
	\label{eq:class13-1}
\end{equation} 

\textbf{Caso I.} $\forall k \in \N$, $t_k <x_0< s_k$
$$
\abs{\frac{f(s_k)-f(t_k)}{s_k-t_k} -f'(x_0)}
$$
$$
=\abs{\frac{f(s_k)-f(x_0)+f(x_0)-f(t_k)}{s_k-t_k} -f'(x_0)}
$$
$$
=\abs{\pqty{\frac{s_k-x_0}{s_k-t_k} } \frac{f(s_k)-f(x_0)}{s_k-x_0} + \pqty{\frac{x_0-t_k}{s_k-t_k} } \frac{f(x_0)-f(t_k)}{x_0-t_k}   -f'(x_0)}
$$
Sea $\alpha_k = \pqty{\frac{s_k-x_0}{s_k-t_k} }$ y $\beta_k = \pqty{\frac{x_0-t_k}{s_k-t_k} } \rightarrow \alpha_k+\beta_k=1, \forall k \in \N$
$$
=\abs{
\alpha_k 
\frac{f(s_k)-f(x_0)}{s_k-x_0} + \beta_k
 \frac{f(x_0)-f(t_k)}{x_0-t_k}   - (\alpha_k +\beta_k )f'(x_0)}
$$
$$
=\abs{
	\alpha_k \bqty{
	\frac{f(s_k)-f(x_0)}{s_k-x_0} - f'(x_0) } + \beta_k \bqty{
	\frac{f(x_0)-f(t_k)}{x_0-t_k} - f'(x_0) }}
$$ 
Debemos mostrar que 
$$
\lim_{k \to \infty} \frac{f(s_k)-f(t_k)}{s_k-t_k} = f'(x_0)
$$
Como $f \in \Rn$ es diferenciable en $x_0$, entonces
$$
\exists f'(x_0) = \lim_{h \to 0} \frac{f(x_0+h)-f(x_0)}{h}
$$
$$
f'(x_0) = \lim_{x \to x_0 } \frac{f(x)-f(x_0)}{x-x_0} \quad \text{existe}
$$
De la caracterización de límites con sucesiones en $\Rn$ se tiene: $\forall (p_k)_k \subset \I - \{x_0\} $, si
$$
\lim_{k \to \infty} p_k = x_0 \longrightarrow \lim_{k \to \infty} \frac{f(p_k)-f(x_0)}{p_k-x_0} = f'(x_0)
$$

De \eqref{eq:class13-1}, tenemos que 
$$
 \lim_{k \to \infty} \frac{f(t_k)-f(x_0)}{t_k-x_0} = f'(x_0)\quad \text{y}  \quad \lim_{k \to \infty} \frac{f(s_k)-f(x_0)}{s_k-x_0} = f'(x_0)
$$
Dado $\varepsilon>0, \exists k_0 \in \N, \forall k \in \N, k\geq k_0$, entonces
$$ 
\abs{\frac{ f(t_k)-f(x_0) }{t_k-x_0} - f'(x_0) }< \varepsilon \wedge \abs{	\frac{f(s_k)-f(x_0)}{s_k-x_0} - f'(x_0) }< \varepsilon      
$$ 
De \textbf{OBS 1.} si $k\geq k_0$ entonces
$$
\abs{\frac{f(s_k)-f(x_0)}{s_k-x_0} - f'(x_0) } \leq  \alpha_k \abs{\frac{f(s_k)-f(x_0)}{s_k-x_0} - f'(x_0) } +\beta_k \abs{\frac{f(x_0)-f(t_k)}{x_0-t_k} - f'(x_0) } 
$$ 
\begin{align*}
	\abs{\frac{f(s_k)-f(x_0)}{s_k-x_0} - f'(x_0) } &\leq  \alpha_k \abs{\frac{f(s_k)-f(x_0)}{s_k-x_0} - f'(x_0) } +\beta_k \abs{\frac{f(x_0)-f(t_k)}{x_0-t_k} - f'(x_0) } \\
	&< \alpha_k \varepsilon + \beta_k\varepsilon\\
	&<(\alpha_k+\beta_k)\varepsilon\\
	&<\varepsilon
\end{align*}
\textbf{Caso II.} $\exists (t_{i(k)})_{k \in \N} \subset (t_k)$ tal que $t_{i(k)} = x_0, \forall k \in \N.$

\rmk{Probar que en casos diferentes a $t_k \rightarrow x_0 \leftarrow s_k$ no cumple.}
}

\section{Desigualdad del valor medio}
\thmrpf{Teorema del Valor Medio Para Funciones Vectoriales de Variable Real}{}{
Sea \( f = (f_1, f_2, \ldots, f_n) \colon [a, b] \to \mathbb{R}^n \) una función continua en \( [a, b] \) y diferenciable en \(]a, b[ \).  
Supongamos que existe una constante \( M > 0 \) tal que
\[
\forall t \in ]a, b[, \quad \abs{f'(t)} \leq M.
\]
Entonces se cumple:
\[
\abs{f(b) - f(a)} \leq M(b - a).
\]
}{
Prueba, Del lema 2 anterior. Si $\varphi \colon [a, b] \to \R $ es continua y diferenciable en $]a, b[$ tal que $M=\varphi'(t), \forall t \in ]a, b[$, entonces 
$$
\abs{f(b)-f(a)} \leq \varphi(b)-\varphi(a)
$$


}

\corp{}{
	Sea \( f = (f_1, f_2, \ldots, f_n) \colon [a, b] \to \mathbb{R}^n \) una función continua en \( [a, b] \) y derivable en \( ]a, b[ \).  
Si \( f'(t) = \mathbf{0} \) para todo \( t \in (a, b) \), entonces \( f \) es constante en \( [a, b] \).
}{
Supongamos que $f'(t) = \Ou, \forall t \in ]a, b[$

Dado $k \in \N$, $$\abs{f'(t)} = 0 < \frac{1}{k}, \forall t \in ]a, b[ \supset [c, d]$$

Dados $c, d \in ]a, b[$, con $c<d$

Aplicando el T.V.M a $ \eval{f}_{[c,d]} $m concluimos que 


$\abs{f(d)-f(c)} \leq \frac{1}{k} (d-c)$

Como $k \in \N$ fue arbitrario,
$$
\abs{f(d)-f(c)}\leq \frac{1}{k}, \quad \forall k \in \N
$$
Haciendo $k \to \infty$
$$
0 \leq \abs{f(d)-f(c)}\leq 0
$$
Así, $f(d)-f(c)\leq 0$,
$$
\rightarrow f(d) = f(c), \forall c, d \in ]a, b[ \text{ con } c<d
$$

Como $f$ es continua en $[a, b] $ entonces $f$ es constante.
}

\pf{
Prueba2 del corolario.\\
Supongamos que $f'(t) = \Ou \in \Rn$$\forall t \in ]a, b[$

$$
(f'_1(t), f'_2(t), \ldots, f'_n(t)) = (0, 0, \ldots, 0)
$$
$$
\rightarrow f'_i (t) = 0, \forall t \in ]a, b[, \forall i \in \{1,2, \ldots, n\}
$$
Como $f_i \colon [a, b] \to \R $ es continua
$$
f_i \text{  es constante} \quad \forall i \in \{1,2, \ldots, n\}
$$
}


\section{Fórmula de Taylor con resto de Lagrange}

Sea \( f \colon [a, b] \to \mathbb{R}^n \) una función de clase \( C^{p-1} \) en \( [a, b] \), y además \( p \) veces derivable en \( ]a, b[ \), tal que existe \( M > 0 \) con
\[
\abs{f^{(p)}(t)} \leq M \quad \forall t \in \; ]a, b[.
\]
Entonces, se cumple que
\[
f(b) = f(a) + (b-a)f'(a) + \frac{(b-a)^2}{2!} f''(a) + \cdots + \frac{(b-a)^{p-1}}{(p-1)!} f^{(p-1)}(a) + r_p,
\]
donde \( r_p \in \mathbb{R}^n \) satisface
\[
\|r_p\| \leq \frac{M(b-a)^p}{p!}.
\]

\pf{
	Sea la función auxiliar \( g \colon [a, b] \to \mathbb{R}^n \) definida por
	\[
	g(t) = f(t) + (b-t)f'(t) + \frac{(b-t)^2}{2!} f''(t) + \cdots + \frac{(b-t)^{p-1}}{(p-1)!} f^{(p-1)}(t).
	\]
	
	Como \( f \in C^{p-1}([a, b]) \) y \( f^{(p)} \) existe en \( ]a, b[ \), entonces \( g  \) es diferenciable en $]a, b[$.
	
	Calculemos la derivada de \( g \) en \( ]a, b[ \). Derivando término a término y aplicando la regla del producto:
	\begin{align*}
		g'(t) &= f'(t) + (-1)f'(t) + (b-t)f''(t) \\
		&\quad + (-1)(b-t)f''(t) + \frac{(b-t)^2}{2!}f^{(3)}(t) \\
		&\quad + \cdots + \frac{(b-t)^{p-2}}{(p-2)!}f^{(p-1)}(t) - \frac{(p-1)(b-t)^{p-2}}{(p-1)!}f^{(p-1)}(t) \\
		&\quad + \frac{(b-t)^{p-1}}{(p-1)!}f^{(p)}(t).
	\end{align*}
	
	Todos los términos intermedios se cancelan debido a las derivadas de los productos, quedando únicamente:
	\[
	g'(t) = \frac{(b-t)^{p-1}}{(p-1)!} f^{(p)}(t), \quad \forall t \in ]a, b[.
	\]
	
	Por lo tanto,
	\[
	\|g'(t)\| = \frac{(b-t)^{p-1}}{(p-1)!} \|f^{(p)}(t)\| \leq \frac{M(b-t)^{p-1}}{(p-1)!}.
	\] 
	
	Definamos, $\varphi \colon [a, b] \to \R$ por $$\varphi(t) = - \frac{M (b-t)^p}{(p)!}$$
	donde 
	$$
	\varphi'(t) = \frac{M (b-t)^{p-1}}{(p-1)!}>0, \quad \forall t \in ]a, b[ \quad \wedge \quad \abs{g'(t)} \leq g(t), \quad\forall t \in ]a, b[ 
	$$
	
	Del Lema 2
	
	\begin{align*}
		\abs{g(b)-g(a)} &\leq \varphi(b)-\varphi(a)\\
		\abs{g(b)-g(a)} &\leq \frac{M (b-t)^p}{(p)!} \\
		\abs{r_p} & \leq \frac{M (b-t)^p}{(p)!}
	\end{align*}
	
}

\begin{itemize}
	\item Sea $f \colon [a, b] \rightarrow \Rn$ una función.
	
	Dada una partición 
	$$
	P = \{t_0 =a< t_1< t_2<\ldots< t_k =b\}
	$$
	La longitud de $f$ asociada a $P$ es
	$$
	\ell(f, P) = \sum_{i=1}^{k} \abs{f(t_1), \ldots, f(t_{i-1})}
	$$
	\item Decimos que $f \colon [a, b] \rightarrow \Rn$ es \textbf{rectificable} si $\exists M>0$ tal que $\forall$ partición $P$ de $[a, b]$
	$$
	\ell(f, P) \leq M
	$$
	En este caso, definamos la longitud de $f \colon [a, b] \rightarrow \Rn$ por 
	\begin{align*}
		\ell(t) & = \sup \left\lbrace \ell(f, P) \in \R \colon P \quad \text{es una partición de} \quad [a, b] \right\rbrace \\
		&= \sup\limits_{P}  \ell(f, P)
	\end{align*}
	\ex{
		Sea $f \colon [a, b ]\to \Real{2}$
		
		
		}
\end{itemize}

 
%\chapter{Clase 14}
\clasedate{14 de mayo de 2025}

\begin{itemize}
	\item Un camino $f \colon [a, b] \to \Rn$ es una función vectorial de variable real continua.
	\item La función vectorial
	\begin{align*}
		g \colon [0, 1] & \longrightarrow \Rightarrow \Real{2}\\
		t & \longmapsto \begin{cases} \pqty{t,\frac{\sin t}{t}}, & \text{si } t \neq 0, \\
			(0,0),  & \text{si } t = 0.
		\end{cases}
	\end{align*}
	claramente $g$ es continua 
	
	$$
	\forall t \in \R -\{0\}, \abs{t \sin(\frac{1}{t})} = \abs{t} \abs{\sin(\frac{1}{t}) } \leq \abs{t}
	$$
	$$
	\forall t \in \R -\{0\}, - \abs{t}\leq t \sin(\frac{1}{t})\leq   \abs{t}
	$$
	Del teorema del sandwich, 
	$$
	\lim\limits_{t \to 0  }  t \sin(\frac{1}{t}) =0
	$$
	
	Mostramos $g\colon [0, 1] \to \Real{2}$ es continua pero no es rectificable.
	
	$\forall k \in \N, \exists P_k = \{t_0=a<t_1<t2< \ldots< t_k=b\}$  partición de $[a, b]$ tal que
	$$ 
	  \ell  (g, P_k ) = \sum_{i=1}^{k} \abs{g(t_i)-g(t_{i-1})} \geq 2 \bqty{\sum_{i=1}^{k} \frac{1}{2i-1}}
	$$
	Si \( k \to \infty \), entonces \( \sum_{i=1}^{k} \frac{1}{2i - 1} \to \infty \),  la serie diverge.
	
	
	Por lo tanto $g$ no es rectificable
	
	\item Sea 
	
	
	En caso $f$ sea rectificable la \textbf{longitud }
	
	$f$ es \textbf{Rectificable} $\leftrightarrow \exists M>0$
	
	
	
\end{itemize}
\newpage

\thmrpf{}{}{
Sea \( f \colon [a, b] \to \mathbb{R}^n \) una función.  
Sean \( P \) y \( Q \) particiones del intervalo \( [a, b] \) tales que \( P \subset Q \), es decir, \( Q \) es una refinación de \( P \). Entonces,
\[
\ell(f, P) \leq \ell(f, Q),
\]
donde \( \ell(f, P) \) denota la longitud aproximada de la curva \( f \) correspondiente a la partición \( P \).
}{

$P=\{t_0=0<t_1<t_2<\ldots<t_k = b\}$\\
$Q= P \cup \{t^{*}\}$, donde $t^{*} \notin P$

$Q =\{t_0=0<t_1<t_2<\ldots<t_{i-1}<t^{*}<t_i< \ldots<t_k = b\}$ para algún $i \in \{1, 2, \ldots, k\}$


Mostremos que
$$
\ell(f, P) \leq \ell(f, Q)
$$
En efecto, 
$$
\ell(f, P) = \sum_{j=1}^{k} \abs{f(t_j)-f(t_{j-1})}
$$
$$
\ell(f, Q) = \sum_{j=1}^{i-1} \abs{f(t_j)-f(t_{j-1})}
 +
\abs{f(t^{*})-f(t_{i-1})}
+
\abs{f(t_{i}) -f(t^{*}) }
+
 \sum_{j=i+1}^{k} \abs{f(t_j)-f(t_{j-1})}
$$
Además $\abs{f(t_i)-f(t_{i-1})} \geq \abs{f(t^{*})-f(t_{i-1})}
+
\abs{f(t_{i}) -f(t^{*}) }$


Así, 
$$
\ell(f, Q)\geq \sum_{j=1}^{i-1} \abs{f(t_j)-f(t_{j-1})}
+
\abs{f(t_i)-f(t_{i-1})} 
+
\sum_{j=i+1}^{k} \abs{f(t_j)-f(t_{j-1})}
$$
$$
\ell(f, P) \leq \ell(f, Q)
$$



}

\rmkb{
Si $f\colon [a, b] \to \Rn$ es una función vectorial rectificable, entonces $f$ es acotada.
\pf{
Sea $f \colon \to \Rn$ rectificable. Entonces
$\{\ell(f, P)  \in \R \colon P\quad \text{ es partición de } [a, b] \}$ es acotado superiormente
$$
\exists \sup\{\ell(f, P) \in \R \colon P \text{ es partición de } [a, b]\} = \ell(f) \in \R
$$
Dado $t \in ]a, b[,$ considere $Q_t = \{a =t_0< t=t_1< b = t_2\}$
 
\begin{align*}
	\ell(f, Q_t) &\leq \ell(f)\\
	\abs{f(t)-f(a)} \leq \abs{f(t)-}+\abs{ }&\leq \ell(f)
\end{align*}
$$
\rightarrow f(t) \in B\bqty{f(a), \ell(f)}
$$
$\therefore f$ es acotada.
}
}


\lemp{}{
Sean $f \colon [a, b] \to \Rn$ función vectorial rectificable.\\
Sea $P_0 = \{t_0 =a <t_1<t_2<t_3<\ldots<t_k=b\}$ una partición en $[a, b]$. Entonces,
$$
\ell(f) = \sup\limits_{P } \ell(f, P) = \sup\limits_{P \supset P_0} \ell(f, P) 
$$
Donde $$
\sup\limits_{P } \ell(f, P) = \sup\{\ell(f, P) \in \R; P \text{ es partición de } [a, b] \}
$$
$$
\sup\limits_{P \supset P_0} \ell(f, P) =  \sup\{\ell(f, P) \in \R ; P \text{ es partición de } [a, b] \text{ y } P \supset P_0\}
$$
}{


Además, $$
\alpha = \ell(f, P) \in \R; P \text{ es partición de } [a, b]
$$
$$
\beta = \ell(f, P) \in \R ; P \text{ es partición de } [a, b] \text{ y } P \supset P_0
$$

Como $\beta \subset \alpha \rightarrow \sup(\beta) \leq \sup(\alpha)$

Por definición $ \ell(f) \in \R$, $\ell(f)=\sup(\alpha)$

Entonces
\begin{itemize}
\item  $\forall Q$ de partición  $[a, b], \ell(f, Q) \leq l(f)$
\item $\forall \varepsilon>0, \exists Q_{\varepsilon}$ partición de $[a, b]$ tal que $\ell(f)-\varepsilon < \ell(f, Q_{\varepsilon})$\\
$\ell(f) $ es la menor cota superior.

Dado $\varepsilon>0$, existe $Q_{\varepsilon}$ partición de $[a, b]$ tal que
$$
\ell(f)-\varepsilon < \ell(f, Q_{\varepsilon} ) \leq \ell \pqty{ }
$$

Así, $\forall \varepsilon>0, \ell(f)$
\end{itemize}
}

\thmrpf{ }{ }{ 
Sea \( f \colon [a, b] \to \mathbb{R}^n \) una función vectorial, y sea \( c \in (a, b) \).  
Entonces, \( f \) es \textbf{rectificable} si y sólo si las funciones restringidas $f \eval_{[a, c]}$ y $f \eval_{[c, d]}$   son rectificables.

En tal caso, se cumple:
\[
\ell(f) = \ell\pqty{f\vert_{[a, c]}} + \ell\pqty{f\vert_{[c, b]}}.
\]
}{ 
Sea

Por definición:
$$
\ell(f) = \sup\limits_{P \in \mathcal{P}} \ell(f, P) 
$$
Por lema anterior
$$
\ell(f) = \sup\limits_{P \in \mathcal{P}} \ell(f, P) = \sup\{\ell(f, Q); Q \in \mathcal{P}\pqty{[a, b]} \wedge c\in Q \supset P_0\}
$$
donde $P_0 = \{ a<c<b\}$

$$
=\sup \left\lbrace \sum_{j=1}^{i} \abs{f(t_j)-f(t_{j-1})}+ \sum_{j=i+1}^{k} \abs{f(t_j)-f(t_{j-1})}  \right\rbrace 
$$ 

$c \in Q =\{t_0=a<t_1<t_2<\ldots<t_{i-1}<t_{i}=c<t_{i+1}< \ldots<t_k = b\}$ para algún $i \in \{1, 2, \ldots, k\}$

Sea  $Q =\{t_0=a<t_1<t_2<\ldots<t_{i-1}<t_{i}=c<t_{i+1}< \ldots<t_k = b\}$


}

\thmrpf{arg}{arg}{
Sea 

}{

}


\rmkb{
Que $f \colon [a, b] \rightarrow \Rn$ sea rectificable  no depende de la norma en $\Rn$.\\
Sea $\abs{\cdot}$ y $\norm{\cdot}$ dos normas en $\Rn$
\pf{
Supongamos que $f$ es rectificable respecto a $\abs{\cdot}$
$$
+\infty > \ell_{\abs{\cdot}} (f) = \sup \left\lbrace \sum_{i=1}^{k} \abs{f(t_i)-f(t_{i-1})} \colon P=\{t_0=a<t_1<t_2<\ldots<t_k=b\}\right\rbrace 
$$


}
} 


\end{document}
