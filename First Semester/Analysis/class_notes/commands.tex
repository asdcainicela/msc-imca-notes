% Conjuntos numéricos
\newcommand{\R}{\mathbb{R}}            % Conjunto de los números reales R
\newcommand{\N}{\mathbb{N}}            % Conjunto de los números naturales N
\newcommand{\Real}[1]{\mathbb{R}^{#1}} % Espacio euclidiano R^n dado un argumento n
\newcommand{\Rn}{\Real{n}}             % Notación abreviada para R^n
\newcommand{\Rm}{\Real{m}}             % Notación abreviada para R^m
\newcommand{\Rp}{\Real{p}}             % Notación abreviada para R^p
\newcommand{\Ou}{\mathds{O}}           % Letra O en doble trazo, requiere \usepackage{dsfont}
\newcommand{\I}{\mathrm{I}}  


% Funciones especiales y operadores
\newcommand{\lcm}{\operatorname{lcm}}  % Mínimo común múltiplo
\newcommand{\dom}{\operatorname{dom}}  % Dominio de una función
\newcommand{\Dom}{\operatorname{Dom}}  % Variante con mayúscula, para uso formal
\newcommand{\ran}{\operatorname{ran}}  % Rango o imagen de una función

\newcommand{\Ll}{\mathscr{L}}          % Letra L caligráfica, espacio de operadores lineales
\newcommand{\Lineal}[2]{\Ll\left( #1, #2 \right)}  % Espacio de operadores lineales de #1 en #2

\newcommand{\inner}[2]{\langle #1, #2 \rangle}     % Producto interno usual ⟨x, y⟩

\newcommand{\interior}[1]{\operatorname{int}\left(#1\right)}

% Others
\newcommand{\clasedate}[1]{%
	\vspace{-3.5em}
	\begin{center}
		\textit{Fecha — #1}
	\end{center}
	\vspace{-1em}
}

\makeatletter
\newcommand{\bigcuplim}[1]{%
	\@ifnextchar\bgroup
	{\@bigcuplim@withsup{#1}}% Si viene otro grupo, usa como superíndice
	{\bigcup\limits_{#1}}% Si no, solo subíndice
}
\newcommand{\@bigcuplim@withsup}[2]{%
	\bigcup\limits_{#1}^{#2}%
}

\newcommand{\bigcaplim}[1]{%
	\@ifnextchar\bgroup
	{\@bigcaplim@withsup{#1}}%
	{\bigcap\limits_{#1}}%
}
\newcommand{\@bigcaplim@withsup}[2]{%
	\bigcap\limits_{#1}^{#2}%
}

\newcommand{\caplim}[1]{%
	\@ifnextchar\bgroup
	{\@caplim@withsup{#1}}%
	{\cap\limits_{#1}}%
}
\newcommand{\@caplim@withsup}[2]{%
	\cap\limits_{#1}^{#2}%
}

\newcommand{\cuplim}[1]{%
	\@ifnextchar\bgroup
	{\@cuplim@withsup{#1}}%
	{\cup\limits_{#1}}%
}
\newcommand{\@cuplim@withsup}[2]{%
	\cup\limits_{#1}^{#2}%
}
\makeatother

