\chapter{Clase 13}
\clasedate{12 de mayo de 2025}

\lemp{}{
Sea \( f \colon I \to \mathbb{R} \) una función definida en un intervalo \( I \subset \mathbb{R} \), y sea \( x_0 \in \operatorname{int}(I) \) un punto interior.

Si \( f \) es diferenciable en \( x_0 \), entonces para toda sucesión \( (t_k)_{k \in \mathbb{N}}, (s_k)_{k \in \mathbb{N}} \subset I \) tal que:
\begin{itemize}
	\item \( t_k \neq s_k \) para todo \( k \in \mathbb{N} \),
	\item \( \lim\limits_{k \to \infty} t_k = x_0 = \lim\limits_{k \to \infty} s_k \),
	\item \( t_k \leq x_0 \leq s_k \) para todo \( k \in \mathbb{N} \),
\end{itemize}
se cumple que:
\[
\lim_{k \to \infty} \frac{f(s_k) - f(t_k)}{s_k - t_k} = f'(x_0).
\]
}{
Sean $(t_k)_{k \in \N}, (s_k)_{k \in \N} \subset \I$, con $t_k \neq s_k \quad \forall k \in \N$, tales que
\begin{equation} 
	t_k \to x_0 \text{ y } s_k \to x_0\\
	\label{eq:class13-1}
\end{equation} 

\textbf{Caso I.} $\forall k \in \N$, $t_k <x_0< s_k$
$$
\abs{\frac{f(s_k)-f(t_k)}{s_k-t_k} -f'(x_0)}
$$
$$
=\abs{\frac{f(s_k)-f(x_0)+f(x_0)-f(t_k)}{s_k-t_k} -f'(x_0)}
$$
$$
=\abs{\pqty{\frac{s_k-x_0}{s_k-t_k} } \frac{f(s_k)-f(x_0)}{s_k-x_0} + \pqty{\frac{x_0-t_k}{s_k-t_k} } \frac{f(x_0)-f(t_k)}{x_0-t_k}   -f'(x_0)}
$$
Sea $\alpha_k = \pqty{\frac{s_k-x_0}{s_k-t_k} }$ y $\beta_k = \pqty{\frac{x_0-t_k}{s_k-t_k} } \rightarrow \alpha_k+\beta_k=1, \forall k \in \N$
$$
=\abs{
\alpha_k 
\frac{f(s_k)-f(x_0)}{s_k-x_0} + \beta_k
 \frac{f(x_0)-f(t_k)}{x_0-t_k}   - (\alpha_k +\beta_k )f'(x_0)}
$$
$$
=\abs{
	\alpha_k \bqty{
	\frac{f(s_k)-f(x_0)}{s_k-x_0} - f'(x_0) } + \beta_k \bqty{
	\frac{f(x_0)-f(t_k)}{x_0-t_k} - f'(x_0) }}
$$ 
Debemos mostrar que 
$$
\lim_{k \to \infty} \frac{f(s_k)-f(t_k)}{s_k-t_k} = f'(x_0)
$$
Como $f \in \Rn$ es diferenciable en $x_0$, entonces
$$
\exists f'(x_0) = \lim_{h \to 0} \frac{f(x_0+h)-f(x_0)}{h}
$$
$$
f'(x_0) = \lim_{x \to x_0 } \frac{f(x)-f(x_0)}{x-x_0} \quad \text{existe}
$$
De la caracterización de límites con sucesiones en $\Rn$ se tiene: $\forall (p_k)_k \subset \I - \{x_0\} $, si
$$
\lim_{k \to \infty} p_k = x_0 \longrightarrow \lim_{k \to \infty} \frac{f(p_k)-f(x_0)}{p_k-x_0} = f'(x_0)
$$

De \eqref{eq:class13-1}, tenemos que 
$$
 \lim_{k \to \infty} \frac{f(t_k)-f(x_0)}{t_k-x_0} = f'(x_0)\quad \text{y}  \quad \lim_{k \to \infty} \frac{f(s_k)-f(x_0)}{s_k-x_0} = f'(x_0)
$$
Dado $\varepsilon>0, \exists k_0 \in \N, \forall k \in \N, k\geq k_0$, entonces
$$ 
\abs{\frac{ f(t_k)-f(x_0) }{t_k-x_0} - f'(x_0) }< \varepsilon \wedge \abs{	\frac{f(s_k)-f(x_0)}{s_k-x_0} - f'(x_0) }< \varepsilon      
$$ 
De \textbf{OBS 1.} si $k\geq k_0$ entonces
$$
\abs{\frac{f(s_k)-f(x_0)}{s_k-x_0} - f'(x_0) } \leq  \alpha_k \abs{\frac{f(s_k)-f(x_0)}{s_k-x_0} - f'(x_0) } +\beta_k \abs{\frac{f(x_0)-f(t_k)}{x_0-t_k} - f'(x_0) } 
$$ 
\begin{align*}
	\abs{\frac{f(s_k)-f(x_0)}{s_k-x_0} - f'(x_0) } &\leq  \alpha_k \abs{\frac{f(s_k)-f(x_0)}{s_k-x_0} - f'(x_0) } +\beta_k \abs{\frac{f(x_0)-f(t_k)}{x_0-t_k} - f'(x_0) } \\
	&< \alpha_k \varepsilon + \beta_k\varepsilon\\
	&<(\alpha_k+\beta_k)\varepsilon\\
	&<\varepsilon
\end{align*}
\textbf{Caso II.} $\exists (t_{i(k)})_{k \in \N} \subset (t_k)$ tal que $t_{i(k)} = x_0, \forall k \in \N.$

\rmk{Probar que en casos diferentes a $t_k \rightarrow x_0 \leftarrow s_k$ no cumple.}
}

\section{Desigualdad del valor medio}
\thmrpf{Teorema del Valor Medio Para Funciones Vectoriales de Variable Real}{}{
Sea \( f = (f_1, f_2, \ldots, f_n) \colon [a, b] \to \mathbb{R}^n \) una función continua en \( [a, b] \) y diferenciable en \(]a, b[ \).  
Supongamos que existe una constante \( M > 0 \) tal que
\[
\forall t \in ]a, b[, \quad \abs{f'(t)} \leq M.
\]
Entonces se cumple:
\[
\abs{f(b) - f(a)} \leq M(b - a).
\]
}{
Prueba, Del lema 2 anterior. Si $\varphi \colon [a, b] \to \R $ es continua y diferenciable en $]a, b[$ tal que $M=\varphi'(t), \forall t \in ]a, b[$, entonces 
$$
\abs{f(b)-f(a)} \leq \varphi(b)-\varphi(a)
$$


}

\corp{}{
	Sea \( f = (f_1, f_2, \ldots, f_n) \colon [a, b] \to \mathbb{R}^n \) una función continua en \( [a, b] \) y derivable en \( ]a, b[ \).  
Si \( f'(t) = \mathbf{0} \) para todo \( t \in (a, b) \), entonces \( f \) es constante en \( [a, b] \).
}{
Supongamos que $f'(t) = \Ou, \forall t \in ]a, b[$

Dado $k \in \N$, $$\abs{f'(t)} = 0 < \frac{1}{k}, \forall t \in ]a, b[ \supset [c, d]$$

Dados $c, d \in ]a, b[$, con $c<d$

Aplicando el T.V.M a $ \eval{f}_{[c,d]} $m concluimos que 


$\abs{f(d)-f(c)} \leq \frac{1}{k} (d-c)$

Como $k \in \N$ fue arbitrario,
$$
\abs{f(d)-f(c)}\leq \frac{1}{k}, \quad \forall k \in \N
$$
Haciendo $k \to \infty$
$$
0 \leq \abs{f(d)-f(c)}\leq 0
$$
Así, $f(d)-f(c)\leq 0$,
$$
\rightarrow f(d) = f(c), \forall c, d \in ]a, b[ \text{ con } c<d
$$

Como $f$ es continua en $[a, b] $ entonces $f$ es constante.
}

\pf{
Prueba2 del corolario.\\
Supongamos que $f'(t) = \Ou \in \Rn$$\forall t \in ]a, b[$

$$
(f'_1(t), f'_2(t), \ldots, f'_n(t)) = (0, 0, \ldots, 0)
$$
$$
\rightarrow f'_i (t) = 0, \forall t \in ]a, b[, \forall i \in \{1,2, \ldots, n\}
$$
Como $f_i \colon [a, b] \to \R $ es continua
$$
f_i \text{  es constante} \quad \forall i \in \{1,2, \ldots, n\}
$$
}


\section{Fórmula de Taylor con resto de Lagrange}

Sea \( f \colon [a, b] \to \mathbb{R}^n \) una función de clase \( C^{p-1} \) en \( [a, b] \), y además \( p \) veces derivable en \( ]a, b[ \), tal que existe \( M > 0 \) con
\[
\abs{f^{(p)}(t)} \leq M \quad \forall t \in \; ]a, b[.
\]
Entonces, se cumple que
\[
f(b) = f(a) + (b-a)f'(a) + \frac{(b-a)^2}{2!} f''(a) + \cdots + \frac{(b-a)^{p-1}}{(p-1)!} f^{(p-1)}(a) + r_p,
\]
donde \( r_p \in \mathbb{R}^n \) satisface
\[
\|r_p\| \leq \frac{M(b-a)^p}{p!}.
\]

\pf{
	Sea la función auxiliar \( g \colon [a, b] \to \mathbb{R}^n \) definida por
	\[
	g(t) = f(t) + (b-t)f'(t) + \frac{(b-t)^2}{2!} f''(t) + \cdots + \frac{(b-t)^{p-1}}{(p-1)!} f^{(p-1)}(t).
	\]
	
	Como \( f \in C^{p-1}([a, b]) \) y \( f^{(p)} \) existe en \( ]a, b[ \), entonces \( g  \) es diferenciable en $]a, b[$.
	
	Calculemos la derivada de \( g \) en \( ]a, b[ \). Derivando término a término y aplicando la regla del producto:
	\begin{align*}
		g'(t) &= f'(t) + (-1)f'(t) + (b-t)f''(t) \\
		&\quad + (-1)(b-t)f''(t) + \frac{(b-t)^2}{2!}f^{(3)}(t) \\
		&\quad + \cdots + \frac{(b-t)^{p-2}}{(p-2)!}f^{(p-1)}(t) - \frac{(p-1)(b-t)^{p-2}}{(p-1)!}f^{(p-1)}(t) \\
		&\quad + \frac{(b-t)^{p-1}}{(p-1)!}f^{(p)}(t).
	\end{align*}
	
	Todos los términos intermedios se cancelan debido a las derivadas de los productos, quedando únicamente:
	\[
	g'(t) = \frac{(b-t)^{p-1}}{(p-1)!} f^{(p)}(t), \quad \forall t \in ]a, b[.
	\]
	
	Por lo tanto,
	\[
	\|g'(t)\| = \frac{(b-t)^{p-1}}{(p-1)!} \|f^{(p)}(t)\| \leq \frac{M(b-t)^{p-1}}{(p-1)!}.
	\] 
	
	Definamos, $\varphi \colon [a, b] \to \R$ por $$\varphi(t) = - \frac{M (b-t)^p}{(p)!}$$
	donde 
	$$
	\varphi'(t) = \frac{M (b-t)^{p-1}}{(p-1)!}>0, \quad \forall t \in ]a, b[ \quad \wedge \quad \abs{g'(t)} \leq g(t), \quad\forall t \in ]a, b[ 
	$$
	
	Del Lema 2
	
	\begin{align*}
		\abs{g(b)-g(a)} &\leq \varphi(b)-\varphi(a)\\
		\abs{g(b)-g(a)} &\leq \frac{M (b-t)^p}{(p)!} \\
		\abs{r_p} & \leq \frac{M (b-t)^p}{(p)!}
	\end{align*}
	
}

\begin{itemize}
	\item Sea $f \colon [a, b] \rightarrow \Rn$ una función.
	
	Dada una partición 
	$$
	P = \{t_0 =a< t_1< t_2<\ldots< t_k =b\}
	$$
	La longitud de $f$ asociada a $P$ es
	$$
	\ell(f, P) = \sum_{i=1}^{k} \abs{f(t_1), \ldots, f(t_{i-1})}
	$$
	\item Decimos que $f \colon [a, b] \rightarrow \Rn$ es \textbf{rectificable} si $\exists M>0$ tal que $\forall$ partición $P$ de $[a, b]$
	$$
	\ell(f, P) \leq M
	$$
	En este caso, definamos la longitud de $f \colon [a, b] \rightarrow \Rn$ por 
	\begin{align*}
		\ell(t) & = \sup \left\lbrace \ell(f, P) \in \R \colon P \quad \text{es una partición de} \quad [a, b] \right\rbrace \\
		&= \sup\limits_{P}  \ell(f, P)
	\end{align*}
	\ex{
		Sea $f \colon [a, b ]\to \Real{2}$
		
		
		}
\end{itemize}

