\chapter{Clase 12}
\clasedate{10 de mayo de 2025}

\thmrpf{Teorema del valor medio}{}{
	Sea $n\in\mathbb{N}$, con $n\geq 1$.
	Sea $f:[a,b]\to\mathbb{R}^n$ continua en $[a,b]$ y diferenciable
	en $]a,b[$. Si existe $M>0$ tal
	\[|f'(t)|\leq M,\;\forall t\in]a,b[,\]
	entonces \[|f(b)-f(a)|\leq M(b-a)\]
}{
	Si $|\cdot|$ es la norma euclidiana ya se probó.\\
	Si $\|\cdot\|$ es otra norma
	en $\mathbb{R}^n$, entonces
	existe $\alpha,\beta\in\mathbb{R}^+$ tal que
	\[\begin{cases}
		\forall w\in\mathbb{R}^n,\; \|w\|\leq \alpha|w|\\
		\forall w\in\mathbb{R}^n,\; |w|\leq \beta\|w\|
	\end{cases}\]
	
	Cuando $f$ es de clase $\mathcal{C}^1$ en $]a,b[$ (Se puede cambiar esto por
	$f'$ es integrable en cada
	subintervalo compacto $[c,d]\subset]a,b[$)\\
	Dado $[c,d]\subset]a,b[$, $f'$ es continua en $[c,d]$
	
	Del T.F. Cálculo
	\[\int_c^d f'(t)dt = f(d)-f(c)\]
	Por tanto, por desigualdad real
	\[|f(d)-f(c)| = \left|\int_c^d f'(t)dt\right| \leq \int_c^d |f'(t)|dt \leq \int_c^d M dt = M(d-c)\]
	ya que $\forall t\in]a,b[$, $|f'(t)|\leq M$
	
	$\forall$ subintervalo compacto $[c,d]\subset]a,b[$
	\begin{equation}
		|f(d)-f(c)|\leq M(d-c)
		\label{eq:class12-1}
	\end{equation}
	Considere $k_0\in\mathbb{N}$ tal $\frac{1}{k_0}<\frac{b-a}{2}$
	
	De \eqref{eq:class12-1}, $\forall k\in\mathbb{N}$, con $k\geq k_0$, $d_k = b-\frac{1}{k}, c_k = a+\frac{1}{k}$
	\[\abs{f\qty(b-\frac{1}{k})-f\qty(a+\frac{1}{k})}\leq M(d_k-c_k)\]
	Haciendo $k\to+\infty$, de la continuidad de $f:[a,b]\to\mathbb{R}^n$
	\[|f(b)-f(a)|\leq M(b-a)\]
}
\lemp{}{
	Sean $f:[c,d] \to\mathbb{R}$ y $\varphi \colon [a,b] \to \R$ continuas  y diferenciables $[a,b] $. Si $\abs{f'(t) \leq \varphi'(t)}$ y $\varphi(t)$ es estrictamente creciente, es decir, $\varphi'(t)>0$ para todo $t \in [a,b] $, entonces 
	$$
	\abs{f(b)-f(a) } \leq \varphi(b)-\varphi(c)
	$$
}{
	Sea  $[c, d] \subset ]a, b[$ subintervalo compacto fijo y arbitrario. 
	
	Mostremos que  $\abs{f(d)-f(c)} \leq \varphi(d)-\varphi(c)$ (basta con hacer $c \to a^{+}$ y $d \to b^{-}$) 
	
	Por contradicción, supongamos que 
	$$
	\abs{f(d)-f(c)} \geq \varphi(d)-\varphi(c)
	$$
	Elegimos $A \in \left] 1, \frac{\abs{f(d)-f(c)}}{\varphi(d)-\varphi(c)}\right[  $

$$
\rightarrow 1 < A < \frac{\abs{f(d)-f(c)}}{\varphi(d)-\varphi(c)}
$$
\begin{equation} 
\rightarrow A[ \varphi(d) -\varphi(c)] < \abs{f(d)-f(c)}
\label{eq:class12-2}
\end{equation}

Si 
\begin{align*}
	A\pqty{ \varphi\pqty{\frac{c+d}{2}} -\varphi(c)} &\geq \abs{ f\pqty{\frac{c+d}{2}} -f(c) }\\
	\text{ y }\quad A\pqty{ \varphi(d)- \varphi\pqty{\frac{c+d}{2}} } &\geq \abs{ f(d)-f\pqty{\frac{c+d}{2}}  }
\end{align*}

Sumando 

\begin{align*}
	A\pqty{  \varphi(d)  -\varphi(c)} &\geq \abs{ f\pqty{\frac{c+d}{2}} -f(c) } + \abs{ f(d)-f\pqty{\frac{c+d}{2}}}\\
 	&\geq \abs{ f\pqty{\frac{c+d}{2}} -f(c)+ f(d)-f\pqty{\frac{c+d}{2}}}\\
 	A\pqty{  \varphi(d)  -\varphi(c)} & \geq \abs{  f(d) -f(c)  } \qquad \text{Contradice a \eqref{eq:class12-2}}
\end{align*}

Entonces necesariamente,
\begin{equation}
A\cdot\qty[\varphi\qty(\frac{c+d}{2})-\varphi(c)] < \qty|f\qty(\frac{c+d}{2})-f(c)| \label{eq:class12-3}
\end{equation}
o,  
\begin{equation}
	A\cdot\qty[\varphi(d)-\varphi\qty(\frac{c+d}{2})] < \qty|f(d)-f\qty(\frac{c+d}{2})|
	\label{eq:class12-4}
\end{equation}


si ocurre \eqref{eq:class12-3}, elegimos $[c_1,d_1]:=\qty[c,\frac{c+d}{2}]$

y si ocurre \eqref{eq:class12-4}, elegimos $[c_1,d_1]:=\qty[\frac{c+d}{2},d]$

Tenemos que $A[\varphi(d_1)- \varphi(c_1)] < |f(d_1)-f(c_1)|$

Análogamente, se tienen dos posibilidades:

\begin{equation}
	A[\varphi(\frac{c_1+d_1}{2})-\varphi(c_1)] < |f(\frac{c_1+d_1}{2})-f(c_1)| 
		\label{eq:class12-5}
	\end{equation}
o
\begin{equation}
	A[\varphi(d_1)-\varphi(\frac{c_1+d_1}{2})] < |f(d_1)-f(\frac{c_1+d_1}{2})|
	\label{eq:class12-6}
\end{equation}
De este modo, continuando este argumento hemos construído una sucesión encajada de intervalos compactos
$\{[c_k,d_k]\}_{k\in\mathbb{N}}$ tal que:

$[c_k,d_k]\supset[c_{k+1},d_{k+1}]$ $\forall k\in\mathbb{N}\cup \{0\}$ con
$[c_0,d_0]\supset[c_1,d_1]\supset[c_2,d_2]\supset\cdots$ 

donde $\ell([c_k,d_k])=\frac{d-c}{2^k}$, $\forall k\in\mathbb{N}$
y $A[\varphi(d_k)-\varphi(c_k)]<|f(d_k)-f(c_k)|$

Así, $\bigcaplim{k\in\mathbb{N}}[c_k,d_k]=\{t_0\}\in[c,d] \subset ]a, b[$
$$
\therefore f \text{ y } \varphi \text{ son diferenciables en }   t_0\in]a,b[
$$

}


 \lem{Ejercicio}{
 	Sea $f : \mathbb{I} \to \mathbb{R}^n$ una función diferenciable en un punto $t_0 \in \interior{\mathbb{I}}$.  
 	Sean $\{c_k\}_{k\in\mathbb{N}}, \{d_k\}_{k\in\mathbb{N}} \subset \interior{\mathbb{I}}$ dos sucesiones tales que:
 	\begin{itemize}
 		\item $c_k < d_k$ para todo $k \in \mathbb{N}$,
 		\item $c_k \leq t_0 \leq d_k$ para todo $k \in \mathbb{N}$,
 		%\item $\lim_{k \to \infty} c_k = \lim_{k \to \infty} d_k = t_0$.
 	\end{itemize}
 	Entonces, se tiene que
 	\[
 	f'(t_0) = \lim_{k \to \infty} \frac{f(d_k) - f(c_k)}{d_k - c_k}.
 	\]
 }
 
 
	
	

	
	$$
	A\cdot\left(\frac{\varphi(d_k)-\varphi(c_k)}{d_k-c_k}\right)<\left|\frac{f(d_k)-f(c_k)}{d_k-c_k}\right|
	$$
	
	Haciendo $k\to\infty$, del Lema 1
	
	$$
	\varphi'(t_0)\leq A\cdot \varphi'(t_0)\leq|f'(t_0)|
	$$
	
	$\exists t_0\in[c,d]\subset]a,b[$ tal que $\varphi'(t_0)<|f'(t_0)|$
	
	Esto contradice la hipótesis!
	
	Por ende, $|f(d)-f(c)|\leq\varphi(d)-\varphi(c)$
	
	 
	\corrp{}{Teorema}{
		Sea $f \colon [a, b] \to \mathbb{R}^n$ una función continua en $[a,b]$ y diferenciable en el intervalo abierto $]a, b[$, tal que
		\[
		f'(t) = \Ou \in \mathbb{R}^n \quad \text{para todo } t \in ]a, b[.
		\]
		Entonces, $f$ es constante en $[a, b]$.
	}{
		Dado que $f'(t) = \Ou$ para todo $t \in ]a, b[$, se tiene:
		\[
		\|f'(t)\| = 0 \quad \text{para todo } t \in ]a, b[.
		\]
		Sea $[c,d] \subset ]a, b[$ un subintervalo compacto arbitrario. Como $f$ es continua en $[c,d]$ y derivable en $]c,d[$, podemos aplicar el teorema del valor medio (T.V.M.) :
		\[
		\exists \, \xi \in ]c,d[ \text{ tal que } f(d) - f(c) = f'(\xi)(d - c).
		\]
		Pero como $f'(\xi) = \mathbf{0}$, se deduce que
		\[
		f(d) - f(c) = \Ou \quad \Rightarrow \quad f(d) = f(c).
		\]
		Como esto vale para todo par de puntos $c, d \in ]a, b[$, se concluye que $f$ es constante en $]a, b[$.  
		Finalmente, como $f$ es continua en $[a,b]$, esta constancia se extiende al intervalo cerrado.
		
		\[
		\therefore \quad f \text{ es constante en } [a, b].
		\]
	}
	
	\section{Teorema Fundamental del Cálculo}
	
\thmrpf{Teorema Fundamental del Cálculo}{}{
\begin{align*}
	\text{Sea} \quad f: [a,b] \subset \mathbb{R} &\longrightarrow \mathbb{R}^n \text{ de clase }  C^1 \text{ (Se puede cambiar esto por } f' \text{ integrable)}\\
	t &\longmapsto f(t) = (f_1(t), \ldots, f_n(t))
\end{align*}
Entonces
\[\int_a^b f(t) dt = f(b) - f(a)\]
}{
	  Como $ f = (f_1, f_2, \ldots, f_n)$  es de clase $C^1$
	
	$\Rightarrow$
	$f$ es diferencial de $[a,b]$ y $f' = (f_1', f_2', \ldots, f_n)$ es continua
	
	$\Rightarrow$
	$f_1, f_2, \ldots, f_n$ es diferenciable.
	$[a,b]$ y $f_1', f_2', \ldots, f_n$ son continuas.
	
	$\Rightarrow$
	$f_1, f_2, \ldots, f_n : [a,b] \rightarrow \mathbb{R}$ son de clase $C^1$. ($f_i^{'}$ es continua $\forall i \in \{1, \ldots, n\}$ )
	
	Del T.F. Cálculo para funciones reales de variable
	real:
	\[\int_a^b f_i^{'}(t) dt = f_i(b) - f_i(a), \quad \forall i \in \{1, 2, \ldots, n\}\]
	
	Además, $f'$ es continua, $D_{f'} = \varnothing, m(D_{f'}) = 0$
	\begin{align*}
		\int_a^b f'(t) dt &= \pqty{ \int_a^b f_1^{'}(t) dt , \cdots , \int_a^b f_n^{'}(t) dt}\\
		&=\pqty{f_1(b) - f_1(a) , \cdots , f_n(b) - f_n(a)}\\
		& = f(b)-f(a)
	\end{align*}
	
}

\propp{
	Sea $f \colon [a,b]   \longrightarrow \mathbb{R}^n$ integrable. Entonces
\begin{align*}
	 \abs{f}\colon [a,b] & \longrightarrow \mathbb{R}^n \text{ es integrable } \\
	t &\longmapsto \abs{f}(t) = \abs{f(t)}=  \abs{\cdot} \circ f
\end{align*}
y además, 
$$
\abs{\int_{0}^{b} f(t) \dd{t}} \leq \int_{a}^{b} \abs{f(t)} \dd{t}
$$
\rmk{
Si $f$ es continua en $t_0 \rightarrow \abs{f}$ es continua en $t_0$. 
}
}{
$\forall x,y \in [a, b]$,
\begin{equation}
	\abs{\abs{f(x)}-\abs{f(y)}} \leq \abs{f(x)-f(y)}
	\label{eq:class12-7}
\end{equation}
\begin{align*}
	\abs{f(x)} &= \abs{f(x)-f(y)+f(y)}\\
	& \leq \abs{f(x)-f(y)}+\abs{f(y)}
\end{align*}
Si $f$ es continua en $x_0 \in  [a, b] \rightarrow \abs{f}$ es continua en $x_0 \in [a, b]$
$$
\forall \varepsilon>0, \exists \delta >0, \forall x \in [a, b], \abs{x-x_0} < \delta \rightarrow \abs{f(x)-f(x_0)}< \varepsilon
$$
De \eqref{eq:class12-7}
$$
\abs{\abs{f(x)}- \abs{f(x_0)}} \leq \abs{f(x)-f(x_0)}< \varepsilon
$$
$\rightarrow \abs{f}$ es continua en $x_0$

Si $f$ es continua en $x_0 \in  [a, b]$ equivale  a $D_{\abs{f}} \subset D_{f} = \{x \in [a, b]\colon f \text{ no es continua en } x \}$\\
$D_{\abs{f}}= \{x \in [a, b]\colon \abs{f} \text{ no es continua en } x \} $

Como $f$ es integrable $\stackrel{Lebesgue}{=} m\pqty{D_f} = 0$\\
Pero
 $$0 \leq m\pqty{D_{\abs{f}}} \leq m\pqty{D_{f}}$$
$$m\pqty{D_{\abs{f}}} = 0\stackrel{Lebesgue}{=} \abs{f} \quad \text{es integrable. }$$ 
}

Como
\begin{equation*}
	\int_a^b f = \int_a^b f(t) \dd{t} \in \mathbb{R}^n,
\end{equation*}
dado que \( f \) es integrable en \( [a, b] \), se tiene que:

Para toda partición puntillada \( P^* = (P, \xi) \), es decir, toda partición \( P = \{t_0 = a < t_1 < \dots < t_k = b\} \) del intervalo \([a,b]\), y toda puntillación \( \xi = (\xi_i)_{i=1}^k \) con \( \xi_i \in [t_{i-1}, t_i] \), se cumple que:

\begin{equation}
	\forall \varepsilon > 0,\ \exists \delta > 0\ \text{tal que si } \abs{P} < \delta,\ \text{entonces } \abs*{ \sum_{i=1}^k (t_i - t_{i-1}) f(\xi_i) - \int_a^b f } < \varepsilon.
	\label{eq:class12-8}
\end{equation}

Análogamente, dado que \( \abs{f} \) es integrable en \([a,b]\), es decir:
\begin{equation*}
	\int_a^b \abs{f} = \int_a^b \abs{f(t)} \dd{t}, 
\end{equation*}
entonces también se cumple que:
\begin{equation}
	\forall \varepsilon > 0,\ \exists \delta > 0\ \text{tal que si } \abs{P} < \delta,\ \text{entonces } \abs*{ \sum_{i=1}^k (t_i - t_{i-1}) \abs{f}(\xi_i) - \int_a^b \abs{f} } < \varepsilon.
	\label{eq:class12-9}
\end{equation}

Además, para cualquier partición y puntillación, por desigualdad del valor absoluto en \( \mathbb{R}^n \), se cumple:
\begin{equation}
	\abs*{ \sum_{i=1}^k (t_i - t_{i-1}) f(\xi_i) } \leq \sum_{i=1}^k (t_i - t_{i-1}) \abs{f(\xi_i)}.
	\label{eq:class12-10}
\end{equation}

Ahora, sea \( \varepsilon > 0 \). Por \eqref{eq:class12-8} y \eqref{eq:class12-9}, existe \( \delta > 0 \) tal que si \( \abs{P} < \delta \), entonces:
\begin{align}
	\abs*{ \sum_{i=1}^k (t_i - t_{i-1}) f(\xi_i) - \int_a^b f } &< \frac{\varepsilon}{2}, \label{eq:class12-11} \\
	\abs*{ \sum_{i=1}^k (t_i - t_{i-1}) \abs{f}(\xi_i) - \int_a^b \abs{f} } &< \frac{\varepsilon}{2}. \label{eq:class12-12}
\end{align}

Luego, usando la desigualdad \eqref{eq:class12-10}, se tiene:
\[
\abs*{ \sum_{i=1}^k (t_i - t_{i-1}) f(\xi_i) }
\leq \sum_{i=1}^k (t_i - t_{i-1}) \abs{f(\xi_i)}.
\]

Aplicando \eqref{eq:class12-11} y \eqref{eq:class12-12}, se obtiene:
\begin{align*}
	\abs*{ \int_a^b f }
	&\leq \abs*{ \sum_{i=1}^k (t_i - t_{i-1}) f(\xi_i) } + \abs*{ \int_a^b f - \sum_{i=1}^k (t_i - t_{i-1}) f(\xi_i) } \\
	&< \sum_{i=1}^k (t_i - t_{i-1}) \abs{f(\xi_i)} + \frac{\varepsilon}{2} \\
	&< \abs*{ \int_a^b \abs{f} } + \frac{\varepsilon}{2} + \frac{\varepsilon}{2} = \int_a^b \abs{f} + \varepsilon.
\end{align*}

Como \( \varepsilon > 0 \) fue arbitrario, se concluye que:
\begin{equation*}
	\abs*{ \int_a^b f } \leq \int_a^b \abs{f}.
	\label{eq:final-ineq}
\end{equation*}

\qed





