\chapter{Clase 3}
\clasedate{07 de abril de 2025}
\section{Unicidad en sucesiones acotadas}

Dado $\varepsilon > 0$, $\exists\, k_0 \in \mathbb{N}$,
tal que $k \geq k_0 \Rightarrow \|x_k - a\| < \varepsilon$

\begin{figure}[H]
	\centering
	\includegraphics[width=0.45\linewidth]{img/class3_figure1.png}
\end{figure}
\thmr{}{teorema3-1}{


	Sea $(x_k)_{k\in \N} \subset \Rn$ una sucesión acotada. Entonces, $(x_k)_{k\in \N}$ es convergente si y sólo si toda subsucesión convergente de $(x_k)_{k\in \N}$ converge al mismo punto de $\Rn$.

}
\pf{Prueba del teorema \ref{thm:teorema3-1}

$(\Rightarrow)$  Supongamos que $\lim\limits_{k \to \infty} x_k = a \in \Rn$. Sea $(x_{i_k})_{k \in \N} \subset (x_k)_{k \in \N}$ una subsucesión.

\begin{equation}
	\forall \varepsilon > 0,\, \exists\, k_0 \in \N:\, \forall k \in \N,\; k \geq k_0 \Rightarrow \|x_k - a\| < \varepsilon    \label{eq:class3-1}
\end{equation}

Como \((i_k)_{k \in \N} \subset \N\) es estrictamente creciente \(\Rightarrow i_k \geq k,\, \forall k \in \N\).

Entonces, \(k \geq k_0 \Rightarrow i_k \geq k \geq k_0 \Rightarrow \|x_{i_k} - a\| < \varepsilon\) por \eqref{eq:class3-1}.

\(\therefore \lim\limits_{k \to \infty} x_{i_k} = a\).


($\Leftarrow$) Supongamos que toda subsucesión de $(x_k)_{k\in \N}$ que converge lo hace al mismo punto.

Sea $A = \qty{a \in \Rn, \exists \, (x_{i_k})_{k\in \N} \subset (x_k)_{k\in \N} : \lim\limits_{k \to \infty } x_{i_k}=a}$ (Conjunto de valores de adherencia de la sucesión $(x_k)_{k\in\N}$)

Como $(x_k)_{k\in \N}$ es acotada, del teorema de \textbf{Bolzano-Weierstrass}, $\exists (x_{i_k})_k \subset (x_k)_k$ tal que $\lim\limits_{k \to \infty } x_{i_k}=a \in \Rn$. Así $ a\in A, A \neq  \varnothing $

Queremos mostrar que $\lim\lim_{k \to \infty} x_k=a$

Por reducción al absurdo, supongamos que \underline{no ocurra} $\lim\limits_{k \to \infty} x_k=a$

$$
	\sim( \forall \varepsilon > 0,\, \exists\, k_0 \in \N:\, \forall k \in \N,\; k \geq k_0 \Rightarrow \|x_k - a\| < \varepsilon )
$$
$$
	\exists\, \varepsilon > 0,\, \forall\, k_0 \in \N:\, \exists\, k \in \N,\; k \geq k_0 \wedge \|x_k - a\| \geq \varepsilon
$$

Podemos construir $(x_{j_k})_{k\in\N} \subset (x_k)_{k\in \N}$ tal que
\begin{equation}
	\abs{x_{i_j}-a}\geq \varepsilon_0 \quad\forall k\in \N \label{eq:class3-eq2}
\end{equation}

Por el teorema de \textbf{Bolzano-Weierstrass}, \(\exists\, \pqty{x_{j_{p_k}}}_{k \in \N} \subset (x_{j_k})_{k \in \N}\) tal que
\[
	\lim\limits_{k \to \infty} x_{j_{p_k}} = b \in \Rn
\]
De \eqref{eq:class3-eq2} $\abs{ x_{j_{p_k}} -a} \geq \varepsilon_0, \quad \forall k \in \N$

Así, $\abs{b-a} = \lim\limits_{k \to \infty}\abs{ x_{j_{p_k}}-a} = \varepsilon_0 >0 \rightarrow b\neq a$
}

\section{Puntos de acumulación}

\begin{itemize}
	\item Sea \( X \subset \mathbb{R}^n \). Decimos que \( a \in \mathbb{R}^n \) es \underline{punto de acumulación} de \( X \), \( a \in X' \), si:
	      \[
		      \forall \varepsilon > 0, \quad \left( B(a, \varepsilon) \setminus \{a\} \right) \cap X \neq \emptyset
	      \]
	      \ex{
		      Si \( B(a, \varepsilon) = X \), entonces
		      \[
			      \forall \delta > 0, \quad \left( B(b, \delta) \setminus \{b\} \right) \cap X \neq \emptyset
		      \]
		      \vspace{-2em}
		      \begin{figure}[H]
			      \centering
			      \includegraphics[width=0.25\linewidth]{img/class3_figure2.png}
		      \end{figure}
		      \vspace{-3em}
		      \[
			      \Rightarrow b \in X'
		      \]
	      }

	      \ex{
		      Mostrar que si \( X = B(a, \varepsilon) = \{ z \in \mathbb{R}^n : |z - a| < \varepsilon \} \), entonces:
		      \[
			      X' = B[a, \varepsilon] = \{ z \in \mathbb{R}^n : |z - a| \leq \varepsilon \}
		      \]
	      }
	\item Sea \( a \in X \). Decimos que \( a \in X \) es un punto aislado de \( X \) si \( a \notin X' \). Esto es,
	      \[
		      \exists \varepsilon_0 >0, (B(a, \varepsilon_0) \setminus \{a\}) \cap X = \varnothing
	      \]

	      \vspace{-2em}
	      \begin{figure}[H]
		      \centering
		      \includegraphics[width=0.25\linewidth]{img/class3_figure3.png}
	      \end{figure}
	      \vspace{-1em}
\end{itemize}


\thmr{}{teorema3-1}{
	Sea $X\subset \Rn$ son equivalentes:
	\begin{description}
		\item[(i)] $a \in X'$
		\item[ (ii)] $\exists (x_k)_{k\in \N} \subset X\setminus\{a\}$, es decir $x_k \neq a \, \forall k\in \N$, tal que $\lim\limits_{k \to \infty} x_k =a$
	\end{description}
}
\pf{ Prueba del teorema \ref{thm:teorema3-1}\\

	\begin{description}
		\item[(ii) $\Rightarrow$ (i)]  Supongamos que $\exists (x_k)_{k \in \N} \subset X\setminus \{a\}$ tal que $\lim\limits_{k \to \infty} x_k =a$

		      Dado $\varepsilon>0, \, \exists \, k_0\in \N, \forall k\in \N$
		      \begin{align*}
			      k \geq k_0 & \rightarrow \abs{x_k-a}<\varepsilon                                                \\
			      k \geq k_0 & \rightarrow  x_k \in \pqty{B(a,\varepsilon)\setminus\{a\}} \cap X \neq \varnothing
		      \end{align*}
		      Por tanto $a \in X'$
		\item[(i) $\Rightarrow$ (ii)] Sea $a \in X'$
		      $$
			      \forall \varepsilon>0,  \pqty{B(a,\varepsilon)\setminus\{a\}} \cap X \neq \varnothing
		      $$
		      \begin{itemize}
			      \item para $\varepsilon= 1, \exists x_1 \in  \pqty{B(a,1)\setminus\{a\}} \cap X \neq \varnothing $

			            $$
				            \exists x_1 \in X \text{ tal que } 0<\abs{x_1-a}<1
			            $$
			            Por el \underline{Principio  Arquimediano}, $\exists i_2 \in \N$ tal que $\frac{1}{i_2} < \abs{x_1-a}$
			      \item para $\varepsilon= \frac{1}{i_2} , \exists x_{i_2} \in  \pqty{B\qty(a,\frac{1}{i_2})\setminus\{a\}} \cap X \neq \varnothing $

			            $$
				            \exists x_{i_2} \in X \text{ tal que } 0<\abs{x_{i_2}-a}<\frac{1}{i_2} \text{ con } i_2>1=i_1
			            $$
			            \textbf{\textcolor{red}{$\ldots$ Tarea-hacerlo inductivo  }}

		      \end{itemize}
		      De este modo hemos construido $(x_{i_k}) \subset X\setminus \{a\}$ tal que $\abs{x_{i_k}-a} \frac{1}{i_k} \, \forall k \in \N$ donde $(i_k)_{k\in \N} \subset \N$ estrictamente creciente $i_1<i_2<i_3<\ldots$.

		      Por lo tanto $\lim\limits_{k \to \infty} \abs{x_{i_k}-a} = 0 \longrightarrow \lim\limits_{k \to \infty} x_{i_k} =a$
	\end{description}
}

\section{Funciones continuas}

Sean $X \in \Rm$ y $f : X \rightarrow \Rn$,
\begin{itemize}
	\item  Decimos que $f$ es \underline{continua} en $a\in X$ si
	      \begin{align*}
		      \forall \varepsilon >0, \exists \delta >0, \forall x \in X, \abs{x-a}<\delta & \longrightarrow \abs{f(x)-f(a)}<\varepsilon       \\
		      \forall x \in X, x\in B(a, \delta)                                           & \longrightarrow f(x) \in B(f(a), \varepsilon)     \\
		      \text{si }\quad x \in X \cap B(a, \delta)                                    & \longrightarrow f(x) \in B(f(a), \varepsilon)     \\
		                                                                                   & \rightarrow x \in f^{-1}\qty(B(f(a),\varepsilon)) \\
		      f(X\cap B(a, \delta ))                                                       & \subset B(f(a), \varepsilon)                      \\
		      X\cap B(a, \delta )                                                          & \subset f^{-1}\qty(B(f(a), \varepsilon))
	      \end{align*}
	      $\delta$ depende de $\varepsilon$ y de $a \in X$, es decir $\delta=\delta(\varepsilon, a)$
	\item $f$ es discontinua (no continua) en $a\in X$ si
	      $$
		      \forall \varepsilon_0 >0, \forall \delta>0, \exists x \in X, \abs{x-a}<\varepsilon \wedge \abs{f(x)-f(a)} \geq \varepsilon_0
	      $$
	\item $f \colon X \to \Rn$ es continua, significa que $f$ es continua en todo punto de $X$.

	\item Sea \( f \colon X \to \mathbb{R}^n \) una función cualquiera. Si $a \in X$ y $a \in X'$, es decir,  si \( a \in X \) es un punto aislado de \( X \), entonces \( f \) es continua en \( a \).

	      \begin{align*}
		      \exists \delta_0>0 \text{ tal que } & \pqty{B(a, \delta_0) \setminus\{a\}}\cap X = \varnothing \\
		      \to                                 & B(a, \delta_0) \cap X =\{a\}
	      \end{align*}
	      $$
		      \text{Dado } \varepsilon>0, \exists \delta_0>0, \in A  \forall x \in X, \abs{x-a} < \delta_0 \rightarrow \abs{f(x)-f(a)}<\varepsilon
	      $$
	      Si $x=a,$ entonces, $ 0<\delta_0 \rightarrow \abs{f(a)-f(a)}=0 < \varepsilon$

	      Por lo tanto, se cumple la condición de continuidad en \( a \):
	      \[
		      \forall \varepsilon > 0,\ \exists \delta > 0\ \text{tal que}\ \forall x \in X,\ \abs{x - a} < \delta \Rightarrow \abs{f(x) - f(a)} < \varepsilon.
	      \]
	      \[
		      \therefore\ f \text{ es continua en } a\in X.
	      \]

	      \ex{
		      Sea $A \colon \Rm \to \Rn $ una transformación lineal, tenemos que A es continua, pues
		      \begin{align*}
			      \forall x = (x_1, \ldots, x_m) \in \Rm, \abs{Ax} & =\abs{A\qty(\sum_{i=1}^{m} x_i e_i)}                                                                       \\
			                                                       & = \abs{\sum_{i=1}^{m} x_i Ae_i}                                                                            \\
			                                                       & \leq \sum_{i=1}^{m} \abs{x_i} \underbrace{\abs{A e_i}}_{\leq c = \max\limits_{1\leq i \leq m} \abs{A e_i}} \\
			                                                       & \leq c \cdot  \sum_{i=1}^{m} \abs{x_i}                                                                     \\
			                                                       & = c \norm{x}_1
		      \end{align*}
		      $	\forall x \in \Rm, \abs{Ax} \leq c \norm{x}$\\
		      En particular, $\forall x, w \in \Rm, \abs{Ax-Aw}\leq c \norm{x-w}_1$, \underline{A es LIPSCHITZIANA}

	      }
	      \rmkb{
		      Sea \( A \colon \mathbb{R}^m \to \mathbb{R}^n \) una aplicación lineal. Decimos que \( A \) es \textbf{Lipschitz} (o que satisface una condición de Lipschitz) si existe una constante \( c > 0 \) tal que
		      \[
			      \forall x \in \mathbb{R}^m,\quad \norm{Ax} \leq c \norm{x}.
		      \]
		      En particular, dado que \( A \) es lineal, para cualesquiera \( x, w \in \mathbb{R}^m \), se cumple:
		      \[
			      \norm{Ax - Aw} = \norm{A(x - w)} \leq c \norm{x - w}.
		      \]
		      Esto significa que \( A \) es una función Lipschitz con constante de Lipschitz \( c \).
	      }

	\item $f: X \subset \Rm \to \Rn$, $f$ es continua en $X$. Decimos que \( f \) es \textbf{uniformemente continua} si:
	      $$
		      \forall \varepsilon > 0,\ \exists \delta > 0\ \text{tal que}\ \forall x, w \in X,\ \abs{x - w} < \delta \Rightarrow \abs{f(x) - f(w)} < \varepsilon
	      $$
	      En este caso, \(\delta\) depende únicamente de \(\varepsilon\), y no de los puntos \(x, w\).
\end{itemize}

\ex{Toda función $f: X \rightarrow \Rn$ lipschitziana es uniformemente continua.\\
	$$
		\exists c>0, \text{ tal que } \forall x, w \in X, \abs{f(x)-f(w)} \leq c \abs{x-w}
	$$
	Dado $\varepsilon >0$, existe $\delta = \frac{\varepsilon}{c}>0, \forall x, w \in X$
	$$
		\abs{x-w} <\delta = \frac{\varepsilon}{c} \rightarrow \abs{f(x)-f(w)} \leq c \abs{x-w} < \varepsilon
	$$
	$$
		\therefore f \text{ es uniformemente continua}
	$$
}

En particular las proyecciones canónicas
\begin{align*}
	\pi_i \colon \Rn          & \longrightarrow \R          \\
	x=(x_1, x_2, \ldots, x_n) & \longmapsto \pi_i (x) = x_i
\end{align*}
$\pi_i$ es una transformación lineal\\
$\therefore $ uniformemente continua.

\ex{Ejercicio\\
	Mostrar que si $\varphi \colon \Rm \times \Rn \to \Rp$ bilineal, entonces $\varphi$ es continua.
}

\thmr{}{teorema3-2}{
	Sean \( X \subset \mathbb{R}^m \), \( Y \subset \mathbb{R}^n \), y funciones \( f \colon X \to \mathbb{R}^n \), \( g \colon Y \to \mathbb{R}^p \) tales que \( f(X) \subset Y \), es decir, la composición \( g \circ f \colon X \to \mathbb{R}^p \) está bien definida.

	Si \( f \) es continua en \( x_0 \in X \) y \( g \) es continua en \( f(x_0) \in Y \), entonces \( g \circ f \) es continua en \( x_0 \).

	En particular, si \( f \) es continua en todo \( X \) y \( g \) es continua en todo \( Y \), entonces \( g \circ f \colon X \to \mathbb{R}^p \) es continua.
}

\pf{ Prueba del teorema \ref{thm:teorema3-2}\\
	Sea \( f \colon X \to Y \subset \Rn \) y \( g \colon Y \to \Rp \). Supongamos que \( f \) es continua en \( x_0 \in X \), y que \( g \) es continua en \( f(x_0) \in Y \).
	Queremos probar que \( g \circ f \colon X \to \Rp \) es continua en \( x_0 \).

	Como \( g \) es continua en \( f(x_0) \), dado \( \varepsilon > 0 \), existe \( n > 0 \) tal que
	\begin{equation} \label{eq:class3-eq3}
		\forall y \in Y, \quad \abs{y - f(x_0)} < n \ \Rightarrow \ \abs{g(y) - g(f(x_0))} < \varepsilon.
	\end{equation}

	Como \( f \) es continua en \( x_0 \), existe \( \delta > 0 \) tal que
	\begin{equation} \label{eq:class3-eq4}
		\forall x \in X, \quad \abs{x - x_0} < \delta \ \Rightarrow \ \abs{f(x) - f(x_0)} < n.
	\end{equation}

	Dado que \( f(x) \in f(X) \subset Y \), se cumple \( f(x) \in Y \), por lo que al aplicar \eqref{eq:class3-eq3} se obtiene:
	\[
		\forall x \in X, \quad \abs{x - x_0} < \delta \ \Rightarrow \ \abs{g(f(x)) - g(f(x_0))} < \varepsilon.
	\]

	Es decir,
	\[
		\abs{(g \circ f)(x) - (g \circ f)(x_0)} < \varepsilon.
	\]

	Por lo tanto, \( g \circ f \) es continua en \( x_0 \in X \).



	\rmkb{Sea $f(x) \in \Rn$ y
		\begin{align*}
			f\colon X \subset \Rm & \longrightarrow \Rn                                 \\
			x                     & \longmapsto f(x) = (f_1(x), f_2(x), \ldots, f_n(x))
		\end{align*}
		decimos que $f1, f2, \ldots, f_n \colon \R $ son las \textcolor{red}{funciones coordenadas de $f$.}\\
		Notación: $f=(f_1, f_2, \ldots, f_n)$\\
		Es claro que $f_i = \pi_i \circ f, \forall i \in \{1, 2, \ldots, n\}$
	}
}

