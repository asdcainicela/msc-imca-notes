\chapter{Clase 8}
\clasedate{23 de abril de 2025}

\corrp{}{class8-coro1}{
	Sea \( K \subset \mathbb{R}^m \) compacto y \( f \colon K \to \mathbb{R}^n \) continua.\\
	Entonces, para todo subconjunto cerrado \( F \subset K \), la imagen \( f(F) \) es cerrada en \( \mathbb{R}^n \).
}{
	Sea \( F \subset K \) un conjunto cerrado. Como \( K \) es compacto, entonces \( F \), al ser un subconjunto cerrado de un compacto, también es compacto.
	
	Dado que \( f \) es continua y \( F \) es compacto, la imagen \( f(F) \) es compacta en \( \mathbb{R}^n \).
	
	Pero en \( \mathbb{R}^n \), todo conjunto compacto es cerrado. Por tanto, \( f(F) \) es cerrado.
} 

\corrp{}{}{
Sea \( f \colon K \subset \mathbb{R}^m \to L \subset \mathbb{R}^n \) una función continua y biyectiva, con \( K \) compacto y \( L = f(K) \).\\
Entonces \( f \) es un homeomorfismo entre \( K \) y \( L \).
}{
Como $f$ es biyección $\exists \colon f^{-1} \colon L \rightarrow K$.

Sea $F \subset K$ cerrado $\rightarrow (f^{-1})^{-1} (F) = f(F)$ es cerrado en $L$ por Corollary~\ref{cor:class8-coro1}. 

$$
\therefore f^{-1} \text{ es continua}
$$

Por tanto, \( f \) es un homeomorfismo entre \( K \) y \( L \).
}

\corp{}{
Sea $\varphi \colon K \subset \Rm \rightarrow L \subset \Rn$ continua, donde $K$ es compacto  y además $\varphi(K) = L$\\
Sea $f \colon L \rightarrow \Rp$. Tenemos que $f$ es continua si, y solo sí $f \circ \varphi \colon K \rightarrow \Rp$ continua.
}{
($\Rightarrow$)\quad Composición de continuas es continua.

\noindent ($\Leftarrow$)\quad 	Supongamos que $f \circ \varphi \colon K\rightarrow \Rp$ es continua.\\

Sea $F \subset \Rp$ cerrado. Mostremos que $f^{-1}(F)$ es cerrado en $L$.

Como $f\circ \varphi \colon K \subset \Rm \rightarrow \Rp$ es continua.
$$
\rightarrow (f \circ \varphi)^{-1} (F) \text{ es cerrado en } K
$$

Del corolario~\ref{cor:class8-coro1},  $\varphi$ lleva cerrados en cerrados.
\begin{align*}
	f^{-1}(F) &= \varphi\qty((f \circ\varphi)^{-1}(F)) \text{ es cerrado}\\
	&= \varphi\pqty{\varphi^{-1 (f^{-1}(F))}} \text{  Ejercicio}
\end{align*}
Como $\varphi \colon K \rightarrow L$ es sobreyectiva y $R \subset L$.
$$
\varphi\pqty{\varphi^{-1}(R)}=R
$$
}

\thmrpf{}{}{
Sea $f \colon K \subset \Rm \rightarrow\Rn$ continua definida en $K \subset \Rm$ compacto. Entonces $f$ es uniformemente continua.
}{

Supongamos que $f$ no es uniformemente continua,
$$
\sim \bqty{ \forall \varepsilon>0, \exists \delta>0, \forall x, y \in K, \abs{x-y}< \delta \rightarrow \abs{f(x)-f(y)}< \varepsilon }
$$
$$
\exists \varepsilon_0 >0, \forall \delta>0, \exists x_{\delta}, y_{\delta} \in K, \abs{x_{\delta}-y_{\delta}}<\delta \wedge \abs{f(x_{\delta}) -f(y_{\delta})} \geq \varepsilon
$$

$\forall k \in \N$, sea $\delta_k = \frac{1}{k}$
$$
\forall k \in \N, \exists x_k, y_k \in K, \abs{x_k-y_k}< \frac{1}{k} \wedge \abs{f(x_k)-f(y_k)} \geq \varepsilon_0
$$

Así $(x_k)_k \subset K$, $(y_k)_k \subset K$

$\exists j \colon \N \rightarrow \N$ estrictamente creciente tal que $\lim\limits_{x \to \infty} x_{j(k)} =x_0 \in K$

De (*), $\lim\limits_{ k \to \infty} y_{j(k)} =x_0 \in K$

Como $f$ es continua

$$
f(x_p) = \lim\limits_{k \to \infty} f(x_{j(k)}) = \lim\limits_{k \to \infty} f(y_{j(k)})
$$
esto contradice a $\abs{expression}$


}



\thmrpf{}{}{
	
}{
	
}



\thmrpf{}{}{

}{

}

\section{Distancia entre conjuntos}
Sean $X, Y \subset  \Rn $ no vacíos, la distancia entre $X \text{ e } Y$ está acotado inferiormente,
\begin{equation}
	d(X,Y)= \inf \qty{\abs{x-y } \in \R : x \in X, y \in Y}
\end{equation}

\rmkb{
	Si $X \subset  M \text{ e } Y \subset M$, entonces
	$$
	d(X,Y) \geq d(M,N)
	$$
	\pf{
	\begin{align*}
		\alpha\pqty{X, Y} & = \qty{ \abs{x-y} \colon x \in X, y \in Y}\\
		\alpha(M, N) & = \qty{ \abs{x-y} \colon x \in M, y \in N}
	\end{align*}
	Se tiene que $\alpha(X, Y ) \subset \alpha(M,N)$ 
	\begin{align*} 
		\rightarrow \inf(\alpha\pqty{X, Y}) &\geq \inf\qty(\alpha\pqty{M, N})\\
	  d\pqty{X, Y} &\geq d\pqty{M, N}
	\end{align*}
	}	
}

\thmrpf{}{}{
	Sean \( X, Y \subset \mathbb{R}^n \) conjuntos no vacíos. Entonces:
\[
d(X, Y) = d(\overline{X}, \overline{Y}),
\]
donde \( d(A, B) := \inf\{ \|a - b\| : a \in A, b \in B \} \) es la distancia entre conjuntos.
}{
\textbf{Primero:} Como \( X \subset \overline{X} \) y \( Y \subset \overline{Y} \), se tiene:
\[
d(\overline{X}, \overline{Y}) \leq d(X, Y).
\]



Mostremos que 
$$
d\pqty{\overline{X}, \overline{Y}} \geq d\pqty{X, Y} .
$$
Supongamos que $d\pqty{\overline{X}, \overline{Y}} < d\pqty{X, Y}  $,  además $d\pqty{X, Y}  $ no es cota inferior de $ \alpha\pqty{\overline{X}, \overline{Y}}$\\
Si $p \in \overline{X}, q \in \overline{Y}$
$$
\inf\{p-q\colon p \in \overline{X}, q \in \overline{Y}\} \quad \text{ mayor cota superior}
$$
$d(X,Y) \rightarrow \exists (y_k)_k \subset Y \text{ tal que } y_k \rightarrow y_0$\\
$\exists x_0 \in \overline{X}, \exists y_0 \in \overline{Y}$ tales que $\abs{x_0-y_0} < d(X,Y)$\\
$\exists (x_k)_k \subset X$ tal que $x_k \rightarrow x_0$,  $\exists (y_k)_k \subset Y$ tal que $y_k \rightarrow y_0$
$$
\abs{\abs{z}-\abs{w}} \geq \abs{z-w}, \varphi(z) = \abs{z}
$$ 
\begin{align*}
	\lim\limits_{k \to \infty} \abs{x_k-y_k} & = \lim\limits_{k \to \infty} \varphi \pqty{{x_k-y_k}} \\
	 & = \varphi\pqty{ \lim\limits_{k \to \infty} \pqty{x_k-y_k}}  \\
	  & =\varphi(x_0-y_0)\\
	  &=   \abs{x_0-y_0}< d(X,Y)
\end{align*}
$\exists k_0 \in \N, x_{k_0} \subset X, y_{k_0} \subset Y$ tal que,
 
\begin{align*}
	\abs{x_{k_0} - y_{k_0}} &< d(X, Y)\\
	&< \inf\{\abs{x-y} \colon x \in X \wedge y \in Y\} \qquad (\rightarrow \leftarrow)
\end{align*}
Entonces $d\pqty{\overline{X}, \overline{Y}} \geq d\pqty{X, Y} $

%------------------------------


\vspace{1ex}
\textbf{Veamos la desigualdad opuesta:} Supongamos, en busca de contradicción, que:
\[
d(\overline{X}, \overline{Y}) < d(X, Y).
\]
Entonces existen puntos \( x_0 \in \overline{X} \), \( y_0 \in \overline{Y} \) tales que:
\[
\|x_0 - y_0\| < d(X, Y).
\]

Como \( x_0 \in \overline{X} \), existe una sucesión \( (x_k) \subset X \) tal que \( x_k \to x_0 \).\\
Análogamente, existe una sucesión \( (y_k) \subset Y \) tal que \( y_k \to y_0 \).

Entonces:
\[
\lim_{k \to \infty} \|x_k - y_k\| = \|x_0 - y_0\| < d(X, Y).
\]

Por lo tanto, para algún \( k_0 \in \mathbb{N} \), se cumple:
\[
\|x_{k_0} - y_{k_0}\| < d(X, Y),
\]
con \( x_{k_0} \in X \), \( y_{k_0} \in Y \), lo cual contradice la definición de \( d(X, Y) \) como el ínfimo de todas las distancias entre puntos de \( X \) y \( Y \).

\vspace{1ex}
\textbf{Conclusión:} La suposición lleva a contradicción, por lo tanto:
\[
d(\overline{X}, \overline{Y}) \geq d(X, Y),
\]
y combinando con la desigualdad anterior, se concluye:
\[
d(X, Y) = d(\overline{X}, \overline{Y}).
\]
}

 

\thmrpf{}{}{

}{

}
 
 
 