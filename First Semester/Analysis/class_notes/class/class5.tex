\chapter{Clase 5}
\clasedate{14 de abril de 2025}
\expf{
Sea \( f \colon X \subset \mathbb{R}^m \to \mathbb{R}^n \). La función \( f \) es \emph{uniformemente continua} si, y solo si \(\forall  (x_k)_{k\in \N} , (y_k)_{k\in \N} \subset X \), se cumple que
\[
	\lim_{k \to \infty} \lvert x_k - y_k \rvert = 0 \quad \Rightarrow \quad \lim_{k \to \infty} \lvert f(x_k) - f(y_k) \rvert = 0
\]
}{\\
($\Rightarrow$) \quad Como $f$ es uniformemente continua.\\
Dado $\varepsilon >0, \exists \delta>0, \forall x, y \in X$
\begin{equation}
	\abs{x-y} <\delta \longrightarrow \abs{f(x)-f(y)} <\varepsilon
	\label{eq:class5-1}
\end{equation}
Sean $(x_k)_k, (y_k)_k \subset X$ tales que $ \lim_{k \to \infty}  \abs{f(x_k) - f(y_k)  }= 0$

Para el $\delta>0 $ anterior, $\exists k_0 \in \N, \forall k \in \N$
$$
	k\geq k_0 \rightarrow \abs{x_k-y_k} <\delta
$$
De \eqref{eq:class5-1},
$$
	\rightarrow \abs{f(x_k)-f(y_k)} < \varepsilon
$$
$$
	\therefore \lim_{k \to \infty} \abs{f(x_k)-f(y_k)} = 0
$$

($\Leftarrow$) \quad  Supongamos que $f$ no es uniformemente continua
$$
	\exists \varepsilon_0 >0, \forall \delta>0, \exists x_{\delta}, y_{\delta} \in X,\qquad \abs{x_{\delta}-y_{\delta}} <\delta \wedge \abs{f(x_{\delta})-f(y_{\delta})} \geq \varepsilon_0
$$
$$
	\forall k \in \N, \exists x_k, y_k \in X, 0\geq \abs{x_k-y_k} < \frac{1}{k} \wedge \abs{f(x_{\delta})-f(y_{\delta})} \geq \varepsilon_0
$$
$$
	\rightarrow \lim_{k \to \infty} \abs{f(x_{\delta})-f(y_{\delta})}=0 \wedge  \forall k \in \N, \abs{f(x_{\delta})-f(y_{\delta})} \geq \varepsilon_0
$$
Por hipótesis, $\lim\limits_{k \to \infty} \abs{f(x_{\delta})-f(y_{\delta})}=0$, pero $\lim_{k \to \infty} \abs{f(x_{\delta})-f(y_{\delta})}\geq \varepsilon_0 $\\
$\rightarrow 0\geq \varepsilon_0 >0 $ (Contradicción).
}

\expf{
	Sea \(f\colon X \subset \mathbb{R}^m \to \mathbb{R}^n\) uniformemente continua, si $
		\forall\, (x_k)_{k\in\mathbb{N}} \subset X,$
	si $(x_k)_{k \in \N}$ es de Cauchy, entonces la sucesión
	$	\bigl(f(x_k)\bigr)_{k\in\mathbb{N}} \subset \mathbb{R}^n$
	también es de Cauchy.\\
	¿El recíproco se cumple?
}{
	No se cumple, por contraejemplo.\\
	Sea $g \colon \R \to \R$ continua.\\
	$f(x) = \frac{1}{x}. \quad x \in \R -\{0\}$\\
	$g(x) = x^2$\\
	$g$ es sucesión de Cauchy en sucesión de Cauchy, pero no es uniformemente continua puesto que,

	$$
		x_k=k, y_k=k+\frac{1}{k}, \quad \forall k \in \N
	$$
	$$
		\lim\limits_{k \to \infty} \abs{x_k-y_k} =0 \wedge\lim_{k \to \infty} \abs{f(x_{\delta})-f(y_{\delta})} = 2
	$$
}


\thmrpf{}{}{
	Sea $f \colon X \in \Rm, \; X \to \Rn$ una función, $a \in X'$ y $b \in \Rn$. Tenemos que
	$$
		\lim\limits_{ x \to a} f(x)  = b = f(a)\in \Rn \Leftrightarrow \forall (x_k)_{k } \subset X-\{a\}, \; \lim_{k \to \infty} x_k =a \rightarrow \lim_{k \to \infty} f(x_k) = b = f(a)
	$$
}{
	($\Rightarrow$)\\
	Agregar graficos\\
	Sea $(x_k)_{k \in \N } \subset X-\{a\}$ tal que $\lim\limits_{k \to \infty} =a$

	Como $\lim\limits_{k\to \infty} f(x)=b$, dado $\varepsilon>0, \exists \delta>0$
	\begin{equation}
		\forall x \in X, 0< \abs{x-a} <\delta \rightarrow\abs{f(x)-b} < \varepsilon
		\label{eq:class5-2}
	\end{equation}

	$$
		k\geq k_0 \rightarrow 0< \abs{x_k-a}<\delta \rightarrow \abs{f(x_k)-b}< \varepsilon
	$$
	$$
		\therefore \lim_{k \to \infty} f(x_k) = b
	$$
	($\Leftarrow$)\\
	Supongamos que $\forall (x_k)_{k \in \N } \subset X-\{a\}$ tal que $\lim\limits_{k \to \infty} =a$

	Por contradicción supongtamos que
	$$
		\exists \varepsilon_0 >0, \forall \delta >0, \exists x_{\delta} \in X, 0<\abs{x_{\delta}-a}<\delta \wedge \abs{f(x_{\delta})-b}\geq \varepsilon_0
	$$

	$\forall k \in \N$, considere  $\delta_k = \frac{1}{k}$, $\exists x_k \in X; 0 <\abs{x_k-a} < \frac{1}{k} \wedge \abs{f(x_k)-b} \geq \varepsilon_0$.

	Hemos construido $(x_k)_{k \in \N} \subset X-\{a\}$ tal que $\lim\limits_{k\to \infty} =0$

	Por hipótesis $\rightarrow \lim\limits_{k \to \infty} f(x_k) = b$, pero $\forall k \in \N,  \abs{f(x_k)-b} \geq \varepsilon_0>0$
	$$
		\equiv \lim\limits_{k \to \infty} \abs{f(x_k)-b} =0 \qquad \rightarrow 0 \geq \varepsilon_0>0 \;(\rightarrow \; \leftarrow)
	$$
}

\propp{
	Sean \( f \colon X \subset \mathbb{R}^m \to \mathbb{R}^n \), \( a \in X' \) (punto de acumulación de \( X \)) y \( b \in \mathbb{R}^n \) tales que
	\[
		\lim_{x \to a} f(x) = b.
	\]
	Sea \( g \colon Y \subset \mathbb{R}^n \to \mathbb{R}^p \) una función continua en \( b \in Y \), y supongamos que \( f(X) \subset Y \), lo cual garantiza que la composición \( g \circ f \colon X \to \mathbb{R}^p \) está bien definida.

	Entonces,
	\[
		\lim_{x \to a} g(f(x)) = g(b).
	\]

}{
	Como \( g \) es continua en \( b \in Y \), dado \( \varepsilon > 0 \), existe \( \eta > 0 \) tal que
	\begin{equation}
		\lvert y - b \rvert < \eta \quad \Rightarrow \quad \lvert g(y) - g(b) \rvert < \varepsilon, \quad \forall y \in Y.
		\label{eq:class5-3}
	\end{equation}

	Por otro lado, como \( \lim\limits_{x \to a} f(x) = b \), para el \( \eta > 0 \) anterior, existe \( \delta > 0 \) tal que
	\[
		0 < \lvert x - a \rvert < \delta \quad \Rightarrow \quad \lvert f(x) - b \rvert < \eta, \quad \text{con } x \in X.
	\]
	Además, como \( f(X) \subset Y \), se cumple \( f(x) \in Y \) para todo \( x \in X \), por lo que podemos aplicar \eqref{eq:class5-3} y obtener:
	\[
		\lvert g(f(x)) - g(b) \rvert < \varepsilon.
	\]

	Por lo tanto,
	\[
		\lim_{x \to a} g(f(x)) = g\left( \lim_{x \to a} f(x) \right) = g(b),
	\]
}



\thmrpf{}{}{
	Sea
	\begin{align*}
		f \colon X \subset \mathbb{R}^m & \longrightarrow \mathbb{R}^n                                                                                          \\
		x                               & \longmapsto f(x) = (f_1(x), f_2(x), \ldots, f_n(x)) \quad \text{y} \quad b = (b_1, b_2, \ldots, b_n) \in \mathbb{R}^n
	\end{align*}
	Tenemos que
	$$
		\lim_{x \to a} f(x) = b \quad \Leftrightarrow \quad \forall i \in \{1, 2, \ldots, n\}, \quad \lim_{x \to a} f_i(x) = b_i
	$$

}{  ($\Rightarrow$) Sea \( f = (f_1, f_2, \dots, f_n) \colon X \to \mathbb{R}^n \), con \( f_i : X \to \mathbb{R} \) para cada \( i \in \{1, 2, \dots, n\} \), y supongamos que \( \lim_{x \to a} f(x) = b \).
	Como cada
	\begin{align*}
		\pi_i \colon \mathbb{R}^n & \longrightarrow \mathbb{R} \\
		(x_1, \dots, x_n)         & \longmapsto x_i
	\end{align*}
	es una proyección, y las proyecciones son funciones continuas, podemos concluir que
	\[
		\lim_{x \to a} \pi_i(f(x)) = \pi_i(b), \quad \forall i \in \{1, \dots, n\}
	\]
	En particular, esto implica que
	\[
		f_i(x) \to b_i \quad \text{cuando} \quad x \to a
	\]

	\noindent($\Leftarrow$) Sea \( (x_k)_{k \in \mathbb{N}} \subset X \setminus \{a\} \) tal que \( \lim_{k \to \infty} x_k = a \).
	Como \( \lim_{x \to a} f_i(x) = b_i \) para cada \( i \in \{1, \dots, n\} \), se cumple que
	\[
		\lim_{k \to \infty} f_i(x_k) = b_i, \quad \forall i \in \{1, \dots, n\}
	\]
	Por lo tanto, tenemos que
	\[
		\lim_{k \to \infty} \left( f_1(x_k), \dots, f_n(x_k) \right) = (b_1, \dots, b_n),
	\]
	lo que implica que
	\[
		\lim_{k \to \infty} f(x_k) = b
	\]
}

\rmkb{
	Sea \( z_k = (z_1^k, \ldots, z_n^k) \in \mathbb{R}^n \) y \( a = (a_1, \ldots, a_n) \in \mathbb{R}^n \). Entonces,
	\[
		\lim_{k \to \infty} z_k = a \quad \Longleftrightarrow \quad \lim_{k \to \infty} z_i^k = a_i, \quad \forall i \in \{1, \ldots, n\}.
	\]
}

\propp{
	Sean \( X \subset \mathbb{R}^m \), \( a \in X' \), y funciones \( f, g \colon X \to \mathbb{R}^n \), \( \alpha \colon X \to \mathbb{R} \) tales que:
	\[
		\lim_{x \to a} f(x) = b \in \mathbb{R}^n, \quad
		\lim_{x \to a} g(x) = c \in \mathbb{R}^n, \quad
		\lim_{x \to a} \alpha(x) = \alpha_0 \in \mathbb{R}.
	\]
	Entonces, se cumplen las siguientes propiedades:
	\begin{description}
		\item[(1)] \( \displaystyle \lim_{x \to a} \left[ f(x) + g(x) \right] = b + c. \)
		\item[(2)] \( \displaystyle \lim_{x \to a} \left[ \alpha(x) f(x) \right] = \alpha_0 b. \)
		\item[(3)] \( \displaystyle \lim_{x \to a} \left\langle f(x), g(x) \right\rangle = \left\langle b, c \right\rangle. \)
	\end{description}
}{
	\begin{description}
		\item[(1)] Definimos la función suma
		      \[
			      S \colon \mathbb{R}^n \times \mathbb{R}^n \to \mathbb{R}^n, \quad S(x,y) = x + y.
		      \]
		      Dado que \( S \) es una función lineal (y por tanto continua), basta verificar que el par \( (f(x), g(x)) \to (b, c) \) cuando \( x \to a \). Definimos:
		      \[
			      (f,g) \colon X \to \mathbb{R}^n \times \mathbb{R}^n, \quad x \mapsto (f(x), g(x)).
		      \]
		      Entonces:
		      \[
			      \lim_{x \to a} (f(x), g(x)) = (b, c) \Rightarrow \lim_{x \to a} S(f(x), g(x)) = S(b, c) = b + c.
		      \]

		\item[(2)] Definimos la función producto escalar:
		      \[
			      \lambda \colon \mathbb{R} \times \mathbb{R}^n \to \mathbb{R}^n, \quad \lambda(t, x) = t x.
		      \]
		      Esta función es bilineal, por lo tanto continua. Como
		      \[
			      (\alpha(x), f(x)) \to (\alpha_0, b) \text{ cuando } x \to a,
		      \]
		      entonces
		      \[
			      \lim_{x \to a} \alpha(x) f(x) = \lambda(\alpha_0, b) = \alpha_0 b.
		      \]

		\item[(3)] Sea el producto interno:
		      \[
			      P \colon \mathbb{R}^n \times \mathbb{R}^n \to \mathbb{R}, \quad P(x, y) = \left\langle x, y \right\rangle.
		      \]
		      \( P \) es bilineal y, por tanto, continua. Como \( (f(x), g(x)) \to (b, c) \), se tiene:
		      \[
			      \lim_{x \to a} \left\langle f(x), g(x) \right\rangle = \left\langle b, c \right\rangle.
		      \]
	\end{description}

	\rmkb{
		Este resultado se basa en la **continuidad de funciones bilineales y lineales** en espacios vectoriales normados, como suma, producto por escalar y producto interno.
	}

	\vspace{1em}
	\textbf{Extensión de funciones al dominio cerrado:}

	Si \( f \colon X \to \mathbb{R}^n \) es tal que \( \lim_{x \to a} f(x) \) existe para todo \( a \in \overline{X} \), entonces se puede definir una extensión continua
	\[
		\hat{f} \colon \overline{X} \to \mathbb{R}^n, \quad \hat{f}(a) = \lim_{x \to a} f(x), \text{ donde } (x_k) \subset X, \; x_k \to a.
	\]
	Esta extensión es única cuando existe el límite para todo \( a \in \overline{X} \).
}



\thmrpf{}{}{
	Sea \( f \colon X \subset \mathbb{R}^m \rightarrow \mathbb{R}^n \) una función uniformemente continua. Si \( a \in X' \) (es decir, \( a \) es un punto de adherencia de \( X \)), entonces
	\[
		\lim_{x \to a} f(x) = w_0
	\]
	existe.
}{
	Como \( f \) es uniformemente continua, se cumple que toda sucesión de Cauchy en \( X \) se transforma en una sucesión de Cauchy en \( \mathbb{R}^n \). Es decir  $\forall (x_k)_k \subset X$ si $ (x_k)_{k \in \N}$ es de Cauchy, $\rightarrow \pqty{f(x_k)_{k \in \N}} $ es de Cauchy.

	Sea \( a \in X' \). Entonces, existe una sucesión \( (z_k)_{k \in \mathbb{N}} \subset X \setminus \{a\} \) tal que
	\[
		\lim_{k \to \infty} z_k = a.
	\]
	Dado que toda sucesión convergente es de Cauchy, \( (z_k) \) es de Cauchy en \( X \). Por la uniformidad de \( f \),
	\[
		(f(z_k))_{k \in \mathbb{N}} \text{ es de Cauchy en } \mathbb{R}^n.
	\]
	Como \( \mathbb{R}^n \) es completo, la sucesión \( (f(z_k)) \) converge. Es decir, existe \( w_0 \in \mathbb{R}^n \) tal que
	\[
		\lim_{k \to \infty} f(z_k) = w_0.
	\]

	Queremos mostrar que el límite es único, es decir, que si \( (x_k)_{k \in \mathbb{N}} \subset X \setminus \{a\} \) con \( \lim_{k \to \infty} x_k = a \), entonces
	\[
		\lim_{k \to \infty} f(x_k) = w_0.
	\]

	Tomemos una sucesión arbitraria \( (x_k) \subset X \setminus \{a\} \) tal que \( \lim_{k \to \infty} x_k = a \). Entonces, \( (x_k) \) también es de Cauchy, y por la uniformidad de \( f \),
	\[
		(f(x_k))_{k \in \mathbb{N}} \text{ es de Cauchy en } \mathbb{R}^n.
	\]
	Nuevamente, por completitud de \( \mathbb{R}^n \), existe \( w_1 \in \mathbb{R}^n \) tal que
	\[
		\lim_{k \to \infty} f(x_k) = w_1.
	\]

	Como \( \lim x_k = \lim z_k = a \), se tiene que \( \lim_{k \to \infty} \lVert x_k - z_k \rVert = 0 \). Por uniformidad de \( f \), esto implica que
	\[
		\lim_{k \to \infty} \lVert f(x_k) - f(z_k) \rVert = 0.
	\]
	Luego,
	\[
		\lVert w_1 - w_0 \rVert = 0 \Rightarrow w_1 = w_0.
	\]

	Por lo tanto, para toda sucesión \( (x_k) \subset X \setminus \{a\} \) que converge a \( a \), se cumple que \( f(x_k) \to w_0 \). Así, concluimos que
	\[
		\lim_{x \to a} f(x) = w_0.
	\]

	\rmkb{
		El resultado depende esencialmente de la \textbf{uniforme continuidad} de \( f \), ya que garantiza que imágenes de sucesiones de Cauchy sigan siendo de Cauchy. Esto no es necesariamente cierto si \( f \) solo fuera continua.
	}
}

\section{Punto interior}

Sea \( X \subset \mathbb{R}^n \).
\begin{itemize}
	\item Se dice que \( a \in X \) es un **punto interior** de \( X \) si
	      \[
		      \exists\, \delta > 0 \text{ tal que } B(a, \delta) \subset X.
	      \]

	\item El interior  de \( X \) es el conjunto de todos sus puntos interiores de $X$. Se denota por:
	      \begin{align*}
		      \interior{X} & = \{a \in X \colon a \text{ es punto interior de } X \}                                 \\
		                   & = \{a \in X \colon \exists\, \delta_a > 0 \text{ tal que } B(a, \delta_a) \subset X \}.
	      \end{align*}
	      Además, se cumple que \( \interior{X} \subset X \).

	\item El conjunto \( X \subset \mathbb{R}^n \) es \textbf{abierto} si todos sus puntos son interiores de \( X \), es decir,
	      \[
		      X = \interior{X}.
	      \]
	      \pf{
		      Supongamos que \( X \) es abierto. Entonces, por definición, todo punto \( a \in X \) es un punto interior, es decir, existe \( \delta_a > 0 \) tal que \( B(a, \delta_a) \subset X \).

		      Por lo tanto,
		      \[
			      X = \bigcup_{a \in X} B(a, \delta_a),
		      \]
		      es decir, \( X \) se puede expresar como unión de bolas abiertas contenidas en \( X \).

		      Recíprocamente, si existe \( \delta_a > 0 \) para cada \( a \in X \) tal que \( B(a, \delta_a) \subset X \), entonces todo punto de \( X \) es interior, por lo tanto \( X \subset \interior{X} \). Como siempre se tiene \( \interior{X} \subset X \), se concluye que \( X = \interior{X} \), y por tanto \( X \) es abierto.
	      }

\end{itemize}


\expf{
	Sea \( \delta > 0 \) y \( a \in \mathbb{R}^n \). Definimos la bola abierta como
	\[
		B(a,\delta) =  \{ z \in \mathbb{R}^n \colon \abs{z-a} < \delta \}.
	\]
	Sea \( x \in B(a, \delta) \). Mostremos que \( x \) es un punto interior de \( B(a,\delta) \); es decir, existe \( r > 0 \) tal que \( B(x, r) \subset B(a, \delta) \).
}{
	\begin{figure}[H]
		\centering
		% Aquí deberías incluir un dibujo que represente las bolas \( B(a,\delta) \) y \( B(x, r) \subset B(a,\delta) \)
		\caption{Representación de \( B(x, r) \subset B(a, \delta) \)}\label{figsa}
	\end{figure}

	Como \( x \in B(a,\delta) \), se tiene \( \abs{x - a} < \delta \). Definimos
	\[
		r = \delta - \abs{x - a} > 0.
	\]
	Afirmamos que \( B(x, r) \subset B(a, \delta) \), lo que probará que \( x \) es un punto interior.

	Sea \( w \in B(x, r) \), es decir, \( \abs{w - x} < r \). Entonces,
	\[
		\abs{w - a} \leq \abs{w - x} + \abs{x - a} < r + \abs{x - a} = \delta.
	\]
	Por lo tanto, \( w \in B(a, \delta) \), y se concluye que \( B(x, r) \subset B(a, \delta) \).
}


\corp{
	Dado \( X \subset \mathbb{R}^n \), se tiene que \( \interior{X} \) es un conjunto abierto.
}{
	Sea \( x \in \interior{X} \). Por definición de interior, existe \( \delta > 0 \) tal que
	\[
		B(x, \delta) \subset X.
	\]
	Sea \( w \in B(x, \delta) \). Como \( B(x, \delta) \) es abierto (toda bola abierta es abierta), existe \( r_w > 0 \) tal que
	\[
		B(w, r_w) \subset B(x, \delta).
	\]
	Por lo tanto,
	\[
		B(w, r_w) \subset B(x, \delta) \subset X,
	\]
	lo que implica que \( w \in \interior{X} \). Es decir, todo punto \( w \in B(x, \delta) \) pertenece a \( \interior{X} \), lo cual muestra que
	\[
		B(x, \delta) \subset \interior{X}.
	\]
	Así, \( x \) es punto interior de \( \interior{X} \), y como esto es cierto para todo \( x \in \interior{X} \), concluimos que
	\[
		\interior{X} \text{ es abierto}, \quad \text{es decir, } \interior{\interior{X}} = \interior{X}.
	\]
}

\begin{itemize}
	\item Sea \( X \subset \mathbb{R}^n \). Un punto \( b \in \mathbb{R}^n \) se dice \textbf{punto de adherencia}  de \( X \) si existe una sucesión \( (x_k)_{k \in \mathbb{N}} \subset X \) tal que
	      \[
		      \lim_{k \to \infty} x_k = b.
	      \]
	      \notatio{
		      La \textbf{cerradura} de un conjunto \( X \subset \mathbb{R}^n \) es el conjunto
		      \[
			      \overline{X} = \left\{ b \in \mathbb{R}^n : b \text{ es punto de adherencia de } X \right\}.
		      \]
	      }

	\item Un conjunto \( X \subset \mathbb{R}^n \) es \textcolor{red}{cerrado} si se cumple que \( X = \overline{X} \).

	      \rmkb{
		      Siempre se tiene \( X \subset \overline{X} \). Además, \( X = \overline{X} \) si y solo si \( \overline{X} \subset X \).
	      }

	\item Fijado \( X \subset \mathbb{R}^n \), decimos que un punto \( a \in \mathbb{R}^n \) es un  \textbf{punto de la frontera}  (o \textbf{borde}) de \( X \) si
	      \[
		      a \in \mathbb{R}^n \setminus \left( \interior{X} \cup \interior{\mathbb{R}^n \setminus X} \right)
	      \]

	      \notatio{
		      La \textbf{frontera} de un conjunto \( X \subset \mathbb{R}^n \) se denota por
		      \[
			      \partial X
		      \]
	      }

	      \pf{
		      \begin{align*}
			      a \in \partial X
			       & \leftrightarrow a \notin \interior{X} \text{ y } a \notin \interior{\mathbb{R}^n \setminus X}                                                    \\
			       & \leftrightarrow \forall \delta > 0, \; B(a, \delta) \cap X \neq \emptyset \text{ y } B(a, \delta) \cap (\mathbb{R}^n \setminus X) \neq \emptyset \\
			       & \leftrightarrow a \in \overline{X} \cap \overline{\mathbb{R}^n \setminus X}
		      \end{align*}
		      \[
			      \therefore \quad \partial X = \overline{X} \cap \overline{\mathbb{R}^n \setminus X}
		      \]
	      }
\end{itemize}

